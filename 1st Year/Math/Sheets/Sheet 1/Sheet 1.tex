\documentclass{article}
\usepackage{amsmath,amssymb,amsthm,amsfonts,fullpage}

\begin{document}

\begin{flushright}
Kris Harper

MATH 16100

Mikl\'{o}s Ab\'{e}rt

October 2, 2007
\end{flushright}

\begin{flushleft}

\Large

Sheet 1: Basics\newline

\normalsize

\textbf{Definition 1 (Empty Set)}
\textsl{The empty set is denoted by $\emptyset$; it contains no elements}\newline

\textbf{Definition 2 (Element)}
\textsl{Instead of saying ``$A$ contains $a$,'' we say that $a$ is an element of $A$, and write this as $a \in A$. For the converse statement, that $a$ is not an element of $A$, we write $a \notin A$.}\newline

\textbf{Exercise 3}
\textsl{Is it true that every element of the empty set is a whistling, flying purple cow?}\newline

Yes. Since the empty set has no elements, it is vacuously true that all elements are whistling, flying purple cows.\newline

\textbf{Definition 4 (Subset)}
\textsl{Let $A$ and $B$ be sets. If each element of $A$ is also an element of $B$, we say that $A$ is a subset of $B$. In symbols $A \subseteq B$.}\newline

\textbf{Exercise 5}
\textsl{How many subsets does the empty set have?}
\begin{proof}
Let $A \subseteq \emptyset$. Then every element of $A$ is in $\emptyset$. But $\emptyset$ has no elements so $A$ must have no elements. Therefore $A = \emptyset$. So the empty set has only one subset, itself.
\end{proof}

\textbf{Exercise 6}
\textsl{Let $A_n=\{1,2,...,n\}$. How many subsets does $A_n$ have?}\newline

$A_n$ has $2^n$ subsets.
\begin{proof}
We use induction on $n$. We see that the statement is true for $n=1$ since $\{1\}$ has one element and its only subsets are $\{1\}$ and $\emptyset$ and $2=2^1$. Let $S_k$ be the set $\{1,2,3,...,k\}$. Then we assume $S_k$ has $2^k$ subsets and show that $S_{k+1}$ has $2^{k+1}$ subsets. Consider a set $A$ such that $A \subseteq S_{k+1}$. Then we see that either $(k+1) \in A$ or $(k+1) \notin A$. Let $(k+1) \notin A$. Then $A \subseteq S_k$. But then there exists a set $A \cup \{k+1\} \subseteq S_{k+1}$ for every $A \subseteq S_k$. Therefore, for every subset $A$ of $S_k$, $A$ and $A \cup \{k+1\}$ are subsets of $S_{k+1}$. Thus, $S_{k+1}$ has at least $2 \cdot 2^k=2^{k+1}$ subsets since there are $2^k$ subsets of $S_k$.\newline

But suppose there are more than $2^{k+1}$ subsets of $S_{k+1}$ Then there exists a subset $B$ of $S_{k+1}$ such that $B \nsubseteq S_k$ and $B \backslash \{k+1\} \nsubseteq S_k$. Then there exists a $b \in B$ such that $b \notin S_k$ and $b \neq k+1$. Since $B \subseteq S_{k+1}$, $b \in S_{k+1}$. But $S_k \cap S_{k+1} = S_k$ which means that $S_{k+1}$ contains every element in $S_k$ as well as $k+1$. Therefore, $b \in S_k$ or $b = k+1$. This is a contradiction. Therefore, $S_{k+1}$ must have exactly $2^{k+1}$ subsets.
\end{proof}

\textbf{Exercise 7}
\textsl{What is the number of subsets of $A_n$ that contain exactly 2 elements?}\newline

$\binom{n}{2} = \frac{n(n-1)}{2}$ subsets.\newline

\begin{proof}
We note that in $A_1 = \{1\}$ there is only one element and so there are no subsets with $2$ elements. So the theorem holds for $n=1$ since $\frac{1(1-1)}{2}=0$. We now assume that $A_{n}$ has $\frac{n(n-1)}{2}$ subsets of size $2$ and use induction on $n$. The set $A_{n+1}$ has $n+1$ elements and $A_{n+1} \backslash A_n = \{n+1\}$. So for every $k \in A_n$ there exists a subset $\{k,n+1\} \subseteq A_{n+1}$. Since there are $n$ elements in $A_n$ we have $n$ more subsets of size $2$ in $A_{n+1}$.\newline

Now suppose there are more than $n$ subsets of size $2$ added to $A_{n+1}$. Then there exists some subset $\{a,b\} \subseteq A_{n+1}$ which we have not yet considered. But then it is not the case that $\{a,b\} \subseteq A_n$ and it is not the case that one element of $\{a,b\}$ is in $A_n$ and the other is $n+1$. Thus, because $A_{n+1} = A_n \cup \{n+1\}$, we see that $a,b \notin A_{n+1}$ and this is a contradiction. Therefore there are exactly $n$ subsets of size $2$ added to $A_{n+1}$.\newline

So there are $\frac{n(n-1)}{2} + n = \frac{n^2-n+2n}{2} = \frac{n(n+1)}{2} = \frac{(n+1)(n+1-1)}{2}$ subsets of $A_{n+1}$ of size $2$. So the statement holds for $n=1$ and $n+1$ when it holds for $n$ so it must hold for all $n \in \mathbb{N}$.
\end{proof}

\textbf{Definition 8 (Union, Intersection, Difference and Direct Product)}
\textsl{If $A$ and $B$ are sets then:
\[
A \cup B = \{x \mid x \in A \text{ or } x \in B\}
\]
the union of $A$ and $B$;
\[
A \cap B = \{x \mid x \in A \text{ and } x \in B\}
\]
the intersection of $A$ and $B$ and
\[
A \backslash B = \{x \mid x \in A \text{ and } x \notin B\}
\]
the difference of $A$ and $B$. If $B = \{b\}$ is the set consisting of a single element $b$, we will write $A \backslash b$ rather than $A \backslash \{b\}$. Finally
\[
A \times B = \{(a,b) \mid a \in A \text{ and } b \in B\}
\]
the set of ordered pairs from $A$ and $B$. The set $A \times B$ is called the direct product of $A$ and $B$.}\newline

\textbf{Definition 9 (Union and Intersection of Many Sets)}
\textsl{Let $S$ be a set consisting of sets. Then the intersection and union of $S$ is defined as follows:
\[
\bigcup_{A \in S} A = \{x \mid \text{there exists } A \in S \text{ such that } x \in A\}
\]
and
\[
\bigcap_{A \in S} A = \{x \mid \text{for all } A \in S \text{ we have } x \in A\}.
\]} 

\textbf{Theorem 10}
\textsl{Let $X$ be a set and let $S$ be a set consisting of subsets of $X$. Then
\[
X \backslash \left( \bigcup_{A \in S} A \right) = \bigcap_{A \in S} \left(X \backslash A\right)
\]
and
\[
\bigcup_{A \in S} \left(X \backslash A\right)=X \backslash \left( \bigcap_{A \in S} A \right).
\]}

\begin{proof}
Let $X$ be a set and let $S$ be a set consisting of subsets of $X$.\newline

Let $a \in X \backslash \left( \bigcup_{A \in S} A \right)$. Then $a \in X$ and $a \notin \left( \bigcup_{A \in S} A \right)$. That is, $a \notin A$ for all sets $A \in S$. Thus, for all sets $A \in S$, $a \in X \backslash A$. Since this is true for all sets $A \in S$ we may write $x \in \bigcap_{A \in S} \left( X \backslash A \right)$. Therefore $X \backslash \left( \bigcup_{A \in S} A \right) \subseteq \bigcap_{A \in S} \left( X \backslash A \right)$.\newline

Let $a \in \bigcap_{A \in S} \left(X \backslash A \right)$. Thus, $a \in X \backslash A$ for every set  $A \in S$. Since $a \notin A$ for all $A \in S$ we can state that $a \notin \bigcup_{A \in S} A$. But since $a \in X$, we can now write $a \in X \backslash \left( \bigcup_{A \in S} A \right)$. Therefore $\bigcap_{A \in S} \left( X \backslash A \right) \subseteq X \backslash \left( \bigcup_{A \in S} A \right)$.\newline

Let $a \in \bigcup_{A \in S} \left( X \backslash A \right)$. Thus, $a \in X \backslash A$ for at least one $A \in S$. Since $ a \notin A$ for at least one $A \in S$, then we can write $ a \notin \bigcap_{A \in S} A$. But since $a \in X$, we can now write $a \in X \backslash \left( \bigcap_{A \in S} A \right)$. Therefore $\bigcup_{A \in S} \left( X \backslash A \right) \subseteq X \backslash \left( \bigcap_{A \in S} A \right)$.\newline

Let $a \in X \backslash \left( \bigcap_{A \in S} A \right)$. Thus, $a \in X$, but $a \notin A$ for at least one $A \in S$. In other words, $a \in X \backslash A$ for at least one $A \in S$. Thus we may write $a \in \bigcup_{A \in S} \left(X \backslash A \right)$. Therefore $X \backslash \left( \bigcap_{A \in S} A \right) \subseteq \bigcup_{A \in S} \left( X \backslash A \right)$.\newline

Since we have shown both inclusions for both of De Morgan's Laws, we can conclude the result is true.
\end{proof}

\textbf{Definition 11 (Function)}
\textsl{A function $f \; : \; A \rightarrow B$ is defined as a subset of $A \times B$ such that for all $a \in A$ there exists a unique $b \in B$ with $(a,b) \in f$. Instead of $(a,b) \in f$ we will use the notation $f(a) = b$.}\newline

\textbf{Definition 12 (Domain and Range)}
\textsl{The domain of $f$ is $A$. The range of $f$ (image under $f$) is
\[
f(A) = \{f(a) \mid a \in A\}.
\]}

\textbf{Definition 13 (Surjective, Injective and Bijective)}
\textsl{A function $f \; : \; A \rightarrow B$ is surjective (onto) if $f(A) = B$. It is injective (1 to 1) if for all $a_1,a_2 \in A$, if $f(a_1)=f(a_2)$ then $a_1=a_2$. It is bijective if it is surjective and injective.}\newline

\textbf{Definition 14 (Inverse Function)}
\textsl{Let $f \; : \; A \rightarrow B$ be a bijection. Then the inverse of $f$, $f^{-1} \; : \; B \rightarrow A$ is defined by
\[
(b,a) \in f^{-1} \text{ if and only if } (a,b) \in f.
\]}

\textbf{Definition 15 (Image and Preimage)}
\textsl{Let $f \; : \; A \rightarrow B$ be a function. Let $X \subseteq A$. Then the image of $X$ under $f$ is
\[
f(X) = \{f(x) \mid x \in A\}.
\]
Let $Y \subseteq B$. Then the preimage of $Y$ under $f$ is
\[
f^{-1}(Y) = \{x \in A \mid f(x) \in Y\}.
\]}

\textbf{Exercise 16}
\textsl{$f^{-1}(f(X)) = X$ for all $X \subseteq A$.}\newline

False. Let $A = \{1,2\}$ and $B = \{3\}$ such that $f(1)=f(2)=3$. Then $f(\{1\}) = \{3\}$. But $f^{-1}(\{3\})=\{1,2\} \neq \{1\}$. We can prove the statement if we assume $f$ is injective.

\begin{proof}
Let $f\, : \, A \rightarrow B$ be an injective function. Let $X \subseteq A$. Suppose $a \in f^{-1}(f(X))$. Then $a \in A$ and $f(a) \in f(X)$ which means there exists some $b \in X$ such that $f(b)=f(a)$. But $f$ is injective and so $a=b$ and $a \in X$. Therefore $f^{-1}(f(X)) \subseteq X$. Now suppose $a \in X$. Then $f(a) \in f(X)$ and $a \in f^{-1}(f(X))$. Thus $X \subseteq f^{-1}(f(X))$. Since both sets are subsets of each other, they are equal.
\end{proof}

\textbf{Exercise 17}
\textsl{$f(f^{-1}(Y))=Y$ for all $Y \subseteq B$.}\newline

False. Let $A = \{1,2\}$ and $B = \{3,4\}$ such that $f(1)=f(2)=3$. Then $f^{-1}(B) = \{1,2\}$. But $f(\{1,2\}) = \{3\} \neq B$. We can prove the statement if we assume $f$ is surjective.

\begin{proof}
Let $f\, : \, A \rightarrow B$ be a surjective function. Let $Y \subseteq B$. Suppose $b \in f(f^{-1}(Y))$. Then there exists an $a \in f^{-1}(Y)$ such that $f(a) = b$. Then $a \in f^{-1}(Y)$ and so $f(a) \in Y$. Thus $b \in Y$ and so $f(f^{-1}(Y)) \subseteq Y$. Now let $b \in Y$. Since $f$ is surjective, $f(A) = B$ and so there exists an element $a \in A$ such that $f(a) = b$. Thus $f(a) \in Y$. Then $a \in f^{-1}(Y)$ and $f(a) \in f(f^{-1}(Y))$. Thus, $Y \subseteq f(f^{-1}(Y))$. Again, since both sets are subsets of each other, they must be equal.
\end{proof}

\textbf{Exercise 18}
\textsl{$f(X_1 \cap X_2) = f(X_1) \cap f(X_2)$ for all $X_1$ and $X_2 \subseteq A$.}\newline

False. Let $f \; : \; A \rightarrow B$ be a function such that $X_1=\{1,3\}$ and $X_2 = \{2,3\}$ are subsets of $A$. Let $f(1)=f(2)=10$ and $f(3)=11$. Then we see that $f(X_1) = \{10,11\}$ and $f(X_2) = \{10,11\}$ and so $f(X_1) \cap f(X_2) = \{10,11\}$. But $X_1 \cap X_2 = \{3\}$ and so $f(X_1 \cap X_2) = \{11\}$. We can prove the statement if we assume $f$ is injective.
\begin{proof}
Let $f \, : \, A \rightarrow B$ be an injective function and let $X_1$ and $X_2$ be subsets of $A$. Suppose $b \in f(X_1 \cap X_2)$. Then there exists an $a \in X_1 \cap X_2$ such that $f(a) = b$. Thus, $a \in X_1 \cap X_2$ and so $a \in X_1$ and $a \in X_2$. Therefore $f(a) \in f(X_1)$ and $f(a) \in f(X_2)$ and so $f(a) \in f(X_1) \cap f(X_2)$ and $b \in f(X_1) \cap f(X_2)$. Thus, $f(X_1 \cap X_2) \subseteq f(X_1) \cap f(X_2)$.\newline

Now suppose $a \in f(X_1) \cap f(X_2)$. Then $a \in f(X_1)$ and $a \in f(X_2)$. Then there exists a $b\in X_1$ such that $f(b) = a$ and a $c \in X_2$ such that $f(c) = a$. But since $f$ is injective and $f(b) = f(c)$, then $b=c$ and so $b \in X_1$ and $b \in X_2$. Thus $b \in X_1 \cap X_2$ and so $f(b) \in f(X_1 \cap X_2)$ and $a \in f(X_1 \cap X_2)$. Therefore $f(X_1) \cap f(X_2) \subseteq f(X_1 \cap X_2)$. Since both sets are subsets of each other, they are equal.
\end{proof}

\textbf{Exercise 19}
\textsl{$f^{-1}(Y_1 \cap Y_2) = f^{-1}(Y_1) \cap f^{-1}(Y_2)$ for all $Y_1$ and $Y_2 \subseteq B$.}
\begin{proof}
Let $f \, : \, A \rightarrow B$ be a function and let $Y_1$ and $Y_2$ be subsets of $B$. Let $b \in f^{-1}(Y_1 \cap Y_2)$. Then $f(b) \in Y_1 \cap Y_2$ and so $f(b) \in Y_1$ and $f(b) \in Y_2$. Thus $b \in f^{-1}(Y_1)$ and $b \in f^{-1}(Y_2)$ and so $b \in f^{-1}(Y_1) \cap f^{-1}(Y_2)$. Therefore $f^{-1}(Y_1 \cap Y_2) \subseteq f^{-1}(Y_1) \cap f^{-1}(Y_2)$.\newline

Now let $b \in f^{-1}(Y_1) \cap f^{-1}(Y_2)$. Then $b \in f^{-1}(Y_1)$ and $b \in f^{-1}(Y_2)$. Thus $ f(b) \in Y_1$ and $f(b) \in Y_2$ and so $f(b) \in Y_1 \cap Y_2$. Therefore $b \in f^{-1}(Y_1 \cap Y_2)$. Thus $f^{-1}(Y_1) \cap f^{-1}(Y_2) \subseteq f^{-1}(Y_1 \cap Y_2)$. Since both sets are subsets of each other, they are equal.
\end{proof}

\textbf{Exercise 20}
\textsl{$f (X_1 \cup X_2) = f(X_1) \cup f(X_2)$ for all $X_1, X_2 \subseteq A$.}
\begin{proof}
Let $f \, : \, A \rightarrow B$ be a function and let $X_1$ and $X_2$ be subsets of $A$. Let $a \in f (X_1 \cup X_2)$. Then there exists a $b \in X_1 \cup X_2$ such that $f(b)=a$. Then $b \in X_1 \cup X_2$ which means $b \in X_1$ or $b \in X_2$. Thus $f(b) \in f(X_1)$ or $f(b) \in f(X_2)$ and so $f(b) \in f(X_1) \cup f(X_2)$ and $a \in f(X_1) \cup f(X_2)$. Therefore $f (X_1 \cup X_2) \subseteq f(X_1) \cup f(X_2)$.\newline

Now let $a \in f(X_1) \cup f(X_2)$. Then $a \in f(X_1)$ or $a \in f(X_2)$. So there exists a $b \in X_1$ and $c \in X_2$ such that $a=f(b)=f(c)$ and $f(b) \in f(X_1)$ or $f(c) \in f(X_2)$. Thus $b \in X_1$ or $c \in X_2$. Therefore $b,c \in X_1 \cup X_2$ and so $f(b),f(c) \in f(X_1 \cup X_2)$ and $a \in f(X_1 \cup X_2)$. Thus $f(X_1) \cup f(X_2) \subseteq f(X_1 \cup X_2)$. Since both sets are subsets of each other, they are equal.
\end{proof}

\textbf{Exercise 21}
\textsl{$f^{-1} (Y_1 \cup Y_2) = f^{-1} (Y_1) \cup f^{-1} (Y_2)$ for all $Y_1, Y_2 \subseteq B$.}
\begin{proof}
Let $f \, : \, A \rightarrow B$ be a function and let $Y_1$ and $Y_2$ be subsets of $B$. Let $a \in f^{-1} (Y_1 \cup Y_2)$. Then $f(a) \in Y_1 \cup Y_2$ which means $f(a) \in Y_1$ or $f(a) \in Y_2$. Thus $a \in f^{-1} (Y_1)$ or $a \in f^{-1} (Y_2)$ and so $a \in f^{-1} (Y_1) \cup f^{-1} (Y_2)$. Therefore $f^{-1} (Y_1 \cup Y_2) \subseteq f^{-1} (Y_1) \cup f^{-1} (Y_2)$.\newline

Now let $a \in f^{-1} (Y_1) \cup f^{-1} (Y_2)$. Thus $a \in f^{-1} (Y_1)$ or $a \in f^{-1} (Y_2)$ which means $f(a) \in Y_1$ or $f(a) \in Y_2$. Therefore $f(a) \in Y_1 \cup Y_2$ and so $a \in f^{-1} (Y_1 \cup Y_2)$. Thus $f^{-1} (Y_1) \cup f^{-1} (Y_2) \subseteq f^{-1} (Y_1 \cup Y_2)$. Since both sets are subsets of each other, they are equal.
\end{proof}

\textbf{Theorem 22}
\textsl{$f(f^{-1}(f(X))) = f(X)$ for all $X \subseteq A$.}
\begin{proof}
Let $f \; : \; A \rightarrow B$ be a function and let $X \subset A$. Let $a \in f(f^{-1}(f(X)))$. Then there exists some $b \in f^{-1}(f(X))$ such that $a = f(b)$. Since $b \in f^{-1}(f(X))$ we have $f(b) \in f(X)$ and so $a \in f(X)$. Thus $f(f^{-1}(f(X))) \subseteq f(X)$. Now let $a \in f(X)$. Then there exists a $b \in X$ such that $f(b) = a$. Since $f(b) \in f(X)$, $b \in f^{-1}(f(X))$. But then $f(b) \in f(f^{-1}(f(X)))$. Thus $X \subseteq f(f^{-1}(f(X)))$. Since the sets are subsets of each other, they are equal.
\end{proof}

\textbf{Theorem 23}
\textsl{$f^{-1}(f(f^{-1}(Y))) = f^{-1}(Y)$ for all $Y \subseteq B$.}\newline
\begin{proof}
Let $f \; : \; A \rightarrow B$ be a function and let $Y \subseteq B$. Let $a \in f^{-1}(f(f^{-1}(Y)))$. Then $f(a) \in f(f^{-1}(Y))$ and so there exists a $b \in f^{-1}(Y)$ such that $f(a) = f(b)$. But then $f(b) \in Y$ and so $f(a) \in Y$ and $a \in f^{-1}(Y)$. Thus $f^{-1}(f(f^{-1}(Y))) \subseteq f^{-1}(Y)$. Now let $a \in f^{-1}(Y)$. Then $f(a) \in f(f^{-1}(Y))$ and $a \in f^{-1}(f(f^{-1}(Y)))$. Thus, $f^{-1}(Y) \subseteq f^{-1}(f(f^{-1}(Y)))$ and since the sets are subsets of each other, they must be equal.
\end{proof}

\textbf{Problem 24}
\textsl{Try to define the direct product of infinitely many sets.}\newline

For the sets $A_1, A_2, A_3 \dots$, we define the direct product of the sets as
\[
A_1 \times A_2 \times A_3 \dots = \{ (a_1, a_2, a_3, \dots) \mid a_{i} \in A_{i} \; \text{for every} \; i \in \mathbb{N} \}
\]

\textbf{Definition 25 (Equivalence Relation)}
\textsl{Let $A$ be a set. Then $\sim \subseteq A \times A$ is an equivalence relation if the following hold:\newline
1) For all $a \in A$ we have $a \sim a$ (reflexivity);\newline
2) For all $a, b \in A$, if $a \sim b$ then $b \sim a$ (reflexivity);\newline
3) For all $a,b,c \in A$, if $a \sim b$ and $b \sim c$ then $a \sim c$ (transitivity).}\newline

\textbf{Exercise 26}
\textsl{$L$ is the set of lines on the plane. For $a$, $b$ $\in L$ let $a \sim b$ if $a$ and $b$ are parallel.}
\begin{proof}
We see that $a \sim b$ is reflexive because every line $a$ has the same slope as itself and is therefore parallel to itself. We also see that it is symmetric as two lines which are parallel have the same slope and so $a \sim b$ implies $b \sim a$. Additionally, if we take three lines $a$, $b$ and $c$ such that $a$ has the same slope as $b$ and $b$ has the same slope as $c$ then $a$ must have the same slope as $c$ and so $a \sim b$ and $b \sim c$ implies $a \sim c$. Since all three conditions are met, the relation is an equivalence relation.
\end{proof}

\textbf{Exercise 27}
\textsl{For $a$, $b$ $\in L$ let $a \sim b$ if $a$ and $b$ intersect each other.}\newline

This is not an equivalence relation since it fails the transitive property. If we take two lines $a$ and $c$ to be parallel, then a line $b$ may intersect both of them, but it doesn't imply that $a$ intersects $c$.\newline

\textbf{Exercise 28}
\textsl{For $a$, $b$ $\in \mathbb{Z}$ let $a \sim b$ if $a-b$ is even.}
\begin{proof}
We see that $\sim$ is reflexive since for some $a \in \mathbb{Z}$ $a-a=0$ which is even so $a \sim a$. Also, if $a \sim b$ for $a, b \in \mathbb{Z}$ then $a-b=2k$ for some $k \in \mathbb{Z}$ and so $b-a=-2k=2(-k)$. Since $-k \in \mathbb{Z}$, $b-a$ is even and so $b \sim a$. Finally, if $a \sim b$ and $b \sim c$, then $a-b=2k$ and $b-c=2l$ for some $k,l \in \mathbb{Z}$. Then adding the equations we have $a-c=2k+2l=2(k+l)$. Since $k+l \in \mathbb{Z}$, $a-c$ is even and $a \sim c$. Since the relation satisfies all three properties, it is an equivalence relation.
\end{proof}

\textbf{Exercise 29}
\textsl{For $a$, $b$ $\in \mathbb{Z}$ let $a \sim b$ if $a-b$ is odd.}\newline
This relation fails the reflexive test since for all $a \in \mathbb{Z}$, $a-a=0$ which is not odd and so $a \nsim a$.\newline

\textbf{Exercise 30}
\textsl{For $a$, $b$ $\in \mathbb{Z}$ let $a \sim b$ if $a-b$ is divisible by 7.}
\begin{proof}
We see that $\sim$ is reflexive since for some $a \in \mathbb{Z}$ $a-a=0$ which is divisible by 7 so $a \sim a$. Also, if $a \sim b$ for $a, b \in \mathbb{Z}$ then $a-b=7k$ for some $k \in \mathbb{Z}$ and so $b-a=-7k=7(-k)$. Since $-k \in \mathbb{Z}$, $b-a$ is divisible by 7 and so $b \sim a$. Finally, if $a \sim b$ and $b \sim c$, then $a-b=7k$ and $b-c=7l$ for some $k,l \in \mathbb{Z}$. Then adding the equations we have $a-c=7k+7l=7(k+l)$. Since $k+l \in \mathbb{Z}$, $a-c$ is divisible by 7 and $a \sim c$. Since the relation satisfies all three properties, it is an equivalence relation.
\end{proof}

\textbf{Exercise 31}
\textsl{Do 2) and 3 imply 1)?}\newline

No. Take the closed interval $[m;n]$ and for $a,b \in [m;n]$ define $a \sim b$ if there exists a region $R \subseteq [m;n]$ such that $a,b \in R$. Then $a \nsim a$ for some $a \in [m;n]$ because there are no points $p \in [m;n]$ such that $p<m$. Thus every region $(p;q)$ such that $m \in (p;q)$ contains a point $x$ such that $p<x<m$ because regions are nonempty. And so $x \notin [m;n]$ and $(p;q) \nsubseteq [m;n]$. If $a \sim b$ then there exists a region $R \subseteq [m;n]$ such that $a,b \in R$ and so $b,a \in R$ and $b \sim a$. And for $a,b,c \in [m;n]$ if $a \sim b$ and $b \sim c$, then there exist regions $R_1$ and $R_2$ such that $a,b \in R_1$ and $b,c \in R_2$. Then let $R_3 = R_1 \cup R_2$. We see that the region $R_3 \subseteq [m;n]$ and contains $a$ and $c$ so $a \sim c$ (5.5). The proof to show $R_3$ is a region containing $a$ and $c$ is found on Sheet 5.\newline

\textbf{Definition 32 (Equivalence Class)}
\textsl{Let $A$ be a set and let $\sim$ be an equivalence relation on $A$. Then for $a \in A$, the $\sim$-equivalence class of $a$ is defined as:
\[
\overline{a} = \{x \in A \mid a \sim x\}.
\]}

\textbf{Theorem 33}
\textsl{We have
\[
\bigcup_{a \in A} \overline{a} = A
\]
Furthermore, for all $a,b \in A$ we have
\[
\overline{a} = \overline{b} \; \text{or} \; \overline{a} \cap \overline{b} = \emptyset.
\]}
\begin{proof}
Let $A$ be a set and let $\sim$ be an equivalence relation on $A$. Let $x \in \bigcup_{a \in A} \overline{a}$. Then $x \in \overline{a}$ for some $a \in A$. By definition, $x \in \{ m \in A \mid a \sim m \}$ and so $x \in A$. Therefore, $\bigcup_{a \in A} \overline{a} \subseteq A$. Now suppose $x \in A$. Since $\sim$ is an equivalence relation on $A$, we know $x \sim x$ and so $x \in \overline{x}$. Thus $x \in \bigcup_{a \in A} \overline{a}$ and $A \subseteq \bigcup_{a \in A} \overline{a}$.\newline

Suppose that there exist $a,b \in A$ such that $\overline{a} \neq \overline{b}$ and $\overline{a} \cap \overline{b} \neq \emptyset$. Then there exists an $x$ such that $x \in \overline{a}$ and $x \in \overline{b}$ and so $a \sim x$ and $b \sim x$. But then $x \sim b$ and so $a \sim b$ and $b \sim a$. Now if we choose an element $c \in \overline{a}$ we see that $a \sim c$. But also $c \sim a$, $c \sim b$ and $b \sim c$. This implies that $c \in \overline{b}$ and so $\overline{a} \subseteq \overline{b}$. A similar argument is used to show that $\overline{b} \subseteq \overline{a}$. Thus, $\overline{a} = \overline{b}$ which is a contradiction.
\end{proof}

\textbf{Exercise 34}
\textsl{Try to write a formal definition of a partition.}\newline

We define a partition of a set $A$ to be a set $P$ consisting of $n$ subsets of $A$ such that
\[
\bigcup_{S \in P} S = A,
\]
where $S$ is a subset of A, and $S_i \cap S_j =\emptyset$ for all $1 \leq i,j \leq n$.\newline

\textbf{Exercise 35}
\textsl{How does a partition naturally define an equivalence relation on a set $A$?}\newline

A partition will divide a set into separate equivalence classes $\overline{a_i}$. If $x \in A$ and $x \in \overline{a_i}$ then $a_i \sim x$. Thus, the equivalence relation $\sim$ is defined by which equivalence class $x$ falls into.\newline

\textbf{Exercise 36}
\textsl{How many equivalence classes are there in Exercise 30?}\newline

There are $7$ equivalence classes. For the proof, we first prove a lemma showing that every $x \in \mathbb{Z}$ can be written as $x=7n+k$ where $k \in \{0,1,2,3,4,5,6\}$.

\begin{proof}
Let $x \in \mathbb{N} \cup \{0\}$ and let $S=\{0,1,2,3,4,5,6\}$. Then let $T=\{k \in \mathbb{N} \cup \{0\} \mid \text{there exists } n \in \mathbb{Z} \text{ such that } x = 7n+k\}$. Then we see that $T \neq \emptyset$ since $x = 7(0) + x$ and $x \in \mathbb{N} \cup \{0\}$ and $n \in \mathbb{Z}$. Then we see there exists a least element $m$ of $T$ and so $x=7n+m$ for some $n \in \mathbb{Z}$. If $m \in S$ then we are done. If $m \notin S$ then $m > 7$ and so $m - 7 > 0$. Therefore we can write $x=7(n+1)+(m-7)$ and so $(m-7) \in T$. But $m-7<m$ and since $m$ is the least element of $T$ this is a contradiction so $m \in S$. Therefore every $x \in \mathbb{N} \cup \{0\}$ can be written as $7n+k$ for some $n \in \mathbb{Z}$ and $k \in S$. We now consider the case where $x \in \mathbb{Z} \backslash (\mathbb{N} \cup \{0\})$. We see that $-x = -7n-k = 7(-n-1) + (-k+7)$. But if $k \neq 0$ then $-k+7 \in S$ and if $k=0$ then $x=7n$ and so $-x=7(-n)$ and so we see that for $x \in \mathbb{Z}$ we can write $x=7n+k$ for $n \in \mathbb{Z}$ and $k \in S$.
\end{proof}

Now we prove the original result.

\begin{proof}
Let $x \in \mathbb{Z}$ and let $S = \{0,1,2,3,4,5,6\}$. Then we see that $x = 7n+k$ and $x-k=7n$ for some $n \in \mathbb{Z}$ and $k \in S$. But then $x \sim k$ and so $x \in \overline{k}$. Since there are only $7$ possible values for $k$, we see that there are at most $7$ equivalence classes. If we choose two elements $p,q \in S$ such that $p \neq q$ then without loss of generality we can assume $p>q$ and so $(p-q) \in S$. But then $p-q \neq 7n$ for some $n \in \mathbb{Z}$ and so $p \nsim q$ and $\overline{p} \neq \overline{q}$. So no two equivalence classes are the same. Additionally, for every $p \in S$ we see that $p = 7(0) + p$ and since $0 \in \mathbb{Z}$ and $p \in S$, we see every element of $p$ is in an equivalence class. So we see that there are at least $7$ and at most $7$ equivalence classes so there must be exactly $7$ equivalence classes.
\end{proof}

\textbf{Problem 37}
\textsl{Try to find a way to multiply and add the equivalence classes from Exercise 30.}\newline

It seems that if $a \in \overline{a}$ and $b \in \overline{b}$ then $(a+b) \in \overline{a+b}$ and $ab \in \overline{ab}$. Also, for a scalar $c$, $ca \in \overline{ca}$ for all $a \in \overline{a}$.\newline

\textbf{Definition 38 (Ordering)}
\textsl{Let $A$ be a set. Then $< \subseteq A \times A$ is an ordering if the following hold:\newline
1) For all $x,y \in A$ such that $x \neq y$ we have $x < y$ or $y < x$;\newline
2) For all $x,y \in A$ if $x < y$ then $x \neq y$;\newline
3) For all $x,y,z \in A$ if $x < y$ and $y < z$ then $x < z$.}\newline

\textbf{Theorem 39}
\textsl{If $x,y \in A$, then it cannot be true that both $x < y$ and $y < x$.}
\begin{proof}
Suppose that $x<y$ and $y<x$. Then $x<x$ and so $x \neq x$. This is a contradiction.
\end{proof}

\end{flushleft}
\end{document}