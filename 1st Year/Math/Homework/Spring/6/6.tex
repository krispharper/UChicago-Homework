\documentclass{article}
\usepackage{amsmath,amsthm,amsfonts,amssymb,fullpage}

\begin{document}
\begin{flushright}
Kris Harper

MATH 16300

Mikl\'{o}s Ab\'{e}rt

May 27, 2008
\end{flushright}

\begin{center}
Homework 5
\end{center}

\begin{flushleft}

\textbf{Exercise 1}
\textsl{Write up the matrices for the following:\\
1) Rotation by degree $a$;\\
2) Reflection about the line spanned by $(a,b)$;\\
3) Reflection about $0$.}
\begin{proof}
1) Let
\[
A=
\left [
\begin{array}{cc}
\cos a & -\sin a \\
\sin a & \cos a\\
\end{array}
\right ].
\]
This uses the fact that on the plane, the $\cos$ of an angle corresponds to the distance on the horizontal axis from the origin and the $\sin$ of an angle corresponds to the distance on the vertical axis from the origin.

2)
Make the angle measured from the horizontal axis to the line spanned by $(a,b)$ by
\[
\arctan \left ( \frac{b}{a} \right ) = \alpha.
\]
Then $f(b_2) = \cos 2 \alpha b_2 + \sin 2 \alpha b_1$ and $f(b_1) = \cos ( 2 (\pi/2 - \alpha) ) b_2 - \sin (2 (\pi/2 - \alpha)) b_1$.

3)
This is just a rotation by an angle $\pi$ so we have
\[
A=
\left [
\begin{array}{cc}
-1 & 0\\
0 & -1\\
\end{array}
\right ].
\]
\end{proof}

\textbf{Exercise 2}
\textsl{Find a $2$ by $2$ matrix $A$ such that $f_A$ is bijective, but is not a linear transformation.}
\begin{proof}
Let
\[
A=
\left [
\begin{array}{cc}
2 & 0 \\
0 & 2 \\
\end{array}
\right ].
\]
Then we have $b_1 = (0,1)$ and $b_2 = (1,0)$ so $f(b_1) = 2b_1 + 0b_2 = (0,2)$ and $f(b_2) = 0b_1 + 2b_2 = (2,0)$. Then if $v = a_1b_1 + a_2b_2$ then $f_A(v) = 2a_1b_1 + 2a_2b_2 = 2(a_1b_1 + a_2b_2) = 2v$. Let $v_1, v_2 \in V$ such that $f_A(v_1) = f_A(v_2)$. Then $2v_1 = f_A(v_1) = f_A(v_2) = 2v_2$ and multiplying by $2^{-1}$ we have $v_1 = v_2$. Thus $f_A$ is injective. Now let $v \in V$ and consider $1/2 v$. Then $f_A(1/2 v) = 2/2 v = v$. Thus $f_A$ is surjective and therefore bijective. But $f_A$ is not Euclidean because it doesn't preserve lengths.
\end{proof}

\textbf{Exercise 3}
\textsl{Let $W$ be a $1$ dimensional vector space over $K$. Let $f \; : \; W \rightarrow W$ be a linear transformation. Then there exists $a \in K$ such that $f(v) = av$ for all $v \in W$.}
\begin{proof}
Let $v \in W$ such that $v \neq 0$. Then since $v$ is linearly independent we know it is a basis for $W$. Since $f$ is a linear map from $W$ to $W$ we have $f(v) = w$ for $w \in W$ and since $v$ is a basis for $W$ we know $w = av$ for some $a \in K$. Thus $f(v) = av$. Note that since $f$ is a linear transformation, $\ker f = \{0\}$ and since $v \neq 0$ we have $a \neq 0$. Now consider some other element of $W$, $v' \neq 0$. We know $f(v') = a'v$ for some $a' \in K$ because $v$ is a basis. Then $av = a'v$ so $v(a-a') = 0$ and since $v \neq 0$ we must have $a = a'$.
\end{proof}

\textbf{Exercise 6}
\textsl{Find a real matrix that has no eigenvalues.}
\begin{proof}
Let
\[
A=
\left [
\begin{array}{cc}
0 & -1 \\
1 & 0 \\
\end{array}
\right ].
\]
Note that this matrix changes the direction of every vector put into it by a $90^{\circ}$ rotation. Thus for all vectors $v$ we have $\langle v \rangle$ is not an invariant subspace under $f_A$.
\end{proof}

\textbf{Exercise 7}
\textsl{Let
\[
A=
\left [
\begin{array}{cc}
a & b\\
c & d\\
\end{array}
\right ]
\]
be an arbitrary complex matrix.\\
1) When is $0$ an eigenvalue of $A$?
2) Find the eigenvalues of $A$.}
\begin{proof}
1) If $a=b=c=d=0$ then $f(v) = 0v$ for all $v$ and so $0$ is an eigenvalue of $A$.
\end{proof}

\textbf{Exercise 9}
\textsl{We have
\[
M_{g \circ f} = M_f M_g.
\]}

\textbf{Exercise 10}
\textsl{Let
\[
A=
\left [
\begin{array}{cc}
a & b\\
c & d\\
\end{array}
\right ]
\]
and assume that $0$ is not an eigenvalue of $A$. Find the inverse of $A$.}
\begin{proof}
We want
\[
A^{-1} =
\left [
\begin{array}{cc}
w & x\\
y & z\\
\end{array}
\right ]
\]
such that
\[
AA^{-1} =
\left [
\begin{array}{cc}
1 & 0\\
0 & 1\\
\end{array}
\right ].
\]
Then we have
\[
\left [
\begin{array}{cc}
aw+by & ax+bz\\
cw+dy & cx+dz\\
\end{array}
\right ]
=
\left [
\begin{array}{cc}
1 & 0\\
0 & 1\\
\end{array}
\right ].
\]
Which means we're left with the equations
\[
aw+by = 1
\]
\[
ax+bz = 0
\]
\[
cw+dy = 0
\]
\[
cx+dz = 1.
\]
Solving these we have
\[
w = \frac{d}{ad-bc} \text{ } x = \frac{b}{bc-ad} \text{ } y = \frac{c}{bc-ad} \text{ } z = \frac{a}{ad-bc}.
\]
and so
\[
A^{-1} =
\left [
\begin{array}{cc}
\frac{d}{ad-bc} & \frac{b}{bc-ad}\\
\frac{c}{bc-ad} & \frac{a}{ad-bc}\\
\end{array}
\right ].
\]
\end{proof}

\textbf{Exercise 11}
\textsl{Let
\[
A=
\left [
\begin{array}{cc}
0 & 1\\
1 & 1\\
\end{array}
\right ]
\]
and let $v = (a,b)$. What is $vA$? What is $(1,1)A^n$?}
\begin{proof}
We have
\[
vA=
\left [
\begin{array}{cc}
a & b\\
\end{array}
\right ]
\left [
\begin{array}{cc}
0 & 1\\
1 & 1\\
\end{array}
\right ]
=
\left [
\begin{array}{cc}
a & a+b\\
\end{array}
\right ].
\]
Since this is true for $v = (1,1)$, for $(1,1)A^n$ we obtain the Fibonacci sequence in vectors. That is, $(1,1) A^n$ = $(f_{n+1}, f_{n+2})$ where $f_n$ is the $n$th number in the Fibonacci sequence.
\end{proof}

\end{flushleft}
\end{document}