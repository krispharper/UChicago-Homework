\documentclass{article}
\usepackage{amsmath,amsthm,amsfonts,amssymb,fullpage}

\begin{document}

\begin{flushright}
Kris Harper

MATH 16300

Mikl\'{o}s Ab\'{e}rt

April 15, 2008
\end{flushright}

\begin{flushleft}

\Large

Sheet 16: Metric Spaces\newline

\normalsize

\textbf{Definition 1}
\textsl{Let $X$ be a set. A topology on $X$ is a set $\mathcal{A}$ of subsets of $X$, that we call open sets, satisfying the following:\newline
1) $\emptyset \in \mathcal{A}$ and $X \in \mathcal{A}$;\newline
2) if $A, B \in \mathcal{A}$ then $A \cap B \in \mathcal{A}$;\newline
3) if $\mathcal{B} \subset \mathcal{A}$ then
\[
\bigcup_{B \in \mathcal{B}} B \in \mathcal{A}.
\]}\newline

\textbf{Definition 2}
\textsl{A topological space is a pair $(X, \mathcal{A})$ such that $\mathcal{A}$ is a topology on $X$.}\newline

\textbf{Definition 3}
\textsl{Let $X$ be a set and let $d \; : \; X \times X \rightarrow \mathbb{R}$ be a function. We say that $(X, d)$ is a metric space if the following hold:\newline
1) $d(x,y) \geq 0$ and $d(x,y) = 0$ if and only if $x=y$;\newline
2) $d(x,y) = d(y,x)$;\newline
3) $d(x,y) + d(y,z) \geq d(x,z)$.}\newline

\textbf{Definition 4}
\textsl{For $c \in X$ and $r \in \mathbb{R}$ with $r > 0$ let
\[
B(c,r) = \{x \in X \mid d(c,x) < r\}
\]
be the ball of radius $r$ centered at $c$.}\newline

\textbf{Definition 5}
\textsl{A subset $A \subseteq X$ is open if for every $a \in A$ there exists $r > 0$ such that $B(a,r) \subseteq A$. This topology is the topology generated by $d$.}\newline

\textbf{Theorem 6}
\textsl{For all $c \in X$ and $r > 0$ the ball $B(c,r)$ is open.}
\begin{proof}
Let $a \in B(c,r)$. Then $d(c,a) < r$. Consider the ball $B(a, r-d(c,a))$. For $x \in B(a, r-d(c,a))$ we have $d(a,x) < r - d(c,a)$ so $d(c,a) + d(a,x) < r$. By the triangle inequality we have $d(c,x) < r$ so $x \in B(c,r)$. Thus, $B(a, r-d(c,a)) \subseteq B(c,r)$ and $B(c,r)$ is open.
\end{proof}

\textbf{Proposition 7}
\textsl{There is a topology on $\{0,1\}$ that cannot be generated by any metric on $\{0,1\}$.}
\begin{proof}
Consider the topology $\mathcal{A} = \{\emptyset, \{0,1\}\}$ and consider some arbitrary metric on $\{0,1\}$, $d(0,1) = a$ for $a \in \mathbb{R}$. Then the ball $B(0,a)$ will be in the topology generated by this metric, but $B(0,a) = \{0\}$ which is not in $\mathcal{A}$.
\end{proof}

\textbf{Theorem 8 (Metric Spaces are Hausdorff)}
\textsl{Let $(X,d)$ be a metric space and let $a,b \in X$ with $a \neq b$. Then there exist $A, B \subseteq X$ open such that $a \in A$, $b \in B$ and $A \cap B = \emptyset$.}
\begin{proof}
Consider the two balls $B(a, d(a,b)/2)$ and $B(b, d(a,b)/2)$. Suppose there exists $x \in X$ such that $x \in B(a, d(a,b)/2)$ and $x \in B(b, d(a,b)/2)$. Then $d(a,x) < d(a,b)/2$ and $d(b,x) < d(a,b)/2$ so $d(a,x) + d(x,b) < d(a,b)$ which contradicts the triangle inequality. Thus $B(a, d(a,b)/2) \cap B(b, d(a,b)/2) = \emptyset$. We also have $B(a, d(a,b)/2)$ and $B(a, d(a,b)/2)$ are open (16.6).
\end{proof}

\textbf{Definition 9}
\textsl{Let $A \subseteq X$ be a subset. We say that $x \in X$ is a limit point of $A$ if for all open sets $B \subseteq X$ with $x \in B$ the intersection $A \cap B$ is infinite.}\newline

\textbf{Lemma 10}
\textsl{Let $A \subseteq X$ be a subset. Then $x \in X$ is a limit point of $A$ if for all $r>0$ the intersection $A \cap B(x,r)$ is infinite.}
\begin{proof}
Suppose that for $x \in X$ and all $r > 0$ we have $A \cap B(x,r)$ is infinite. Consider some open set $B \subseteq X$ with $x \in B$. Then there exists $B(x,r) \subseteq B$ because $B$ is open. But then $B \cap A$ is infinite since $B(x,r) \cap A$ is infinite.
\end{proof}

\textbf{Theorem 11}
\textsl{A subset of $X$ is closed if and only if it contains all its limit points.}
\begin{proof}
Let $A \subseteq X$ be closed and consider some point $p \in X \backslash A$. Since $X \backslash A$ is open, there exists some ball $B(p, r) \subseteq X \backslash A$. But since this ball is open and disjoint from $X$ we have $p$ is not a limit point of $A$ (16.6). Thus there are no limit points of $A$ in $X \backslash A$ so $A$ must contain all its limit points. Conversely let $A \subseteq X$ be a subset which contains all its limit points and let $p \in X \backslash A$. Since $p$ is not a limit point of $A$, there exists some ball $B(p,r)$ such that $B(p,r) \cap A$ is finite. Then consider the point $x \in B(p,r) \cap A$ such that $d(p,x) = \min \{d(p,y) \mid y \in B(p,r) \cap A\}$. The ball $B(p,x)$ will then contain no points of $A$ which means $B(p,x) \subseteq X \backslash A$ and thus $X \backslash A$ is open. Then $A$ is closed.
\end{proof}

\textbf{Theorem 12 (Metric Spaces are T3)}
\textsl{Let $C \subseteq X$ be closed and let $b \in X$ such that $b \notin C$. Then there exist $A, B \subseteq X$ open such that $C \subseteq A$, $b \in B$ and $A \cap B = \emptyset$.}
\begin{proof}
Since $C$ is closed, $X \backslash C$ is open and so there exists a ball $B = B(b, r) \subseteq X \backslash C$. Consider the set $S = \{B(a, (d(a,b)-r)/2) \mid a \in C\}$. Then let
\[
A = \bigcup_{B(a,r) \in S} B(a,r)
\]
so that $C \subseteq A$. Now let $x \in A$. Then there exists some ball $B(a, (d(a,b)-r)/2) \subseteq A$ such that $a \in C$ and $x \in B(a, (d(a,b)-r)/2)$. Then $d(x,a) < d(a,b)-r$ so $r < d(a,b) - d(a,x) \leq d(x,b)$. Thus $x \notin B(b,r)$ and so $A \cap B = \emptyset$.
\end{proof}

\textbf{Definition 13}
\textsl{A subset $C \subseteq X$ is compact if every open cover of $C$ has a finite subcover.}\newline

\textbf{Definition 14}
\textsl{A sequence on $X$ is a function from $\mathbb{N}$ to $X$. The sequence $(a_n)$ converges to $a$ (or $\lim_{n \rightarrow \infty} a_n = a$) if for every open set $A \subseteq X$ with $a \in A$ there are only finitely many $n$ with $a_n \notin A$.}\newline

\textbf{Proposition 15}
\textsl{There is a topological space on every set where every sequence converges to every element.}
\begin{proof}
Consider the trivial topology, $\{\emptyset, X\}$. Consider some sequence $(a_n) \in X$ and let $a \in X$. The only open set which contains $a$ is $X$, but there are no terms of $(a_n)$ not in $X$ so we have for all open sets $A$ with $a \in A$, there are finitely many terms of $(a_n)$ not in $A$. Thus $(a_n)$ converges to $a$. This is true of all sequences and points in $X$.
\end{proof}

\textbf{Proposition 16}
\textsl{There is a topological space on every set where the only convergent sequences are the ones that are constant up to finitely many elements.}
\begin{proof}
Consider the full topology where every subset is open. Then for all $x \in X$, the set $\{x\}$ is open. Thus for a sequence $(a_n)$, there are finitely many $n$ such that $a_n \notin \{x\}$ which means there are finitely many $n$ such that $a_n \neq x$.
\end{proof}

\textbf{Definition 17}
\textsl{Let $(X, \mathcal{A})$ and $(Y, \mathcal{B})$ be topological spaces. A function $f \; : \; X \rightarrow Y$ is continuous if for all $B \in \mathcal{B}$ the preimage $f^{-1}(B) \in \mathcal{A}$}\newline

\textbf{Theorem 18}
\textsl{Let $(X, \mathcal{A})$ be a Hausdorff topological space and let $(a_n)$ be a sequence on $X$. If $\lim_{n \rightarrow \infty} a_n = a$ and $\lim_{n \rightarrow \infty} a_n = b$ then $a=b$.}
\begin{proof}
Suppose that $a \neq b$. Then there exist two open sets $A$ and $B$ such that $a \in A$ and $b \in B$ and $A \cap B = \emptyset$ by the Hausdorff property. There are finitely many $n$ with $a_n \notin A$ so there are infinitely many $n$ with $a_n \in A$. But then there are finitely many $n$ with $a_n \notin B$ which is a contradiction because $\lim_{n \rightarrow \infty} a_n = b$. Thus $a=b$.
\end{proof}

\textbf{Theorem 19}
\textsl{Let $(X,d)$ and $(X',d')$ be metric spaces and let $f \; : \; X \rightarrow X'$ be a function. Then the following are equivalent:\newline
1) $f$ is continuous;\newline
2) for all $x \in X$ and for all $\varepsilon > 0$ there exists $\delta > 0$ such that for all $y \in X$ with $d(x,y) < \delta$ we have $d'(f(x),f(y)) < \varepsilon$;\newline
3) for all convergent sequences $a_n \in X$ we have
\[
\lim_{n \rightarrow \infty} f(a_n) = f \left ( \lim_{n \rightarrow \infty} a_n \right ).
\]}
\begin{proof}
Let $f$ be continuous and let $x \in X$ and consider the ball $B(f(x), \varepsilon)$ for $\varepsilon > 0$. Then since $f$ is continuous, $f^{-1}(B(f(x), \varepsilon))$ is open. And since $x \in f^{-1}(B(f(x), \varepsilon))$ there exists some ball $B(x, \delta) \subseteq B(f(x), \varepsilon)$. But then for all $y \in B(x, \delta)$, $f(y) \in B(f(x), \varepsilon)$. Thus for all $y \in X$ such that $d(x,y) < \delta$ we have $d'(f(x),f(y)) < \varepsilon$.\newline

Now suppose that for all $x \in X$ and for all $\varepsilon > 0$ there exists $\delta > 0$ such that for all $y \in X$ with $d(x,y) < \delta$ we have $d'(f(x),f(y)) < \varepsilon$. Let $a_n \in X$ be a sequence which converges to $a$ and let $\varepsilon > 0$. Consider $B(a, \delta)$. Since $\lim_{n \rightarrow \infty} a_n = a$, there are finitely many $n$ with $a_n \notin B(a, \delta)$. But then there are finitely many $n$ such that $d(a,a_n) \geq \delta$ which means there are finitely many $n$ with $d'(f(a),f(a_n)) \geq \varepsilon$. Therefore there are finitely many $n$ with $f(a_n) \notin B(f(a), \varepsilon)$ and since this is true for all $\varepsilon > 0$, we have $\lim_{n \rightarrow \infty} f(a_n) = f(a)$.\newline

Finally use the contrapositive and assume that $f$ is not continuous. Then there exists some set $A \subseteq X'$ such that $f^{-1}(A)$ is not open. Then there exists $a \in f^{-1}(A)$ such that for all $r>0$ there exists $x \in B(a,r)$ such that $x \notin A$. Create a sequence $a_n \in X$ where $a_n \in B(a, 1/n)$, but $a_n \notin A$. We know that $a_n$ exists for all $n$ because $f^{-1}(A)$ is not open. Note that for the ball $B(a,r)$ with $r>1$ there are no terms of $(a_n)$ not in $B(a,r)$ and for $r \leq 1$ we can use the Archimedean Property to show that there are finitely many terms not in $B(a,r)$. Thus $(a_n)$ converges to $a$. Note that for all $n$, $a_n \notin f^{-1}(A)$ and thus $f(a_n) \notin A$, while $a \in f^{-1}(A)$ and so $f(a) \in A$. But $A$ is open so there exists some ball $B(a,r) \subseteq A$ for which $a_n \notin B(a,r)$ for all $n$. But then $\lim_{n \rightarrow \infty} f(a_n) \neq f(a)$.
\end{proof}

\textbf{Theorem 20}
\textsl{Let $(X, \mathcal{A})$ and $(Y, \mathcal{B})$ be topological spaces and let $f \; : \; X \rightarrow Y$ be continuous. Then for every compact subset $C \subseteq X$ the image $f(C)$ is also compact.}
\begin{proof}
Let $\mathcal{E} \subseteq \mathcal{B}$ be an open cover of $f(C)$. For all $x \in C$ we have $x \in f(C)$ and so for all $x \in C$ there exist an open set $B \in \mathcal{E}$ such that $f(x) \in B$. But then for all $x \in C$, $x \in f^{-1}(B)$ for some $B \in \mathcal{E}$. So we have $C \subseteq \bigcup_{B \in \mathcal{E}} f^{-1}(B)$ and since $f$ is continuous $\{f^{-1}(B) \mid B \in \mathcal{E}\} \subseteq \mathcal{A}$ is an open cover for $C$. But $C$ is compact so there exists a finite subcover, $\{f^{-1}(B_1), f^{-1}(B_2), \dots , f^{-1}(B_n)\}$ which covers $C$. So for all $x \in C$ there exists some $B_i \in \mathcal{E}$ such that $x \in f^{-1}(B_i)$. But then $f(x) \in B_i$ and since $f(C) = \{ y \in Y \mid x \in C, y=f(x) \}$, we have for all $y \in f(C)$, $y \in B_i$. Since every $B_i \in \mathcal{E}$ we have found a finite subcover of $\mathcal{E}$ which covers $f(C)$. Thus $f(C)$ is compact.
\end{proof}

\textbf{Theorem 21}
\textsl{Let $(X, d)$ be a metric space. Then every compact subset of $X$ is bounded and closed.}
\begin{proof}
Let $C$ be a compact subset of $X$ and suppose that $C$ is not bounded below. Let $\mathcal{A}$ be the set of all balls centered at $c \in C$. Then $\mathcal{A}$ covers $C$ and since $C$ is compact there exists a finite subcover $\mathcal{B} \subseteq \mathcal{A}$ which covers $C$. Then $\mathcal{B} = \{B(c,r_1), B(c,r_2), \dots , B(c,r_n)\}$. Take the largest $r_i$ such that $B(c,r_i) \in \mathcal{B}$. But we have $C$ is not bounded below so there exists $x \in C$ such that $d(x,c) > r_i$. Thus $C \nsubseteq \bigcup_{B \in \mathcal{B}} B$ and so $\mathcal{B}$ doesn't cover $C$. This is a contradiction and so compact sets are bounded below. A similar proof holds to show compact sets must be bounded above.\newline

Now suppose that $C \subseteq X$ is compact and $C$ is not closed. Let $p \notin C$ be a limit point of $C$. Let $\mathcal{A} = \{X \backslash B(p,r) \mid r \in \mathbb{R}\}$. Since $p \notin C$ we see that $\mathcal{A}$ covers $C$. Since $C$ is compact, let $\mathcal{B}$ be a finite subset of $\mathcal{A}$ which covers $C$. We have $X$ is open and $X \backslash \emptyset$ is closed so $X \neq \emptyset$. Thus if $\mathcal{B} = \emptyset$, $\mathcal{B}$ does not cover $X$. Then $\mathcal{B} = \{X \backslash B_1(p,r_1), X \backslash B_2(p,r_2), \dots , X \backslash B_n(p,r_n)\}$. Take the smallest $r_i$ such that $B_i(p,r_i) \in \mathcal{B}$ and consider $B(p,r_i/2)$. This ball contains $p$, which is a limit point of $C$, and since balls are open, $B(p,r_i/2) \cap C \neq \emptyset$. But $B(p,r_i/2)$ is defined such that $B(p,r_i/2) \nsubseteq \bigcup_{B \in \mathcal{B}} B$ and so $C \nsubseteq \bigcup_{B \in \mathcal{B}} B$. But then $\mathcal{B}$ doesn't cover $C$ which is a contradiction. Therefore compact sets are closed.
\end{proof}

\textbf{Proposition 22}
\textsl{Let $X$ be an infinite set. Then there is a metric on $X$ such that there exists a bounded and closed set that is not compact.}
\begin{proof}
Consider the metric $d(x,y) = a$ for some $a \in \mathbb{R}$. Let $Y \subseteq X$ be a bounded closed infinite set and let $\mathcal{A} = \{B(y, a) \mid y \in Y\}$. This set covers $Y$, but each element contains only one element of $Y$ so a finite subset of $\mathcal{A}$ will only contain finitely many elements of $Y$.
\end{proof}

\textbf{Definition 23}
\textsl{Let $(X,d)$ and $(X',d')$ be metric spaces and let $f \; : \; X \rightarrow X'$ be a function. We say that $f$ is uniformly continuous if for all $\varepsilon > 0$ there exists $\delta > 0$ such that for all $x, y \in X$ with $d(x,y) < \delta$ we have $d'(f(x),f(y)) < \varepsilon$.}\newline

\textbf{Theorem 24}
\textsl{Let $(X,d)$ and $(X',d')$ be metric spaces and let $f \; : \; X \rightarrow X'$ be a continuous function. If $X$ is compact then $f$ is uniformly continuous.}
\begin{proof}
Let $\varepsilon > 0$ and consider $\varepsilon/2 > 0$. We have $f$ is continuous so for all $x \in X$ there exists $\delta (x) > 0$ such that for all $y \in X$ with $d(x,y) < \delta (x)$ we have $d'(f(x),f(y)) < \varepsilon/2$ (16.19). Consider the set of balls $\mathcal{A} = \{B(x, \delta (x)) \mid x\in X\}$ and let $\mathcal{A}' = \{B(x, \delta(x)/2) \mid B(x, \delta(x)) \in \mathcal{A}\}$. $\mathcal{A}'$ is an open cover for $X$ and since $X$ is compact there exists a finite subcover, $\mathcal{B} \subseteq \mathcal{A}'$. Let $\delta = \min \{\delta(x)/2 \mid B(x, \delta(x)/2) \in \mathcal{B} \}$. Then consider two points $x,y \in X$ such that $d(x,y) < \delta$. $\mathcal{B}$ is an open cover for $X$ so there exists some ball $B(z, \delta(z)/2) \in \mathcal{B}$ such that $x \in B(z, \delta(z)/2)$. Then $d(x,z) < \delta(z)/2 < \delta(z)$ and $d(x,y) < \delta \leq \delta(z)/2$ so $d(z,y) \leq d(z,x) + d(x,y) < \delta(z)$. But then $d'(f(z),f(x)) < \varepsilon/2$ and $d'(f(z),f(y)) < \varepsilon/2$ so $d'(f(x),f(y)) \leq d'(f(x),f(z)) + d'(f(z),f(y)) < \varepsilon$. Therefore for every $\varepsilon > 0$ there exists a $\delta > 0$ such that for all $x,y \in X$ with $d(x,y) < \delta$ we have $d'(f(x),f(y)) < \varepsilon$.
\end{proof}

\end{flushleft}
\end{document}