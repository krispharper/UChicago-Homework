\documentclass{article}
\usepackage{amsmath,amsthm,amsfonts,amssymb,fullpage}

\newcommand{\J}{\textup{J-rad\;}}
\newcommand{\ann}{\textup{ann}}
\newcommand{\nil}{\textup{nil}\;}

\newtheorem{problem}{Problem}

\begin{document}

\begin{flushright}
Kris Harper\\

MATH 26800\\

January 11, 2011
\end{flushright}

\begin{center}
Homework 1
\end{center}

\noindent
In the following, $A$ denotes a commutative ring.

\begin{problem}
Let $E$ be an $A$-module and $F \subseteq E$ a submodule of $E$ such that $E/F$ is a finite $A$-module. Let $I \subseteq \J A$ be an ideal. Suppose $E = F + IE$. Show that $E = F$.
\end{problem}
\begin{proof}
Since $E = F + IE$, we know $E/F = (F + IE)/F$. The right hand side is
\begin{align*}
\left \{ \left (f + \sum a_ie_i \right ) + F \mid f \in f, a_i \in I, e_i \in E \right \}
&= \left \{(f + F) + \left (\sum a_ie_i + F \right ) \mid f \in F, a_i \in I, e_i \in E \right \}\\
&= \left \{\sum a_ie_i + F \mid a_i \in I, e_i \in E \right \}\\
&= IE/F\\
&= I(E/F).
\end{align*}
Thus $E/F = I(E/F)$. Now since $I \subseteq \J A$ and $E/F$ is finitely generated, we can apply Nakayama's Lemma to get $E/F = 0$. Thus $E = F$.
\end{proof}

\begin{problem}
For an $A$-module $E$, we denote by $\ann(E) = \{a \in A \mid aE = 0\}$; \ann(E) is an ideal called the annihilator of $E$. Let $E$ be a finite $A$-module. Suppose $E$ is a Noetherian $A$-module (respectively Artinian $A$-module). Show that $A/\ann(E)$ is a Noetherian ring (Artinian ring).
\end{problem}
\begin{proof}
Let $E = Ax_1 + \dots + Ax_n$ and define $f : A \to E^n$ as $f : a \mapsto (ax_1, \dots , ax_n)$ where $E^n$ is the direct sum of $n$ copies of $E$. Note that if $a \in \ker f$ then $ax_1 = \dots = ax_n = 0$ and since the $x_i$ generate $E$, $a \in \ann(E)$. Conversely, if $a \in \ann(E)$ then clearly $f(a) = 0$. But then an isomorphic copy of $A/\ker f$ sits inside $E^n$, which is Noetherian (Artinian). Since it's a submodule of a Noetherian (Artinian) module, it too must be Noetherian (Artinian).
\end{proof}

\begin{problem}
Let $E$ be an $A$-module. Let $E_1$, $E_2$ be submodules of $E$ such that $E/E_1$ and $E/E_2$ are Noetherian (respectively Artinian) $A$-modules. Show that $E/(E_1 \cap E_2)$ is an Noetherian (Artinian) $A$-module.
\end{problem}
\begin{proof}
Consider the following exact sequence
\[
0 \to E/E_1 \to E/(E_1 \cap E_2) \to E/E_2 \to 0
\]
where the first map is inclusion and the second map is projection. Since the outside terms are Noetherian (Artinian), the middle term is also Noetherian (Artinian).
\end{proof}

\begin{problem}
Let $I$ be an ideal in $A$. $I$ is called a \emph{nil ideal} if every element of $I$ is nilpotent, i.e. $I \subseteq \nil A$. An ideal $I$ is called \emph{nilpotent} if $I^m = 0$ for some $m > 0$.\\
(a) Five an example of a commutative ring and a nil ideal which is not nilpotent.\\
(b) Show that any finitely generated nil ideal is nilpotent. Thus in a Noetherian ring every nil ideal is nilpotent.
\end{problem}
\begin{proof}
(a) Let $A = \bigoplus_{i=1}^{\infty} \mathbb{Z}/(p^i)$ for some prime $p$. Let $I$ be the set of all nilpotent elements of $A$. Note that $I$ is nontrivial since, for example, $(0 + (p), p + (p^2), 0 + (p^3), 0 + (p^4), \dots )$ is an element of $I$. By definition, $I$ is a nil ideal since it contains only nilpotent elements. Suppose $I^k = 0$ for some integer $k$. But then consider the nilpotent element
\[
a = (0 + (p), 0 + (p^2), \dots , 0 + (p^k), p + (p^{k+1}), 0 + (p^{k+2}), \dots )
\]
and note that $a^k \neq 0$, a contradiction. Thus $I$ is not nilpotent, but is a nil ideal.

(b) Let $I = Ax_1 + \dots + Ax_r$ be a finitely generated nil ideal. Since each $x_i$ is nilpotent, write $x_i^{n_i} = 0$ for $1 \leq i \leq r$. An element of $I^n$, for a positive integer $n$, is of the form
\[
x = \prod_{j=1}^n \left (\sum_{i=1}^r a_{ij} x_i \right ).
\]
If $n$ is sufficiently large then each term in the expansion will contain a factor of $x_i$ raised to a power greater than or equal to $n_i$. Each of these terms will go to $0$ and so $I^n = 0$ for large enough $n$. Thus $I$ is nilpotent. Since a Noetherian ring has all ideals finitely generated, every nil ideal is nilpotent.
\end{proof}

\begin{problem}
Let $K$ be a field and $A$ the subring of $K[x,y]$ generated by $K \cup \{x, xy, xy^2, \dots \}$, i.e. $A = K[x, xy, xy^2, \dots ] = \{f(x,y) \in K[x,y] \mid f(0,y) \in K\}$. Show that $A$ is \emph{not} a Noetherian ring.
\end{problem}
\begin{proof}
Let $I_n = (x, xy, xy^2, \dots , xy^n)$. Then $I_1 \subseteq I_2 \subseteq I_3 \subseteq \dots$. Suppose that this chain terminates at $I_n$ for some $n$. Then $I_{n+1} = I_n$ so we must be able to write $xy^{n+1}$ as a sum and product of $x, xy, \dots , xy^n$ and the elements of $K$. Since the degree of a polynomial can't increase under addition, we must multiply two or more polynomials from $I_n$ to get $xy^{n+1}$. But if we multiply two polynomials to get a $y^{n+1}$ term, we must also get a $x^2$ term. There's no way to from $xy^{n+1}$ from the generators of $I_n$, so this chain of ideals doesn't terminate and $A$ is not Noetherian.
\end{proof}

\begin{problem}
Let $C$ denote the set of all real valued functions $f : \mathbb{R} \to \mathbb{R}$. $C$ is a commutative ring with operations $(f \pm g)(x) = f(x) \pm g(x)$, $(f \cdot g)(x) = f(x) \cdot g(x)$ for each $f, g \in C$. Show that $C$ is not a Noetherian ring.
\end{problem}
\begin{proof}
Let $I_n$ be the ideal of functions $f$ such that $f(x) = 0$ for each $x \geq n$. This is an ideal since if $g$ is an arbitrary element of $C$ then $(g \cdot f)(x) = g(x) \cdot f(x) = g(x) \cdot 0 = 0$ for $x \geq n$. Also $I_1 \subseteq I_2 \subseteq I_3 \dots$ since any function which is $0$ for $x \geq n$ is certainly $0$ for $x \geq n+1$. Now suppose that this chain terminates for some $n$. Then $I_n = I_{n+1}$. But this is clearly false since $I_{n+1}$ contains the function which is $1$ for $x < n+1$ and $0$ for $x \geq n$, for example. Since $I_n$ doesn't contain this function, this is a contradiction and this chain doesn't terminate. Thus $C$ is not Noetherian.
\end{proof}

\begin{problem}
Let $E$ be an $A$-module and $E_i$, $0 \leq i \leq n$ submodules such that $E = E_0 \supseteq E_1 \supseteq E_2 \supseteq \dots \supseteq E_n = 0$. Suppose each $E_i/E_{i+1}$ is Noetherian (respectively Artinian). Show that $E$ is Noetherian (Artinian).
\end{problem}
\begin{proof}
Note that $E_n$ is trivially Noetherian and $E_{n-1}/E_n$ is Noetherian by assumption. Thus $E_{n-1}$ is Noetherian. Similarly, since $E_{n-1}$ and $E_{n-2}/E_{n-1}$ are both Noetherian, we know $E_{n-2}$ is Noetherian. Continuing in this fashion we inductively have $E_1$ and $E_0/E_1$ are both Noetherian so $E_0 = E$ must be Noetherian.
\end{proof}

\begin{problem}
Let $A$ be a commutative ring and $I \subseteq A$ an ideal. Let $E$ be an $A$-module. Suppose that $I/I^2$ and $E/IE$ are finite $A$-modules (hence also finite $A/I$-modules).\\
(a) Show that $IE/I^2E$ is a finite $A$-module.\\
(b) Show by induction that $I^nE/I^{n+1}E$ is a finite $A$-module for all $n \geq 0$.\\
(c) Suppose further that $A/I$ is a Noetherian ring. Show that $E/I^nE$ is a Noetherian $A$-module for all $n \geq 1$.
\end{problem}
\begin{proof}
(a) Suppose $a_i \in I$ for $1 \leq i \leq r$ and $x_j \in E$, $1 \leq j \leq m$ are such that $a_i + I^2$ and $x_j + IE$ generate $I/I^2$ and $E/IE$ respectively as $A$-modules. Let $b = \sum_{i=1}^n b_iy_i + I^2E$ be an arbitrary element of $IE/I^2E$. Then $b = \sum_{i=1}^n (b_iy_i + I^2E)$. Letting $i$ and $j$ vary we get all possible products of elements from $I/I^2$ and $E/IE$. Since $b_iy_i + I^2E$ is of this form, we must have $b$ in the set generated by $a_ix_j + I^2E$, $1 \leq i \leq r$, $1 \leq j \leq m$. Thus, this is a generating set for $IE/I^2E$.

(b) For $n = 0$ we have $E/IE$ is finite by assumption. Now suppose $I^{n-1}E/I^nE$ is finite for some $n$ so that $I^{n-1}E/I^nE$ is generated by $b_ky_l + I^nE$ where $b_k \in I^{n-1}$ and $y_l \in E$ for $1 \leq k \leq s$ and $1 \leq l \leq t$. Then by the same argument as in part (a), a generating set for $I^n/I^{n+1}$ is $b_ka_iy_lx_j + I^{n+1}E$ for $1 \leq k \leq s$, $1 \leq i \leq r$, $1 \leq l \leq t$ and $1 \leq j \leq m$ where the $a_i$ and $x_j$ are as in part (a).

(c) Since $A/I$ is Noetherian and $I^nE/I^{n+1}E$ is a finite $A$-module (and thus a finite $A/I$ module), it's also a Noetherian $A/I$ module. Now consider the exact sequence
\[
0 \to I^{n-1}E/I^nE \to E/I^nE \to E/I^{n-1}E \to 0.
\]
This sequence is exact by the third isomorphism theorem for modules which states that $(E/I^nE)/(I^{n-1}E/I^nE) \cong E/I^{n-1}E$. When $n = 1$ we know $E/IE$ is a finite $A/I$-module by assumption and is thus Noetherian. Suppose that $E/I^{n-1}E$ is Noetherian for some $n$. Then using this hypothesis and the statement above we see that the outer two terms in the exact sequence are Noetherian, so $E/I^nE$ must also be Noetherian. Therefore $E/I^nE$ is Noetherian for all $n$.
\end{proof}

\end{document}