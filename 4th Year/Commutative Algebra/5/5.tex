\documentclass{article}
\usepackage{amsmath,amsthm,amsfonts,amssymb,fullpage}

\input xy
\xyoption{all}

\newcommand{\spec}{\textup{Spec}}
\newcommand{\Max}{\textup{Max}}
\newcommand{\nil}{\textup{nil}\,}

\newtheorem{problem}{Problem}

\begin{document}

\begin{flushright}
Kris Harper\\

MATH 26800\\

February 8, 2011
\end{flushright}

\begin{center}
Homework 5
\end{center}

\noindent
Here $A$ is a commutative ring.

\begin{problem}
Let $S \subseteq A$ be a multiplicative set. Let $E$ be an $A$-module and $F \subseteq E$ a submodule. We say that $F$ is $S$-saturated if $sx \in F$, $s \in S$, $x \in E$ implies that $x \in F$. For a module $F$, the submodule $\overline{F}$, defined by $\overline{F} = \{x \in E \mid sx \in F \text{ for some } s \in S\}$ is called the saturation of $F$. Note that $F \subseteq \overline{F}$.\\
(a) Show that for any submodule $F \subseteq E$, $i_{S,E}^{-1}(S^{-1}F) = \overline{F}$.\\
(b) The map $F' \mapsto i_{S,E}^{-1}(F')$ form the set of all submodules of $S^{-1}E$ to the set of all $S$-saturated submodules of $E$ is a bijection.
\end{problem}
\begin{proof}
(a) Let $x \in i_{S,E}^{-1}(S^{-1}F)$. Then $i_{S,E}(x) = x/1 = f/s$ for some $f \in F$ and $s \in S$. This means there exists $t \in S$ such that $tsx = tf$. Since $tf \in F$ and $ts \in S$,  we see that $x \in \overline{F}$. Conversely, if $x \in \overline{F}$, then there exists $s \in S$ such that $sx = f$ for some $f \in F$. Apply $i_{S,E}^{-1}$ to get $sx/1 = f/1$ or $x/1 = f/s$, which is an element of $S^{-1}F$. Thus $x \in i_{S,E}^{-1}(S^{-1}F)$.

(b) Let $F'$ and $G'$ be submodules of $S^{-1}E$ and suppose $i_{S,E}^{-1}(F') = i_{S,E}^{-1}(G')$. Then pick $x/s \in F'$. Since $F'$ is a submodule of $S^{-1}E$, it's a module of $S^{-1}A$ and therefore closed under $S^{-1}A$-scalar multiplication. Therefore $x/s \cdot s/1 = x/1$ is in $F'$. But this means $x \in i_{S,E}^{-1}(F')$ and so $x \in i_{S,E}^{-1}(G')$ as well. Therefore $x/1 \in G'$ and since $G'$ is a $S^{-1}A$ module, $x/s \in G'$ which means $F' \subseteq G'$. A similar proof shows that $G' \subseteq F'$ so this map is injective.

Now let $F$ be an $S$-saturated submodule of $E$. Let $F'$ be the submodule of $S^{-1}E$ generated by $x/1$ for all $x \in F$. Let $x \in i_{S,E}^{-1}(F')$. Then $x/1 \in F'$ so $x/1 = sf/t$ for some $s,t \in S$ and $f \in F$. Then there exists $u \in S$ such that $utx = usf$. Note that $usf \in F$ and $ut \in S$. Since $F$ is $S$-saturated, we know $x \in F$. On the other hand, if $x \in F$ then $i_{S,E}(x) = x/1$ is in $F'$ by construction, so $x \in i_{S,E}^{-1}(F')$. Therefore $F = i_{S,E}^{-1}(F')$ and $F'$ maps to $F$ under this map. Hence the map is surjective and thus a bijection.
\end{proof}

\begin{problem}
For a multiplicative set $S \subseteq A$, we define the saturation of $S$
\[
\overline{S} = \{a \in A \mid \text{there exists $b \in A$ with $ab \in S$}\}.
\]
Then $\overline{S}$ is a multiplicative set and $S \subseteq \overline{S}$. Show that the natural maps $\xymatrix{S^{-1}A \ar[r]^{\eta_{S,\overline{S}}} & \overline{S}^{-1}A}$ and $S^{-1}E \to \overline{S}^{-1}E$ are a ring isomorphism and a $S^{-1}A$-module isomorphism respectively (where $\overline{S}^{-1}E$ is regarded as an $S^{-1}A$-module via $\eta_{S,\overline{S}}$).
\end{problem}
\begin{proof}
Let $a/s$ and $b/t$ be elements of $S^{-1}A$. Then $\eta_{S,\overline{S}}(a/s) = a/s$ and $\eta_{S,\overline{S}}(b/t) = b/t$ in $\overline{S}^{-1}A$. Suppose the images are equal so that $a/s = b/t$. Then there exists $u \in \overline{S}$ such that $uat = ubs$. Since $u \in \overline{S}$, there exists $c \in A$ such that $cu \in S$. Therefore $cuat = cubs$ and since $cu \in S$, this means $a/s = b/t$ in $S^{-1}A$. Thus $\eta_{S, \overline{S}}$ is injective.

Now take $a/s \in \overline{S}^{-1}A$. Since $s \in \overline{S}$, there exists $b \in A$ such that $bs \in S$. Now note that $ba/bs$ is an element of $S^{-1}A$ and $\eta_{S,\overline{S}}(ba/bs) = ba/bs = a/s$ since $s \in \overline{S}$. Therefore $\eta_{S,\overline{S}}$ is surjective.

To show $\eta_{S,\overline{S}}$ is a homomorphism take $a/s, b/t \in S^{-1}A$ and note that
\[
\eta_{S,\overline{S}}(a/s \cdot b/t) = \eta_{S,\overline{S}}(ab/st) = ab/st = a/s \cdot b/t = \eta_{S,\overline{S}}(a/s) \cdot \eta_{S,\overline{S}}(b/t).
\]
Similarly,
\[
\eta_{S,\overline{S}}(a/s + b/t) = \eta_{S,\overline{S}}((at+bs)/st) = (at+bs)/st = a/t + b/s = \eta_{S,\overline{S}}(a/s) + \eta_{S,\overline{S}}(b/t).
\]
Thus $\eta_{S,\overline{S}}$ is an isomorphism.

Similarly, take $x/s, y/t \in S^{-1}E$ which have images $x/s$ and $y/t$ in $\overline{S}^{-1}E$. Suppose the images are equal so that $x/s = y/t$. Then there exists $u \in \overline{S}$ such that $utx = usy$. Since $u \in \overline{S}$ there exists $a \in A$ such that $au \in S$. Then $autx = ausy$ and $au \in S$ so $x/s = y/t$ in $S^{-1}E$. Thus our map is injective.

Now take $x/s \in \overline{S}^{-1}E$ and pick $a \in A$ such that $as \in S$. Then $ax/as \in S^{-1}E$ and our map takes this element to $ax/as = x/s$ in $\overline{S}^{-1}E$ so it's surjective as well.

Finally, let $x/s \in S^{-1}E$ and $a/t \in S^{-1}A$. Then the element $a/t \cdot x/s$ is mapped to $a/t \cdot x/s$ in $\overline{S}^{-1}E$ where now $a/t \in \overline{S}^{-1}A$ and $x/t \in \overline{S}^{-1}E$. Since $a/t$ is the image under $\eta_{S,\overline{S}}$ of $a/t \in S^{-1}A$, this map is $S^{-1}A$ linear.

The fact that the map is a module homomorphism is the same as the additive statement for $\eta_{S,\overline{S}}$ above. Thus, this map is an $S^{-1}A$-module isomorphism.
\end{proof}

\begin{problem}
Let $f : A \to B$ be a ring homomorphism and $S \subseteq A$ be a multiplicative set. Let $E$ be a $B$-module. One can form $S^{-1}E$ by regarding $E$ as an $A$-module via $f$. Also regarding $E$ as a $B$-module, one can form $f(S)^{-1}E$. Show that the natural map $\eta_E : S^{-1}E \to f(S)^{-1}E$, given by $\eta_E(x/s) = x/f(s)$, is an $A$-linear isomorphism. In fact $\eta_B$ is an isomorphism of rings. We sometimes identify $S^{-1}E$ with $f(S)^{-1}E$.
\end{problem}
\begin{proof}
Let $x/s, y/t \in S^{-1}E$. Then since $t,s \in S$ act on $x,y \in E$ as $t \cdot x = f(t)x$ and $s \cdot y = f(s)y$, we have
\[
\eta_E(x/s + y/t) = (t \cdot x + s \cdot y)/f(st) = (f(t)x + f(s)y)/f(s)f(t) = x/f(s) + y/f(t) = \eta_E(x/s) + \eta_E(y/s).
\]
Furthermore, if $a \in A$ then
\[
\eta_E(a \cdot x/s) = \eta_E(f(a)x/s) = f(a)x/f(s) = f(a) \eta_E(x/s) = a \cdot \eta_E(x/s).
\]
So $\eta_E$ is an $A$-module homomorphism. Now suppose $\eta_E(x/s) = x/f(s) = y/f(t) = \eta_E(y/t)$. Then there exists $u \in f(S)$ such that $uf(t)x = uf(s)y$. Since $u \in f(S)$, there is some $v \in S$ such that $u = f(v)$. Making this substitution we have
\[
vt \cdot x = f(vt)x = f(v)f(t)x = f(v)f(s)y = f(vs)y = vs \cdot y
\]
which means $x/s = y/t$ in $S^{-1}E$. This shows that $\eta_E$ is injective.

Finally, take $x/s \in f(S)^{-1}E$ and write $s = f(t)$ for some $t \in S$. Then clearly $\eta_E(x/t) = x/f(t) = x/s$ so $\eta_E$ is surjective as well. Hence, $\eta_E$ is an $A$-module isomorphism.
\end{proof}

\begin{problem}
Let $E$ be a finite $A$-module. Then show that $E = 0$ if and only if $E_M/ME_M = 0$ for all $M \in \Max(A)$.
\end{problem}
\begin{proof}
If $E = 0$ then $E_M = 0$ for all $M \in \Max(A)$, thus $E_M/ME_M = 0$ for all $M \in \Max(A)$. Conversely, suppose $E_M/ME_M = 0$ for all $M \in \Max(A)$. Then $E_M = ME_M$ so there exists $x-1 \in M$ such that $xE_M = 0$. Since $E$ is finitely generated, so is $E_M$. Since $x-1 \in M$, we know $x \in A \backslash M$. Therefore $xE_M = 0$ implies $(E_M)_M = E_M = 0$. Since this is true for each $M \in \Max(A)$, we much have $E = 0$.
\end{proof}

\begin{problem}
Let $E$ be an $A$-module and $F$, $F'$ submodules of $E$. Suppose $F_M \subseteq F_M'$, for all $M \in \Max(A)$. Show that $F \subseteq F'$. Thus deduce that $F_M = F_M'$ for all $M \in \Max(A)$ if and only if $F = F'$.
\end{problem}
\begin{proof}
Let $x \in F$ so that $x/1 \in F_M$ and $x/1 \in F_M'$. Then $x/1 = x'/s$ for some $x' \in F'$ and $s \in A \backslash M$. So there exists $s_M \in A \backslash M$ such that $s_Msx = s_Mx'$. Consider the ideal $I = \{a \in A \mid ax \in F'\}$. This is not in any maximal ideal since for each $M \in \Max(A)$ we found $s_Ms \notin M$ such that $s_Msx \in F'$. Therefore $I = A$ which means $1 \in I$. Therefore $x \in F'$ and $F \subseteq F'$. Now if $F_M = F_M'$ for all $M \in \Max(A)$ then we have both inclusions so $F = F'$.
\end{proof}

\begin{problem}
(a) Let $f : A \to B$ be a ring homomorphism. Show that the map $f^* : \spec(B) \to \spec(A)$, $f^*(A) = f^{-1}(Q)$ is continuous.\\
(b) With $f$ as in (a), further suppose that $b \in B$ implies $b = f(a)u$ for some $a \in A$ and $u \in B^*$. Show that $f^*$ is a homeomorphism of $\spec(B)$ onto its image in $\spec(A)$.\\
(c) For a multiplicative set $S \subseteq A$, show that $i_{S,A}^* : \spec(S^{-1}A) \to \spec(A)$ is a homeomorphism with image $\{P \in \spec(A) \mid P \cap S = \emptyset \}$.
\end{problem}
\begin{proof}
(a) Let $V(I)$ be a closed set in $\spec(A)$. Then $(f^*)^{-1}(V(I)) = f(V(I))$. Since $V(f(I))$ is a closed set in $\spec(B)$, it's enough to show that $V(f(I)) = f(V(I))$.

So let $P \in f(V(I))$. Then $P = f(Q)$ for some prime $Q \supseteq I$. Applying $f^{-1}$ then we have $f^{-1}(P) \supseteq Q \supseteq I$ so $P \supseteq f(I)$ and $P \in V(f(I))$. Conversely if $P \in V(f(I))$ then $P \supseteq f(I)$ and $f^{-1}(P) \supseteq I$. Thus there exists some prime $Q \supseteq I$ with $f(Q) = P$ so $P \in f(V(I))$.

(b) Take an ideal $J \subseteq B$. Then
\begin{align*}
f(f^{-1}(J))
&= f(\{a \in A \mid f(a) \in J\})\\
&= \{f(x) \mid x \in \{a \in A \mid f(a) \in J\}\}\\
&= \{f(x) \mid x \in A, f(x) \in J\}\\
&= J \cap f(A).
\end{align*}
But note that every $b \in B$ can be written as $f(a)u$ for $a \in A$ and $u \in B^*$. Thus $f(a) = bu^{-1}$ so the ideal generated by $f(f^{-1}(J))$ is just $J$ since all the elements of $f(A)$ can be expressed as terms in $b$. Now to show injectivity note that we have a left inverse for $f^*$ by taking $f(f^{-1})$ of some ideal and forming the ideal generated by this set. Since $f^*$ has a left inverse, it must be injective.

So $f^*$ will be bijective onto its image. Since $f^*$ is continuous from part (a), it now suffices to show that $f^*$ is a closed map.

Let $V(I)$ be a closed set in $\spec(B)$. Then $f^*(V(I))$ is the set of all preimages of primes $P \in \spec(B)$ which contain $I$. But this is simply $V(f^{-1}(I)) \cap f^*(\spec(B))$. This is a closed set in the subspace $f^*(\spec(B))$ of $\spec(A)$ so $f^*$ is a closed map and therefore a homeomorphism.

(c) Note that $i_{S,A}$ is a homomorphism from $A$ to $S^{-1}A$. Furthermore, for $a/s \in S^{-1}A$ we can write $a/s = i_{S,A}(a) (i_{S,A}(s))^{-1}$. Thus $i_{S,A}$ meets the conditions of both parts (a) and (b) so $i_{S,A}^*$ must be a homeomorphism onto its image. Furthermore, we know $\spec(S^{-1}A) = \{S^{-1}P \mid P \in \spec(A), P \cap S = \emptyset\}$. Thus $i_{S,A}^*(S^{-1}P) = i_{S,A}^{-1}(S^{-1}P) = P$, so if we pick $P \in \{P \in \spec(A) \mid P \cap S = \emptyset\}$ then $i_{S,A}^*(S^{-1}P) = P$, so this set must be its image.
\end{proof}

\begin{problem}
(a) Let $f : E \to F$ be an $A$-linear map of $A$-modules and $S \subseteq A$ a multiplicative set. Show that $S^{-1}(\ker f) = \ker(S^{-1}f)$ and $S^{-1}(f(E)) = S^{-1}f(S^{-1}E)$.\\
(b) Let $f : E' \to E$ and $g : E' \to E''$ be $A$-linear maps of $A$-modules. Suppose that for all $M \in \Max(A)$, the sequence
\[
\xymatrix{
E_M' \ar[r]^{f_M} & E_M \ar[r]^{g_M} & E_M''
}
\]
is exact. Show that the sequence
\[
\xymatrix{
E' \ar[r]^f & E \ar[r]^g & E''
}
\]
is exact.
\end{problem}
\begin{proof}
(a) Take $x/s \in S^{-1}(\ker f)$ so $x \in \ker f$ and $s \in S$. Then $f(x) = 0$ so $S^{-1}f(x/s) = f(x)/s = 0/s$ and $x/s \in \ker(S^{-1})$. On the other hand, if $x/s \in \ker(S^{-1}f)$ then $S^{-1}f(x/s) = 0/s = f(x)/s$ so $f(x) = 0$ and $x/s \in S^{-1}(\ker f)$.

Now take $x/s \in S^{-1}(f(E))$ so that $s \in S$ and $x = f(y)$ for some $y \in E$. So $x/s = f(y)/s$ which is the image of $y/s$ under $S^{-1}f$ and $S^{-1}(f(E)) \subseteq S^{-1}f(S^{-1}E)$. Conversely, suppose $x/s \in S^{-1}f(S^{-1}E)$. Then $x/s = f(y)/s$ for some $y \in E$ which means $x/s = S^{-1}f(y/s)$ and we have the other inclusion.

(b) By exactness and part (a) we know $(\ker f)_M = \ker f_M = f_M(E_M) = (f(E))_M$. In particular we have $(\ker f)_M/(f(E))_M = (\ker f/f(E))_M = 0$ for all $M \in \Max(A)$. Thus $\ker f/f(E) = 0$ and the second sequence is exact.
\end{proof}

\begin{problem}
Let $f : A \to B$ be a ring homomorphism and $f^* : \spec(B) \to \spec(A)$, the map $f^*(Q) = f^{-1}(Q)$, $Q \in \spec(B)$. Show that the closure of $f^*(\spec(B))$ in $\spec(A)$ is $V(\ker f)$. Thus deduce that $f^*(\spec(B))$ is dense in $\spec(A)$ if and only if $\ker f \subseteq \nil A$.
\end{problem}
\begin{proof}
The closure of $f^*(\spec(B))$ is $V(J)$ where $J = \bigcap_{P \in f^*(\spec(B))} P$. Then $J = \bigcap_{P \in \spec(B)} f^{-1}(P)$. If $x \in J$ then $f(x) \in f(J)$ and $f(J) \subseteq \bigcap_{P \in \spec(B)} P$. But then $f(x)^n = 0$ which means $f(x^n) = 0$ and $x \in \sqrt{\ker f}$. Then $V(\sqrt{\ker f}) = V(\ker f) = V(J)$.

If $f^*(\spec(B))$ is dense in $\spec(A)$ then the closure of $f^*(\spec(B))$ is $\spec(A)$. So then $\spec(A) = V(\ker f)$. Since $\nil(A)$ is the intersection of all $P \in \spec(A)$ we have $V(\nil(A)) = \spec(A) \subseteq V(\ker f)$ so $\ker f \subseteq \nil(A)$. Conversely, if $\ker f \subseteq \nil(A)$ then $V(\nil A) \subseteq V(\ker f)$ and $V(\nil(A)) = \spec(A)$ so the closure of $f^*(\spec(B))$ is $\spec(A)$ and $f^*(\spec(B))$ is dense in $\spec(A)$.
\end{proof}

\end{document}
