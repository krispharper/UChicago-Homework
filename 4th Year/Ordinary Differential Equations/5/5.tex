\documentclass{article}
\usepackage{amsmath,amsthm,amsfonts,amssymb,fullpage}

\newtheorem{problem}{Problem}

\begin{document}

\begin{flushright}
Kris Harper\\

MATH 27300\\

February 25, 2011
\end{flushright}

\begin{center}
Homework 5
\end{center}

\begin{problem}
Consider the linear initial-value problem $w' = a(z)w + b(z)$, $w(0) = 0$. Suppose that the functions $a$ and $b$ are analytic in a disk centered at the origin and of radius $R$. Compare the estimate of equation (4.32) with that which know from the linear theory.
\end{problem}
\begin{proof}
Note that in equation (4.29), $\rho$ and $\sigma$ are picked to be within the radius of convergence for $f(z,w)$. In our case this corresponds to picking two points in $(0,R)$. Then since $\rho < R$ and $\rho \exp(-\sigma/2M\rho) > 0$, we know $r < R$ in equation (4.32). So in this case we're guaranteed a radius of convergence at least as large as some $r < R$. On the other hand, in the linear case we're guaranteed a radius of convergence at least as large as $R$, so the later is a stronger statement.
\end{proof}

\begin{problem}
Show that, for any complex constant $\mu$, the functions $z^{\mu}$ and $z^{\mu} \ln z$ are linear independent over the complex numbers.
\end{problem}
\begin{proof}
Suppose we have $c_1z^{\mu} + c_2z^{\mu} \ln z = 0$. Differentiate to get
\[
0 = c_1 \mu z^{\mu - 1} + c_2 \mu z^{\mu - 1} \ln z + c_2 z^{\mu - 1} = z^{\mu - 1}(c_1 \mu + c_2(\mu \ln z + 1)).
\]
Assuming $z \neq 0$, we can divide by $z^{\mu}$ in the first equation and $z^{\mu - 1}$ in the second to obtain $0 = c_1 + c_2 \ln z$ and $0 = c_1 \mu + c_2(\mu \ln z + 1)$. If $\mu = 0$ then the functions $1$ and $\ln z$ are clearly linearly independent since $\ln z$ is nonconstant. Thus we may assume otherwise and solve for $c_1$ as $c_1 = -c_2 \ln z + c_2/\mu$. Putting this into the first equation gives $c_2/\mu = 0$ and $c_2 = 0$. Using the first equation again we have $c_1 = 0$ as well so the two functions are linearly independent.
\end{proof}

\begin{problem}
Find the influence function for the differential equation
\[
u'' + 2x^{-1}u' - 2x^{-2}u = r(x)
\]
on the interval $[1, \infty]$ of the real axis.
\end{problem}

The indicial equation is given by $\mu^2 + (2 - 1) \mu - 2 = 0$, which has roots $\mu = -2$ and $\mu = 1$. Then the influence function is defined as
\begin{align*}
G(z,s)
&=
\begin{cases}
\frac{u_1(s) u_2(z) - u_2(s) u_1(z)}{W(u_1,u_2;s)} & s < z\\
0 & s \geq z
\end{cases}\\
&=
\begin{cases}
\frac{zs^{-2} - sz^{-2}}{W(u_1,u_2;s)} & s < z\\
0 & s \geq z
\end{cases}
\end{align*}
where $W(u_1,u_2;s) = \exp \left ( - 2\int_{z_0}^z \zeta^{-1} d \zeta \right ) = K z^2$.

\begin{problem}
In equation (5.1) put $p(z) = 1/z^2$ and verify that the solution $w(z)$ cannot have the form given in equation (5.6).
\end{problem}
\begin{proof}
Making the substitution in equation (5.1) gives $w(z) + w'(z)/z^2 = 0$. If we write $w(z) = z^c \sum_{k=0}^{\infty} a_k z^k$ then
\[
z^{-2} w'(z) = z^{-2} \sum_{k=0}^{\infty} (k+c) a_k z^{k+c-1} = \sum_{k=0}^{\infty} (k+c) a_k z^{k+c-3}.
\]
Since $w(z) + w'(z)/z^2 = 0$, we must have $a_k + (k+c) a_{k+3} = 0$. Then $a_k/a_{k+3} = -(k+c)$. But this means $w(z)$ is only convergent at $z = 0$.
\end{proof}

\begin{problem}
Find the indicial equation for Bessel's equation (5.10) and find the indices. For the case when $n$ is a non-negative integer, obtain the series solution for the index with maximum real part. Do the indices differ by an integer?
\end{problem}

In this case $P_0 = 1$ and $Q_0 = -n^2$ so the indicial equation is $I(t) = t(t-1) + t - n^2 = t^2 - n^2 = (t+n)(t-n)$. The solutions are thus $\pm n$ which clearly differ by an integer.

The larger root of the indicial equation is $\mu = n$. Then $I(\mu + m) = m^2 + 2mn$. Now we can use the recursion relation
\[
a_m = \frac{-1}{I(\mu + m)} \sum_{k=0}^{m-1}((\mu + k) P_{m-k} + Q_{m-k}) a_k
\]
to find $a_m$. Note that we can arbitrarily pick $a_0 \neq 0$. Further, $P_0 = 1$ and $P_k = 0$ for all $k \neq 0$ and $Q_2 = 1$ and $Q_k = 0$ for all $k \neq 0$, $k \neq 2$. Using this we can deduce that
\[
a_m = \frac{-a_{m-2}}{m^2 + 2mn}
\]

\begin{problem}
Suppose the coefficients of equation (5.7) are analytic and single-valued in a punctured neighborhood of the origin. Suppose it is known that the function $w(z) = f(z)\ln z$ is a solution, where $f$ is analytic and single-valued in the punctured neighborhood. Deduce that $f$ is also a solution.
\end{problem}
\begin{proof}
Note that $w'(z) = f'(z) \ln z + f(z)/z$ and $w''(z) = f''(z) \ln z + f'(z)/z + f'(z)/z - f(z)/z^2$. Then
\begin{align*}
0
&= w'' + p(z)w' + q(z)w\\
&= f''(z) \ln z + 2f'(z)/z - f(z)/z^2 + p(z)f'(z) \ln z + p(z)f(z)/z + q(z)f(z)\ln z\\
&= \ln z (f''(z) + p(z) f'(z) + q(z) f(z)) + 2 f'(z)/z - f(z)/z^2 + p(z)f(z)/z.
\end{align*}
But now note that we must have the term inside the parentheses equal to $0$ since the terms outside of it must evaluate to $0$. This means precisely that $f$ satisfies the equation.
\end{proof}

\begin{problem}
Find the recursion relation for the coefficients $\{b_k\}$ in the power series expansion of the function $f_2$ appearing in equation (5.18). What determines the constant $C$?
\end{problem}

We have $w_2 = z^{\mu_2} f_2(z) + C w_1(z) \ln(z)$. Then $w_2'(z) = \mu_2 z^{\mu_2 -1} f_2(z) + z^{\mu} f_2'(z) + C w_1'(z) \ln(z) + C w_1(z)/z$ and
\[
w_2''(z) = \mu_2(\mu_2 - 1) z^{\mu_2 - 2} f_2(z) + 2\mu_2 z^{\mu_2 - 1}f_2'(z) + z^{\mu_2} f_2''(z) + Cw_1''(z) \ln(z) + 2Cw_1'(z)/z - Cw_1(z)/z^2.
\]
Since $z^2 w_2'' + zP(z)w_2' + Q(z)w_2 = 0$, we can now plug the above values in to get a recursion relation. Note that $f_2(z) = \sum_{k=0}^{\infty} b_k z^k$. Then we have
\begin{align*}
0
&= z^2 w_2''(z) + zP(z)w_2'(z) + Q(z)w_2(z)\\
&= \mu_2(\mu_2 - 1) z^{\mu_2 - 2} \sum_{k=0}^{\infty} b_k z^{k+2} + 2\mu_2 z^{\mu_2 - 1} \sum_{k=0}^{\infty} b_{k+1}(k+1)z^{k+2} + z^{\mu_2} \sum_{k=0}^{\infty} b_{k+2}(k+2)(k+1)z^{k+2}\\
&~~~~ + z^2 C w_1''(z) \ln(z) + 2 z C w_1'(z) - C w_1(z) + \mu_2 z^{\mu_2 - 1} \left ( \sum_{k=0}^{\infty} b_{k+1}z^{k+2} \right ) \left ( \sum_{k=-2}^{\infty} P_{k+2} z^{k+2} \right )\\
&~~~~ + z^{\mu_2} \left ( \sum_{k=0}^{\infty} b_k k z^k \right ) \left ( \sum_{k=0}^{\infty} P_k z^k \right ) + C z w_1'(z) \ln(z) P(z) + C w_1(z) P(z) + z^{\mu_2} \left ( \sum_{k=0}^{\infty} b_k z^k \right ) \left ( \sum_{k=0}^{\infty} Q_k z^k \right )\\
&~~~~ + C w_1(z) \ln(z) Q(z).
\end{align*}
Now we can use the expansion of $w_1(z) = \sum_{k=0}^{\infty} c_k z^k$ and find the coefficients of the expanded and combined power series. Since each term must be zero, we can solve for the one $b_{k+2}$ term to get a recursion relation. The constant $C$ is $c_n$ in the power series expansion for $\psi/f_1^2$, where $\psi(z) = z^{P_0} W(z)$. So $C$ depends on the Wronskian, $P_0$ and $f_1 = z^{- \mu_1} w_1(z)$.

\begin{problem}
Consider the equation
\[
(1-z) z^2 w'' + (z-4) z w' + 6w = 0.
\]
(a) Verify that this equation has a regular singular point at the origin.\\
(b) Find the indicial equation and the indices relative to this point.\\
(c) For the index with the greater real part, find the recursion relation for the coefficients in the series solution.\\
(d) Determine whether the second solution is given purely by a series solution (as in equation (5.12)) or involves in addition a logarithmic term (as in equation (5.18)).
\end{problem}

(a) Divide by $(1-z)$ to get $z^2 w'' + ((z-4)/(1-z)) z w' + 6/(1-z) = 0$. Since $(z-4)/(1-z)$ and $6/(1-z)$ are both analytic, this equation is of the form $z^2 w'' + zP(z)w' + Q(z)w = 0$, with $P$ and $Q$ analytic on a punctured disk, so the equation must have a regular singular point at the origin.

(b) The indicial equation is $I(\mu) = \mu(\mu - 1) + P_0 \mu + Q_0 = \mu(\mu - 1) - 4 \mu + 6 = 0$ This factors as $(\mu - 2)(\mu - 3)$.

(c) The index with greater real part is $\mu = 3$. Then $I(\mu + n) = n^2 + n$. Further note that $Q(z) = \sum_{k=0}^{\infty} 6z^k$ and $P(z) = -4 + \sum_{k=1}^{\infty} -3z^k$. Then we can use the following formula for the recursion relation. We have
\[
a_n = \frac{-1}{I(\mu + n)} \sum_{k=0}^{n-1} ((\mu + k) P_{n-k} + Q_{n-k}) a_k = \frac{-1}{n^2 + n} \sum_{k=0}^{n-1} ((3 + k) (-3) + 6) a_k = \frac{3}{n^2 + n} \sum_{k=0}^{n-1}(1-k)a_k.
\]

(d) Since the zeros of the indicial equation differ by an integer, the second solution has the structure of equation (5.18) and has a logarithmic term in it.

The following four problems relate to singular points at infinity. These are investigated by making the transformation $t = 1/z$ and investigating the singular points at $t = 0$. In each case determine whether the point in question is a point of analyticity, a regular singular point or an irregular singular point. In the case of a regular singular point, find the indices.

\begin{problem}
Bessel's equation $z^2 w'' + z w' + (n^2 - z^2)w = 0$ where $n$ is a constant.
\end{problem}

Making the substitution $t = 1/z$ we now have $w''/t^2 + w'/t + (n^2 - 1/t^2) w = 0$. Multiply through by $t^4$ to obtain $t^2 w''+ t^3 w' + (n^2 t^4 - t^2) w = 0$. We now see that $0$ is a regular singular point, since we can write this as $t^2 w'' + tP(t)w' + Q(t)w = 0$ where $P(t) = t^2$ and $Q(t) = n^2t^4 - t^2$ are both analytic. The indices will be given by solving the equation $I(t) = t(t-1) + P_0t + Q_0 = t(t-1) + 0 + 0 = t^2 - t$. This has solutions $t = 0$ and $t = 1$.

\begin{problem}
Legendre's equation $(1-z^2) w'' - 2z w' + \lambda w = 0$ where $\lambda$ is a constant.
\end{problem}

Making the substitution $t = 1/z$ we now have $(1 - 1/t^2) w'' - 2w'/t + \lambda w = 0$. If we divide by $(1 - 1/t^2)$ then we have $w'' - 2w'/(t(1 - 1/t^2)) + \lambda w/(1 - 1/t^2) = 0$. At this point we note that $-2/(t^2(1 - 1/t^2)) = -2/(t^2 - 1) = 2/(1 - t^2)$ and $\lambda/(t^2(1 - 1/t^2)) = \lambda/(t^2 - 1)$ are both analytic functions. Letting $P(t) = 2/(1 - t^2)$ and $Q(t) = \lambda/(t^2 - 1)$ we can write the equation as $t^2 w'' + tP(t)w' + Q(t)w = 0$. Thus $0$ is a regular singular point. To find the indices we solve the equation $I(t) = t(t-1) + P_0t + Q_0 = t^2 + t - \lambda$, where we've used the power series expansions for $P$ and $Q$ to fill in $P_0$ and $Q_0$. This has solutions $t = (-1 \pm \sqrt{1 + 4 \lambda})/2$.

\end{document}
