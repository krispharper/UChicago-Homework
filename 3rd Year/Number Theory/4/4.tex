\documentclass{article}
\usepackage{amsmath,amsthm,amsfonts,amssymb,fullpage}

\newtheorem{problem}{Problem}

\begin{document}

\begin{flushright}
Kris Harper\\

MATH 24200\\

May 10, 2010
\end{flushright}

\begin{center}
Homework 4
\end{center}

\begin{problem}
Use the Jacobi symbol to determine $(113/997)$, $(215/761)$, $(514/1093)$, $(401/757)$.
\end{problem}
\begin{proof}
We see $113$ and $997$ are both prime. Note $113 \equiv 1 \equiv 5 \pmod{4}$ so
\begin{align*}
(113/997) = (997/113) = (93/113)
&= (3/113)(31/113) = (113/3)(113/31) = (2/3)(20/31)\\
&= (2/3)(2/31)^2(5/31) = (2/3)(5/31)\\
&= (2/3)(31/5) = (2/3)(1/5) = (-1)(1) = -1.
\end{align*}
We see $761$ is prime. Note also $761 \equiv 1 \equiv 5 \pmod{4}$ and $43 \equiv 3 \pmod{4}$. Then
\begin{align*}
(215/761) = (5/761)(43/761) = (761/5)(761/43)
&= (1/5)(30/43) = (2/43)(3/43)(5/43)\\
&= -(-1)^{(43^2-1)/8}(43/3)(43/5)\\
&= (1/3)(3/5) = (1)(-1) = -1.
\end{align*}
We see $1093$ is prime. Note $1093 \equiv 5 \pmod{8}$, $1093 \equiv 1 \equiv 65 \pmod{4}$ and $31 \equiv 3 \pmod{4}$. Then
\begin{align*}
(514/1093) = (2/1093)(257/1093)
&= -(1093/257) = -(65/257) = -(257/65)\\
&= -(62/65) = -(65/62) = -(3/62) = -(3/2)(3/31)\\
&= (31/3) = (1/3) = 1.
\end{align*}
We see both $401$ and $757$ are prime. Note also $401 \equiv 1 \equiv 45 \pmod{4}$. Then
\begin{align*}
(401/757) = (757/401) = (356/401)
&= (401/356) = (45/356) = (356/45)\\
&= (41/45) = (45/41)\\
&= (4/41) = (2/41)^2 = 1.
\end{align*}
\end{proof}

\begin{problem}
An integer is called a biquadratic residue modulo $p$ if it is congruent to a fourth power. Using the identity $x^4 + 4 = ((x+1)^2 + 1)((x-1)^2 + 1)$ show that $-4$ is a biquadratic residue modulo $p$ iff $p \equiv 1 \pmod{4}$.
\end{problem}
\begin{proof}
We want to find a solution to the equation $-4 \equiv x^4 \pmod{p}$ or equivalently $x^4 + 4 \equiv ((x+1)^2 + 1)((x-1)^2 + 1) \equiv 0 \pmod{p}$. Note then that this has a solution if and only if one of the factors $((x+1)^2+1)$ or $((x-1)^2+1)$ is congruent to $0$ modulo $p$. Thus we either have $(x+1)^2+1 \equiv 0 \pmod{p}$ or $(x-1)^2+1 \equiv 0 \pmod{p}$ In either case $-1$ is a quadratic residue modulo $p$ which is true if and only $p \equiv 1 \pmod{4}$.
\end{proof}

\begin{problem}
\label{parts}
This exercise and Exercises 27 and 28 give Dirichlet's beautiful proof that $2$ is a biquadratic residue modulo $p$ iff $p$ can be written in the form $A^2 + 64B^2$, where $A, B \in \mathbb{Z}$. Suppose that $p \equiv 1 \pmod{4}$. Then $p = a^2 + b^2$ by Exercise 24. Take $a$ to be odd. Prove the following statements:\\
(a) $(a/p) = 1$.\\
(b) $((a+b)/p) = (-1)^{((a+b)^2-1)/8}$.\\
(c) $(a+b)^2 \equiv 2ab \pmod{p}$.\\
(d) $(a+b)^{(p-1)/2} \equiv (2ab)^{(p-1)/4} \pmod{p}$.
\end{problem}
\begin{proof}
(a) From part (c) and the fact that $p \equiv 1 \pmod{8}$ we know $1 = (2ab/p) = (2/p)(a/p)(b/p) = (a/p)(b/p)$ so $(a/p) = (b/p)$. But it's not possible that both $a$ and $b$ are nonresidues modulo $p$ so we must have $(a/p) = 1$.

(b) Note that $2p = (a+b)^2 + (a-b)^2$ and $a+b$ is odd. Thus $(2p/(a+b)) = 1$ since $2 \nmid a+b$. Then $1 = (2/(a+b)) (p/(a+b)) = (-1)^{((a+b)^2-1)/8} ((a+b)/p)$ since $p \equiv 1 \pmod{4}$.

(c) We have $(a+b)^2 \equiv a^2 + 2ab + b^2 \equiv p + 2ab \equiv 2ab \pmod{p}$.

(d) Since $p \equiv 1 \pmod{4}$ we know $k = (p-1)/4$ is an integer. Then from part (c) we have $(a+b)^{2k} \equiv (2ab)^k \pmod{p}$. Putting in the value of $k$ gives the result.
\end{proof}

\begin{problem}
\label{f}
Suppose that $f$ is such that $b \equiv af \pmod{p}$. Show that $f^2 \equiv -1 \pmod{p}$ and that $2^{(p-1)/4} \equiv f^{ab/2} \pmod{p}$.
\end{problem}
\begin{proof}
Note that $b^2 \equiv a^2f^2 \pmod{p}$ and that $0 \equiv a^2 + b^2 \equiv a^2 + a^2f^2 = a^2(1 + f^2)$. Since $a^2$ is not equivalent to $0$ modulo $p$ we see that $0 \equiv 1 + f^2 \pmod{p}$ and $f^2 \equiv -1 \pmod{p}$. Raising this to the power $ab/2$ and using Problem~\ref{parts} gives the second result.
\end{proof}

\begin{problem}
Show that $x^4 \equiv 2 \pmod{p}$ has a solution for $p \equiv 1 \pmod{4}$ iff $p$ is of the form $A^2 + 64B^2$.
\end{problem}
\begin{proof}
If $p = A^2 + 64B^2$ then let $a = A$ and $b = 8B$ so that $p = a^2 + b^2$. Then using Problem~\ref{f} we know there exists $f$ such that $f^{ab/2} \equiv 2^{(p-1)/2} \pmod{p}$. Since $4 \mid ab/2$ we see that $x^4 \equiv 2 \pmod{p}$ is solvable. Conversely, suppose that $x^4 \equiv 2 \pmod{p}$ is solvable. Since $p \equiv 1 \pmod{4}$ we know $p = a^2 + b^2$ and we only need to show that $8 \mid b$. But this must be the case since Problem~\ref{f} tells us that $2^{(p-1)/4} \equiv f^{ab/2} \pmod{p}$ and $2 \equiv x^4 \pmod{p}$ for some $x$. Raising $2$ the the power $(p-1)/4$ shows that $4 \mid ab/2$. Since $a$ is odd we must have $8 \mid b$.
\end{proof}

\begin{problem}
Show that $\sqrt{2} + \sqrt{3}$ is an algebraic integer.
\end{problem}
\begin{proof}
Note that $\sqrt{2}$ is a root to $x^2 - 2$ and $\sqrt{3}$ is a root to $x^2 - 3$. These are both monic polynomials with coefficients in $\mathbb{Z}$, so $\sqrt{2}$ and $\sqrt{3}$ are both algebraic integers. Since the algebraic integers form a ring, it follows that $\sqrt{2} + \sqrt{3}$ is also an algebraic integer.
\end{proof}

\begin{problem}
Let $\alpha$ be an algebraic number. Show that there is an integer $n$ such that $n \alpha$ is an algebraic integer.
\end{problem}
\begin{proof}
Since $\alpha$ is algebraic there exists some polynomial $p(x) \in \mathbb{Q}[x]$ such that $\alpha^m + a_1 \alpha^{m-1} + \dots + a_m = 0$ and $a_i \in \mathbb{Q}$. Now find the least common multiple of the $a_i$ and call it $n$. Multiply our polynomial by $n$ so we have $n \alpha^m + b_1 \alpha^{m-1} + \dots + b_m = 0$ where $b_i \in \mathbb{Z}$. Finally, multiply both sides by $n^{m-1}$ so we have $n^m \alpha^m + b_1 n^{m-1} \alpha^{m-1} + b_2 n^{m-1} \alpha^{m-2} + \dots + b_m n^{m-1} = 0$. We can now pass the appropriate exponent of $n$ inside each exponent of $\alpha$ for every term which results in the equation $(n \alpha)^m + b_1 (n \alpha)^{m-1} + b_2 n (n \alpha)^{m-2} + \dots + b_{m-1} n^{m-2} (n \alpha) + b_m n^{m-1} = 0$. Since each $b_i n^{i-1}$ is an integer we see that $n \alpha$ is an algebraic integer.
\end{proof}

\begin{problem}
If $\alpha$ and $\beta$ are algebraic integers, prove that any solution to $x^2 + \alpha x + \beta = 0$ is an algebraic integer. Generalize this result.
\end{problem}
\begin{proof}
Since $\alpha$ and $\beta$ are algebraic integers they satisfy polynomials in $\mathbb{Z}[x]$ of the form $\alpha^n + a_{n-1} \alpha^{n-1} \dots + a_0 = 0$ and $\beta^m + b_{m-1} \beta^{m-1} + \dots + b_0 = 0$. Let $\gamma$ be a root of $x^2 + \alpha x + \beta$ and let $V$ be the $\mathbb{Z}$ module generated by $\alpha^i \beta^j \gamma^k$ where $0 \leq i \leq n$, $0 \leq j \leq m$ and $0 \leq k \leq 1$. Then consider $\gamma \alpha^i \beta^j \gamma^k$. If $k = 0$ then this is clearly in $V$. If $k = 1$ then $\gamma \alpha^i \beta^j \gamma^k = \alpha^i \beta^j \gamma^2 = \alpha^i \beta^j (-\alpha \gamma - \beta) = -\alpha^{i+1} \beta^j \gamma^k - \alpha^i \beta^{j+1}$. This is also definitely an element of $V$ except for the possibility that $i = n$ or $j = m$. In this case we simply rewrite $\alpha^n = -(a_{n-1} \alpha^{n-1} + \dots + a_0)$ and $\beta^m = -(b_{m-1} \beta^{m-1} + \dots + b_0)$. Expanding this out gives an element of $V$. This statement generalizes so that if $\gamma$ is a root of $x^n + \alpha_{n-1} x^{n-1} + \dots + \alpha_0$ where the $\alpha_i$ are algebraic integers then $\gamma$ is an algebraic integer.
\end{proof}

\begin{problem}
Let $\omega = e^{2 \pi i/3}$. $\omega$ satisfies $x^3 - 1 = 0$. Show that $(2 \omega + 1)^2 = -3$ and use this to determine $(-3/p)$ by the method of section $2$.
\end{problem}
\begin{proof}
We have $(2 \omega + 1)^2 = 4 \omega^2 + 4 \omega + 1 = 4(\omega^2 + \omega + 1) - 3 = -3$. Note that if $p = 3$ then $(-3/p) = 0$ so we can assume $p \neq 3$. Let $\tau = 2 \omega + 1$. Then $\tau^{p-1} = (\tau^2)^{(p-1)/2} = (-3)^{(p-1)/2} \equiv (-3/p) \pmod{p}$ and $\tau^p \equiv (-3/p) \tau \pmod{p}$. Note $\tau^p = (2 \omega + 1)^p \equiv 2^p \omega^p + 1 \pmod{p}$. Since $\omega^3 = 1$ we have $2^p \omega^p + 1 \equiv 2^p \omega + 1 \equiv 2 \omega + 1 \equiv \tau \pmod{p}$ if $p \equiv 1 \pmod{3}$ and $2^p \omega^p \equiv 2^p \omega^2 + 1 \equiv 2 (-\omega - 1) + 1 \equiv -2 \omega - 1 \equiv -\tau \pmod{p}$ if $p \equiv 2 \pmod{3}$. We can now express this as $(-1)^{\varepsilon} \tau \equiv (-3/p) \tau \pmod{\tau}$ where $\varepsilon = 3((p/3) - \lfloor (p/3) \rfloor)-1$. Multiply both sides by $\tau$ and note that we can divide by $-3$ to get $(-3/p) = (-1)^{\varepsilon}$.
\end{proof}

\begin{problem}
By calculating $\sum_t (1 + (t/p)) \zeta^t$ in two ways prove that $g = \sum_t \zeta^{t^2}$.
\end{problem}
\begin{proof}
Note that
\[
g = \sum_t \left ( \frac{t}{p} \right ) \zeta^t = \sum_t \zeta^t + \sum_t \left ( \frac{t}{p} \right ) \zeta^t = \sum_t \left ( 1 + \left ( \frac{t}{p} \right ) \right ) \zeta^t = \sum_t \zeta^{t^2}
\]
since $1 + (t/p)$ is the number of solutions to $x^2 \equiv t \pmod{p}$.
\end{proof}

\end{document}