\documentclass{article}
\usepackage{amsmath,amsthm,amsfonts,amssymb,fullpage}

\newtheorem{problem}{Problem}

\newcommand{\ord}{\textup{ord}}

\begin{document}

\begin{flushright}
Kris Harper\\

MATH 24200\\

April 12, 2010
\end{flushright}

\begin{center}
Homework 1
\end{center}

\begin{problem}
\label{roots}
Prove that $\sqrt[n]{m}$ is irrational if $m$ is not the $n$th power of an integer.
\end{problem}
\begin{proof}
Suppose $\sqrt[n]{m} = a/b$ where $a,b \in \mathbb{Z}$ and $(a,b) = 1$. Then $mb^n = a^n$. We can uniquely prime factor $a$ and $b$ as $p_1^{a_1} \dots p_r^{a_r}$ and $q_1^{b_1} \dots q_s^{b_s}$. Then we can group the prime factors of $a^n$ as $n$ identical groups of $p_1^{a_1} \dots p_r^{a_r}$. It follows that $mb^n$ can be written as the product of $n$ identical groups of prime powers. But then each of these groups must contain $q_1^{b_1} \dots q_s^{b_s}$ since this is the prime factorization of $b$. Therefore $m$ must have a prime factorization such that it can be evenly divided into these $n$ groups. In other words, we must have $m = c^n$ for some integer $c$.
\end{proof}

\begin{problem}
Suppose $a^2 + b^2 = c^2$ with $a,b,c \in \mathbb{Z}$. For example $3^2 + 4^2 = 5^2$ and $5^2 + 12^2 = 13^2$. Assume that $(a,b) = (b,c) = (c,a) = 1$. Prove that there exist integers $u$ and $v$ such that $c - b = 2u^2$ and $c + b = 2v^2$ and $(u,v) = 1$ (there is no loss in generality in assuming that $b$ and $c$ are odd and $a$ is even). Consequently $a = 2uv$, $b = v^2 - u^2$ and $c = v^2 + u^2$. Conversely show that if $u$ and $v$ are given, then the three numbers $a$, $b$ and $c$ given by these formulas satisfy $a^2 + b^2 = c^2$.
\end{problem}
\begin{proof}
Since $c$ and $b$ are relatively prime and both odd we can write $(c-b)(c+b)$ as $4(p_1^{a_1}p_2^{a_2} \dots p_n^{a_n})(q_1^{b_1}q_2^{b_2} \dots q_m^{b_m})$ where $2p_1^{a_1} \dots p_n^{a_n} = c-b$, $2q_1^{b_1} \dots q_m^{b_m} = c+b$, $p_i$, $q_i$ are primes and $p_i \neq q_j$ for all $i$ and $j$. That is, $c-b$ and $c+b$ are relatively prime except for a factor of $2$. Now write $4(a/2)^2 = (c-b)(c+b)$. Now associate each of the factors corresponding to $c-b$ with the same prime factors in $(a/2)^2$. Since $c-b$ and $c+b$ share no common factors (except for $2$) we see that none of the squares get split up in this process. Thus $c-b = 2r_1^{2c_1} \dots r_{n'}^{2c_{n'}} = 2u^2$ where $u = r_1^{c_1} \dots r_{n'}^{c_{n'}}$. Likewise $c+b = 2s_1^{2d_1} \dots s_{m'}^{2d_{m'}} = 2v^2$ where $v = s_1^{d_1} \dots s_{m'}^{d_{m'}}$. Since $(c-b,c+b) = 2$ and it immediately follows that $(u,v) = 1$.

Conversely, suppose we are given such $u$ and $v$. Then $a^2 = 4u^2v^2 = (c-b)(c+b) = c^2-b^2$ so we have the desired formula.
\end{proof}

\begin{problem}
If $a^n-1$ is prime, show that $a = 2$ and that $n$ is a prime. Assume $a > 0$ and $n > 1$
\end{problem}
\begin{proof}
Note that $a^n-1 \neq 2$ since the equation $a^n = 3$ has no integer solutions by Problem~\ref{roots}. Then $a^n-1 = p$ where $p$ is necessarily odd and so $a^n = p+1$ which shows $a^n$ is even. Therefore $2 \mid a^n$ which means $2 \mid a$ since $2$ is prime. We can then write $a^n = 2^nm^n$ for some positive integer $m$. But now note that
\[
2^nm^n - 1 = (2m-1)(1 + 2m + 2^2m^2 + \dots + 2^{n-1}m^{n-1})
\]
so if $m \neq 1$ we have a factorization of $p$. Thus $a = 2$. A similar argument shows that $n$ must be prime because if $n = rs$ then we have
\[
2^n - 1 = 2^r2^s - 1 = (2^r - 1)(1 + 2^r + 2^{2r} + \dots + 2^{rs - r}).
\]
In order for this to be prime we must have $r = 1$ so that $n$ is prime.
\end{proof}

\begin{problem}
Prove that $\frac{1}{2} + \frac{1}{3} + \dots + \frac{1}{n}$ is not an integer.
\end{problem}
\begin{proof}
Find $k$ such that $2^k \leq n \leq 2^{k+1}$. Now find the lowest common multiple of $\{2, \dots , 2^k-1, 2^k+1, \dots , n\}$. This will necessarily be of the form $2^{k-1} m$ where $m$ is an odd integer. Now multiply this by the sum in question. We have
\[
2^{k-1}m \left ( \frac{1}{2} + \frac{1}{3} + \dots + \frac{1}{n} \right ).
\]
Every term in this product is an integer except $2^{k-1}m (1/2^k) = m/2$ since $m$ is odd. Thus the sum in question cannot be an integer.
\end{proof}

\begin{problem}
Show that $3$ is divisible by $(1-\omega)^2$ in $\mathbb{Z}[\omega]$.
\end{problem}
\begin{proof}
We have $(1 - \omega)^2 = 1 - 2 \omega + \omega^2 = 1 - 2 \omega + (-\omega - 1) = -3 \omega$. Now multiply both sides by $\omega + 1$. On the left we have $(\omega + 1)(1 - \omega)^2$ and on the right we have $3(-\omega (\omega + 1)) = 3(-\omega^2 - \omega) = 3(\omega + 1 - \omega) = 3$. Therefore $3 = (\omega+1)(1-\omega)^2$.
\end{proof}

\begin{problem}
\label{unit}
For $\alpha = a + b\omega \in \mathbb{Z}[\omega]$ we defined $\lambda(\alpha) = a^2 - ab + b^2$. Show that $\alpha$ is a unit iff $\lambda(\alpha) = 1$. Deduce that $1$, $-1$, $\omega$, $-\omega$, $\omega^2$ and $-\omega^2$ are the only units in $\mathbb{Z}[\omega]$.
\end{problem}
\begin{proof}
Suppose $\alpha = a + b \omega$ is a unit with inverse $\beta = c + d \omega$. Note that $\lambda$ is multiplicative so we have $1 = \lambda(\alpha \beta) = \lambda (\alpha) \lambda (\beta) = (a^2-ab+b^2) (c^2-cd+d^2)$. Since each of these factors is a positive integer we must have $a^2-ab+b^2 = 1$ so that $\lambda (\alpha) = 1$.

Conversely, suppose $\lambda (\alpha) = 1$. Then $a^2 - ab + b^2 = 1$. We wish to find $\beta$ such that $\alpha\beta = 1$. Multiplying out the terms we get the equations $ac-bd = 1$ and $ad + bc - bd = 0$. Solving the first equation for $c$ and plugging it into the second gives us $a^2c - a + b^2c - abc + b = 0$. Using the fact that $a^2 - ab + b^2 = 1$ we now have $c = a-b$. We can then use this to find $d = -b$. It's a quick check to see that $\beta = (a-b) - b \omega$ is $\alpha^{-1}$. Thus $\alpha$ is a unit.
\end{proof}

\begin{problem}
Define $\mathbb{Z}[\sqrt{-2}]$ as the set of all complex numbers of the form $a + b\sqrt{2}$, where $a,b \in \mathbb{Z}$. Show that $\mathbb{Z}[\sqrt{-2}]$ is a ring. Define $\lambda(\alpha) = a^2 + 2b^2$ for $\alpha = a + b \sqrt{2}$. Use $\lambda$ to show $\mathbb{Z}[\sqrt{-2}]$ is a Euclidean domain.
\end{problem}
\begin{proof}
Since $\mathbb{Z}[\sqrt{-2}]$ is contained in the ring $\mathbb{C}$ we need only show that $\mathbb{Z}[\sqrt{-2}]$ is nonempty and closed under subtraction and multiplication. Let $\alpha = a + b \sqrt{-2}$ and $\beta = c + d \sqrt{-2}$. Then $\alpha - \beta = (a-c) + (b-d) \sqrt{-2}$ which is in $\mathbb{Z}[\sqrt{-2}]$. Likewise $\alpha \beta = (ac - 2bd) + (ad + bc)\sqrt{-2}$ which is also in $\mathbb{Z}[\sqrt{-2}]$. Thus $\mathbb{Z}[\sqrt{-2}]$ is a ring.

Let $\alpha$ and $\beta$ be as before and suppose $\beta \neq 0$. Now $\alpha/\beta = r + s\sqrt{2}$ where $r,s \in \mathbb{Q}$. Choose integers $m,n \in \mathbb{Z}$ such that $|r-m| \leq \frac{1}{2}$ and $|s-n| \leq \frac{1}{2}$. Let $\delta = m + ni$ so that $\delta \in \mathbb{Z}[\sqrt{-2}]$. We have $\lambda (\alpha/\beta - \delta) = (r-m)^2 + 2(s-n)^2 \leq \frac{1}{4} + 2\frac{1}{4} = \frac{3}{4}$. Let $\rho = \alpha - \beta \delta$. Then $\rho \in \mathbb{Z}[\sqrt{-2}]$ and we must have either $\rho = 0$ or
\[
\lambda (\rho) = \lambda(\beta((\alpha/\beta) - \delta)) \leq \lambda(\beta) \lambda((\alpha/\beta) - \delta) \leq \frac{3}{4} \lambda (\beta) < \lambda (\beta).
\]
Therefore $\mathbb{Z}[\sqrt{-2}]$ is a Euclidean domain by $\lambda$.
\end{proof}

\begin{problem}
\label{root2}
Show that the only units in $\mathbb{Z}[\sqrt{-2}]$ are $1$ and $-1$.
\end{problem}
\begin{proof}
Suppose $\alpha \beta = 1$ with $\alpha = a + b \sqrt{-2}$ and $\beta = c + d \sqrt{-2}$. Then $ac - 2bd = 1$ and $ad + bc = 0$. Solving the second equation for $c$ and plugging it into the first we see that $d = -b/(a^2 + 2b^2)$. Since the denominator is necessarily greater than $b$ we see that this can only be an integer if $a^2 + 2b^2 = 1$. But this can only happen if $b = 0$ and $a = \pm 1$.
\end{proof}

\begin{problem}
Suppose $\pi \in \mathbb{Z}[i]$ and that $\lambda(\pi) = p$ is a prime in $\mathbb{Z}$. Show that $\pi$ is a prime in $\mathbb{Z}[i]$. Show that the corresponding result holds in $\mathbb{Z}[\omega]$ and $\mathbb{Z}[\sqrt{-2}]$.
\end{problem}
\begin{proof}
Suppose $\pi = \alpha \beta$. Then $p = \lambda (\pi) = \lambda (\alpha \beta) = \lambda (\alpha) \lambda (\beta)$. Since $\lambda (\alpha)$ and $\lambda(\beta)$ are both integers, we see that one of them must be $1$ which means $\alpha$ or $\beta$ is a unit in $\mathbb{Z}[i]$. Thus $\pi$ must be irreducible and therefore prime since $\mathbb{Z}[i]$ is a P.I.D.. The exact same proof holds for $\mathbb{Z}[\omega]$ and $\mathbb{Z}[\sqrt{-2}]$ using Problem~\ref{unit} and Problem~\ref{root2} because $\lambda$ is multiplicative in these cases too.
\end{proof}

\begin{problem}
\label{order}
For a rational number $r$ let $[r]$ be the largest integer less than or equal to $r$, e.g., $[\frac{1}{2}] = 0$, $[2] = 2$ and $[3\frac{1}{3}] = 3$. Prove $\ord_p n! = [n/p] + [n/p^2] + [n/p^3] + \dots$.
\end{problem}
\begin{proof}
Consider the set of pairs $(s,t)$ where $p^st \leq n$. If we fix $s$ we can increment $t$ starting at $t = 1$ and stopping when $p^st > n$. Then there's some value $t_s$ such that $p^st_s \leq n$ and $p^s(t_s+1) > n$. Moreover, it's clear that $[n/p^s] = t_s$. But note that the pairs $(s,t)$ for all integer values of $s > 0$ and $1 \leq t \leq t_s$ together represent all the possible divisors of $n!$ which include a factor of $p$. Therefore to count factors of $p$ in $n!$ we need only count these pairs. But we've already seen that for each $s$ there are $t_s$ pairs so the total is simply $\sum_{s=1}^{\infty} t_s = \sum_{s=1}^{\infty} [n/p^s]$.
\end{proof}

\begin{problem}
\label{product}
Deduce from Exercise 6 that $\ord_p n! \leq n/(p-1)$ and that $\sqrt[n]{n!} \leq \prod{p \mid n!} p^{1/(p-1)}$.
\end{problem}
\begin{proof}
We know each term in the series in Problem~\ref{order} is less than or equal to $n/p^k$. Thus $\ord_p n! \leq \sum_{k=1}^{\infty} n/p^k = n/(p-1)$.

Since the order of each prime appearing in $n!$ is less than or equal to $n/(p-1)$ it follows that
\[
n! \leq \prod_{p \mid n!} p^{\frac{n}{p-1}} = \left ( \prod_{p \mid n!} p^{\frac{1}{p-1}} \right )^n
\]
so $\sqrt[n]{n!} \leq \prod_{p \mid n!} p^{1/(p-1)}$.
\end{proof}

\begin{problem}
Use Exercise 7 to show that there are infinitely many primes.
\end{problem}
\begin{proof}
Suppose there are only finitely many primes $p_1, \dots p_m$. Let $n = p_1p_2 \dots p_m$. Using Problem~\ref{product} and the fact that $n^n \leq (n!)^2$ we have
\[
n^n \leq (n!)^2 \leq (n!)^n \leq \prod_{p \mid n!} p^{\frac{n}{p-1}} = \prod_{i=1}^{m} p_i^{\frac{n}{p_i-1}} = \left ( \prod_{i=1}^{m} p_i^{\frac{1}{p-1}} \right )^n < n^n
\]
since $1/(p-1) \leq 1$. This is a contradiction and so there must be infinitely many primes.
\end{proof}

\begin{problem}
\label{zeta}
Consider the function $\zeta(s) = \sum_{n=1}^{\infty} 1/n^s$. $\zeta(s)$ is called the Riemann zeta function. It converges for $s > 1$. Prove the formal identity (Euler's identity) $\zeta(s) = \prod_p (1-(1/p^s))^{-1}$.
\end{problem}
\begin{proof}
For each prime $p$ multiply both sides of $\zeta(s) = \sum_{n=1}^{\infty} 1/n^s$ by $1/p^s$ and then subtract the result from the previous result. We have
\[
\zeta(s) = \frac{1}{1^s} + \frac{1}{2^s} + \frac{1}{3^s} + \frac{1}{4^s} + \dots
\]
and
\[
\frac{1}{2^s} \zeta(s) = \frac{1}{2^s} + \frac{1}{4^s} + \frac{1}{6^s} + \frac{1}{8^s} + \dots.
\]
Subtracting we have
\[
\left (1 - \frac{1}{2^s} \right ) \zeta(s) = \frac{1}{1^s} + \frac{1}{3^s} + \frac{1}{5^s} + \frac{1}{7^s} + \dots.
\]
Repeating the process for $p=3$ we get
\[
\left (1 - \frac{1}{3^s} \right ) \left ( 1 - \frac{1}{2^s} \right ) \zeta(s) = \frac{1}{1^s} + \frac{1}{5^s} + \frac{1}{7^s} + \frac{1}{11^s} + \dots.
\]
Applying this to every prime we arrive at the formula
\[
\prod_p \left (1 - \frac{1}{p^s} \right ) \zeta (s) = 1
\]
which then gives the desired formula $\zeta(s) = \prod_p (1-(1/p^s))^{-1}$.
\end{proof}

\begin{problem}
Verify the formal identities\\
(a) $\zeta(s)^{-1} = \sum_{n=1}^{\infty} \mu(n)/n^s$.\\
(b) $\zeta(s)^2 = \sum_{n=1}^{\infty} \nu(n)/n^s$.\\
(c) $\zeta(s)\zeta(s-1) = \sum_{n=1}^{\infty} \sigma(n)/n^s$.
\end{problem}
\begin{proof}
(a) Using Problem~\ref{zeta} we can write $\zeta(s)^{-1} = \prod_{p} (1-(1/p^s))$. If we expand the right hand side we see that we get get a sum of terms $1/n^s$ where $n$ is a squarefree integer. We know $n$ must be squarefree because each prime $p$ appears only once in the product so we will never multiply a prime by itself. Furthermore if $n$ has an odd number of prime factors then the term $1/n^s$ will be negative and if it has an even number of prime factors then it will be positive since terms being multiplied have a $-1/p^s$ term. This explicitly gives the formula $\sum_{n=1}^{\infty} \mu(n)/n^s$.

(b) For some $0 \leq k < s$ we have
\[
\zeta(s)\zeta(s-k) = \sum_{u=1}^{\infty} \frac{1}{u^s} \sum_{v=1}^{\infty} \frac{v^k}{v^s} = \sum_{u,v}^{\infty} \frac{v^k}{(uv)^s} = \sum_{n=1}^{\infty} \frac{1}{n^s} \sum_{uv=n} v^k = \sum_{n=1}^{\infty} \frac{1}{n^s} \sum_{d \mid n} d^k.
\]
When $k = 0$ we get the formula $\zeta(s)^2 = \sum_{n=1}^{\infty} \nu(n)/n^s$.

(c) This is a special case of the formula in part (b). Putting in $k = 1$ gives $\zeta(s)\zeta(s-1) = \sum_{n=1}^{\infty} \sigma(n)/n^s$.
\end{proof}

\end{document}