\documentclass{article}
\usepackage{amsmath,amsthm,amsfonts,amssymb,fullpage}

\input xy
\xyoption{all}

\newtheorem{problem}{Problem}

\newcommand{\im}{\textup{im}\,}

\begin{document}

\begin{flushright}
Kris Harper\\

MATH 26300\\

June 1, 2010
\end{flushright}

\begin{center}
Homework 8
\end{center}

\begin{problem}
For finite $CW$ complexes $X$ and $Y$, show that $\chi(X \times Y) = \chi(X) \chi(Y)$.
\end{problem}
\begin{proof}
We know $\chi(X) = \sum_n (-1)^n a_n$ and $\chi(Y) = \sum_n (-1)^n b_n$ where $a_n$ and $b_n$ are the number of $n$-cells in $X$ and $Y$ respectively. The space $X \times Y$ has cells $c_n^{\alpha} \times c_m^{\beta}$ where $c_n^{\alpha}$ ranges over every cell in $X$ and $c_m^{\beta}$ ranges over every cell in $Y$. Thus $\chi(X \times Y) = \sum_n \sum_{i+j = n} (-1)^na_ib_j$. This is also the product $\chi(X) \chi(Y) = \left ( \sum_n (-1)^n a_n \right ) \left (\sum_n (-1)^n b_n \right )$.
\end{proof}

\begin{problem}
If a finite $CW$ complex is the union of subcomplexes $A$ and $B$, show that $\chi(X) = \chi(A) + \chi(B) - \chi(A \cap B)$.
\end{problem}
\begin{proof}
Let $a_n$, $b_n$ and $c_n$ denote the number of $n$-cells in $A$, $B$ and $A \cap B$ respectively. Then we can write the number of $n$-cells in $X$ as $a_n + b_n - c_n$. We then have
\[
\chi(X) = \sum_n (-1)^n (a_n + b_n - c_n) = \sum_n (-1)^n a_n + \sum_n (-1)^n b_n - \sum_n (-1)^n c_n = \chi(A) + \chi(B) - \chi(A \cap B).
\]
\end{proof}

\begin{problem}
\label{cover}
For a finite $CW$ complex and $p : \widetilde{X} \to X$ an $n$-sheeted covering space, show that $\chi(\widetilde{X}) = n \chi(X)$.
\end{problem}
\begin{proof}
Every $k$-cell $\sigma$ in $X$ has a map $f_{\sigma} : D^k \to X$. Since $D^k$ has trivial fundamental group we get a lift $\widetilde{f_{\sigma}} : D^k \to \widetilde{X}$. Since $p$ is an $n$-sheeted covering space, there are $n$ such distinct lifts and so $\widetilde{X}$ must have $n$ times as many $k$-cells as $X$. Thus $\chi(\widetilde{X}) = n \chi(X)$.
\end{proof}

\begin{problem}
Show that if the closed orientable surface $M_g$ of genus $g$ is a covering space of $M_h$, then $g = n(h-1) + 1$ for some $n$, namely, $n$ is the number of sheets in the covering.
\end{problem}
\begin{proof}
Let $M_g$ be a covering space for $M_h$ and suppose the number of sheets in the cover is $n$. Then using Problem~\ref{cover} we know $2-2g = \chi(M_g) = n \chi(M_h) = n(2-2h)$. We can then solve for $g$ by dividing by $-2$ and adding $1$ to get $g = n(h-1) + 1$ as desired. In particular, $h = 1$ then $g = 1$ so the only closed orientable surface with a genus which is a covering space for $M_1$ is $M_1$ itself.
\end{proof}

\begin{problem}
The surface $M_g$ embedded in $\mathbb{R}^3$ in the standard way, bounds a compact region $R$. Two copies of $R$, glued together by the identity map between their boundary surfaces $M_g$, form a closed $3$-manifold $X$. Compute the homology groups of $X$ via the Mayer-Vietoris sequence for this decomposition of $X$ into two copies of $R$. Also compute the relative homology groups $H_i(R, M_g)$.
\end{problem}
\begin{proof}
Note that if we choose $g$ circles on $M_g$ which go around the ``holes'' of $M_g$, we see that $R$ is homotopy equivalent to $\bigvee_{i=1}^g S^1$. Now let $U$ be one copy of $R$ along with a neighborhood the boundary $M_g$ and let $V$ be the other copy of $R$ along with a neighborhood of $M_g$. Then $U \cap V$ deformation retracts onto $M_g$ and $R$ is a retract of both $U$ and $V$. We now have the Mayer-Vietoris sequence
\[
0 \to H_3(X) \to H_2(M_g) \to H_2(R) \oplus H_2(R) \to H_2(X) \to H_1(M_g) \to H_1(R) \oplus H_1(R) \to H_1(X) \to 0.
\]
We know $H_2(M_g) \approx \mathbb{Z}$, $H_2(R) \oplus H_2(R) = 0$, $H_1(M_g) \approx \mathbb{Z}^{2g}$ and $H_1(R) \oplus \mathbb{H}_1(R) \approx H_1 \left ( \bigvee_{i=1}^g S^1 \right ) \oplus H_1 \left ( \bigvee_{i=1}^g S^1 \right ) \approx \mathbb{Z}^{2g}$. This leaves the sequence
\[
0 \to H_3(X) \to \mathbb{Z} \to 0 \to H_2(X) \to \mathbb{Z}^{2g} \to \mathbb{Z}^{2g} \to H_1(X) \to 0.
\]
We immediately have $H_3(X) \approx \mathbb{Z}$. Note that the map $M_g \to R$ takes a generator in $M_g$ which goes ``through'' a hole in $M_g$, to $0$ in $R$ since it's nullhomotopic. Thus the kernel of the map $\Phi : \mathbb{Z}^{2g} \to \mathbb{Z}^{2g}$ is $\mathbb{Z}^g$ since half the generators are sent to $0$. Furthermore, since $H_2(R) \oplus H_2(R) = 0$ we see $H_2(X)$ maps injectively in $\mathbb{Z}^{2g}$ and we know it's image is the kernel of $\Phi$. Therefore $H_2(X) \approx \mathbb{Z}^g$. Additionally, the image of $\Phi$ is generated by the generators going around the ``holes'' of $M_g$, so $\im \Phi \approx \mathbb{Z}^g$. This is the kernel of the map going into $H_1(X)$ and since that map is surjective, we have $H_1(X) \approx \mathbb{Z}^{2g}/\mathbb{Z}^g \approx \mathbb{Z}^{g}$. We know $H_0(X) \approx \mathbb{Z}$ since it's path connected.

The relative homology sequence is
\[
0 \to H_3(R, M_g) \to H_2(M_g) \to H_2(R) \to H_2(R, M_g) \to H_1(M_g) \to H_1(R) \to H_1(R, M_G) \to 0.
\]
Filling these in for known values we get the sequence
\[
0 \to H_3(R, M_g) \to \mathbb{Z} \to 0 \to H_2(R, M_g) \to \mathbb{Z}^{2g} \to \mathbb{Z}^g \to H_1(R, M_g) \to 0.
\]
This immediately gives $H_3(R, M_g) \approx \mathbb{Z}$ again. As before the kernel of the map $\varphi : \mathbb{Z}^{2g} \to \mathbb{Z}^g$ is $\mathbb{Z}^g$. Since $H_2(R, M_g)$ injects into $\mathbb{Z}^{2g}$ and has image equal to $\ker \varphi$, we see $H_2(R, M_g) \approx \mathbb{Z}^g$. Furthermore, $\varphi$ is surjective so the kernel of the map $\mathbb{Z}^g \to H_1(R, M_g)$ is all of $\mathbb{Z}^g$ and since this map is surjective we must have $H_1(R, M_g) = 0$. Finally note that the sequence terminates in $H_0(M_g) \to H_0(R) \to H_0(R, M_g) \to 0$ and since the first map is an isomorphism we must have $H_0(R, M_g) = 0$.
\end{proof}

\begin{problem}
Suppose the space $X$ is the union of open sets $A_1, \dots , A_n$ such that each intersection $A_{i_1} \cap \dots \cap A_{i_k}$ is either empty or has trivial reduced homology groups. Show that $\widetilde{H}_i(X) = 0$ for $i \geq n-1$, and give an example showing this inequality is best possible, for each $n$.
\end{problem}
\begin{proof}
In the case $n = 1$ we know $X = A_1$ and $X$ has trivial homology groups. Suppose the statement is true for $n - 1$ and suppose $X = A_1 \cup \dots \cup A_n$. Let $A = A_1 \cup \dots \cup A_{n-1}$ and $B = A_n$. The we have a Mayer-Vietoris sequence
\[
H_{k+1}(X) \to H_k(A \cap B) \to H_k(A) \oplus H_k(B) \to H_k(X) \to H_{k-1}(A \cap B).
\]
We know $H_k(A \cap B) = 0$ and $H_k(B) = 0$ by assumption. By our inductive hypothesis, $H_k(A) = 0$ for $k \geq n-2$. In particular, for $k \geq n-1$ we have
\[
H_{k+1}(X) \to 0 \to 0 \to H_k(X) \to 0
\]
so $H_k(X) = 0$ for $n \geq n-1$. As a particular example take $X$ to be the boundary of an $n-1$-simplex and take the sets $A_i$ to be neighborhoods of the faces. Then we know $X$ has trivial homology groups for $k \geq n-1$ since $X$ can be viewed as a CW complex with dimension $n-2$. But $X$ doesn't have trivial homology for $k = n-2$ since the kernel of the boundary map $\partial_i$ is nontrivial for each $i \leq n-2$.
\end{proof}

\begin{problem}
Show that $H_i(X \times S^n) \approx H_i(X) \oplus H_{i-n}(X)$ for all $i$ and $n$, where $H_i = 0$ for $i < 0$ by definition. Namely, show $H_i(X \times S^n) \approx H_i(X) \oplus H_i(X \times S^n, X \times \{x_0\})$ and $H_i(X \times S^n, X \times \{x_0\}) \approx H_{i-1}(X \times S^{n-1}, X \times \{x_0\})$.
\end{problem}
\begin{proof}
We'll use the relative Mayer-Vietoris sequence
\[
\xymatrix{
H_n(A \cap B, C \cap D) \ar[r]^-{\Phi} & H_n(A, C) \oplus H_n(B,D) \ar[r]^-{\Psi} & H_n(X,Y).
}
\]
Let $D_1^n$ and $D_2^n$ be the upper and lower hemispheres of $S^n$ respectively. Then in the sequence above let $A = X \times D_1^n$, $B = X \times D_2^n$, $C = D = Y = X \times \{x_0\}$ and $X = A \cup B = X \times S^n$, so that $A \cap B = X \times S^{n-1}$. We then have
\[
\xymatrix{
H_k(X \times D_1^n, X \times \{x_0\}) \oplus H_k(X \times D_2^n, X \times \{x_0\}) \ar[rr]^-{\Psi} && H_k(X \times S^n, X \times \{x_0\})\\
&\ar[r]^-{\partial} & H_{k-1}(X \times S^{n-1}, X \times \{x_0\}).&
}
\]
Since $D_i^n$ is contractable we have $H_k(X \times D_i^n, X \times \{x_0\}) \approx H_k(X)/H_k(X) = 0$. Thus we have an isomorphism $H_k(X \times S^n, X \times \{x_0\}) \approx H_{k-1}(X \times S^{n-1}, X \times \{x_0\})$.

Now use the above sequence and let $A = X \times D^{n+1}$, $B = X \times S^n$, $C = \{x_0\} \times \{y_0\}$ and $D = Y = X \times \{y_0\}$ where $y_0 \in \partial D^{n+1}$ so that $A \cap B = B = X \times S^n$ and $C \cap D = \{x_0\} \times \{y_0\}$. Finally set $X = A \simeq X$. We now have the sequence
\[
\xymatrix{
H_k(X \times S^n, \{x_0\} \times \{y_0\}) \ar[r]^-{\Phi} \ar@{=}[d] & {\begin{matrix}H_k(X \times D^{n+1}, \{x_0\} \times \{y_0\})\\ \oplus H_k(X \times S^n, X \times \{x_0\})\end{matrix}} \ar[r]^-{\Psi} \ar@{=}[d] & H_k(X \times D^{n+1}, X \times \{y_0\}) \ar@{=}[d]\\
\widetilde{H}_k(X \times S^n) \ar[r] & \widetilde{H}_k (X) \oplus \widetilde{H}_k(X \times S^n, X \times \{x_0\}) \ar[r] & 0.
}
\]
This gives an isomorphism $H_k(X \times S^n) \approx H_k(X) \oplus H_k(X \times S^n, X \times \{x_0\})$. We now use these two isomorphisms to note that $H_k(X \times S^n) \approx H_k(X) \oplus H_k(X \times S^n, X \times \{x_0\}) \approx H_k(X) \oplus H_{k-1}(X \times S^{n-1}, X \times \{x_0\})$. We inductively continue this argument until we get $H_k(X \times S^n) \approx H_k(X) \oplus H_{k-n}(X \times S^0, X \times \{x_0\}) \approx H_k(X) \oplus H_{k-n}(X)$.
\end{proof}

\begin{problem}
(a) Show that a chain complex of free abelian groups $C_n$ splits as a direct sum of subcomplexes $0 \to L_{n+1} \to K_n \to 0$ with at most two nonzero terms.\\
(b) In the case the groups are finitely generated, show there is a further splitting into summands $0 \to \mathbb{Z} \to 0$ and $0 \to \mathbb{Z} \stackrel{m}{\to} \mathbb{Z} \to 0$.\\
(c) Deduce that if $X$ is a CW complex with finitely many cells in each dimension, then $H_n(X;G)$ is the direct sum of the following groups:
\begin{itemize}
\item a copy of $G$ for each $\mathbb{Z}$ summand of $H_n(X)$
\item a copy of $G/mG$ for each $\mathbb{Z}_m$ summand of $H_n(X)$
\item a copy of the kernel of $G \stackrel{m}{\to} G$ for each $\mathbb{Z}_m$ summand of $H_{n-1}(X)$.
\end{itemize}
\end{problem}
\begin{proof}
(a) Consider the exact sequence $0 \to \ker \partial_n \to C_n \to \im \partial_n$. We can find a map $\im \partial_n \to C_n$ by taking a representative from each fibre and the composition of this map with $\partial_n$ gives the identity on $\im \partial_n$. The sequence then splits so that $C_n \approx \ker \partial_n \oplus \im \partial_n$. Now we can let $K_n = \ker \partial_n$ and $L_n = \im \partial_n$ so we get the sequence $0 \to L_{n+1} \to K_n \to 0$.

(b) In the case $C_n$ is finitely generated we can represent the map $L_{n+1} \to K_n$ by an $k \times k$ matrix where $k$ is the number of generators. The $i^{\textup{th}}$ column represents where the $i^{\textup{th}}$ generator gets sent and to which multiplicities. We can reduce this matrix with elementary row and column operations and retain the same map. After doing this we'll be left with maps which send $1 \mapsto 1$ or $1 \mapsto m$. Thus we can break down the map $L_{n+1} \to K_n$ into maps $\mathbb{Z} \to \mathbb{Z}$ and $\mathbb{Z} \stackrel{m}{\to} \mathbb{Z}$.

(c) If $X$ is a CW complex with finitely many cells in each dimension then the corresponding chain complex consists of finitely generated abelian groups. Parts (a) and (b) show that this chain complex splits as sums of $\mathbb{Z}$, $\mathbb{Z}_m$ and kernels of maps $\mathbb{Z} \stackrel{m}{\to} \mathbb{Z}$. If we switch to homology with coefficients in $G$ then we can replace all the instances of $\mathbb{Z}$ with $G$ to get the described structure of the homology groups $H_n(X;G)$.
\end{proof}

\end{document}