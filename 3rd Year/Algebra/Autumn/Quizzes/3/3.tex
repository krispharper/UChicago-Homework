\documentclass{article}
\usepackage{amsmath,amsthm,amssymb,amsfonts,fullpage,fancyhdr}

\pagestyle{fancy}
\renewcommand{\headheight}{50pt}
\renewcommand{\footskip}{10pt}
\renewcommand{\textheight}{609pt}
\renewcommand{\headrulewidth}{0pt}

\newtheorem{problem}{Problem}

\newcommand{\normal}{\unlhd}

\begin{document}

\rhead{Kris Harper\\MATH 25700\\November 4, 2009\\}
\chead{Quiz 3\\}

\begin{problem}
If a subgroup $G$ of $S_n$ contains an odd permutation, then the order of $G$ is even and exactly half of the elements in $G$ are odd permutations.
\end{problem}
\begin{proof}
Let $\sigma \in G$ be an odd permutation. Then we know the cycle decomposition of $\sigma$ contains an odd number of cycles of even length. In particular, there is at least one cycle of even length. The order of $\sigma$ is the least common multiple of the lengths of all the cycles in its cycle decomposition. Therefore $|\sigma|$ is even which means $|\langle \sigma \rangle|$ is even. By Lagrange's Theorem, an even number divides $|G|$ and thus $|G|$ is even.

Now note that $\epsilon : G \to Z_2$ is a homomorphism. Furthermore, since $G$ is a subgroup, $(1) \in G$ and $\epsilon((1)) = 1$. Since $\epsilon(\sigma) = -1$ we know that $\epsilon(G) = Z_2$, that is, $\epsilon$ is surjective. Now note that $\ker \epsilon$ is the set of even permutations in $G$ and by the First Isomorphism Theorem, $G/\ker \epsilon \cong Z_2$. That is $|G : \ker \epsilon| = 2$ which means precisely half of $G$ is made up of even permutations. This leaves the remaining half to be odd permutations.
\end{proof}

\begin{problem}
If $p$ is prime and $G$ has order a power of $p$ ($p$ to the $a$, some $a$), and if $N$ is a nontrivial normal subgroup of $G$, show that $N$ intersects $Z(G)$ nontrivially.
\end{problem}
\begin{proof}
Let $G$ act on $N$ by conjugation. Since $N \normal G$, this satisfies the axioms for group actions. We can then write the class equation for $N$ as
\[
|N| = |N \cap Z(G)| + \sum_{i = 1}^{r} |G : C_G(g_i)|
\]
where $C_G(g_i)$ are proper subgroups of $G$. Now from Lagrange's Theorem $|N| = p^b$ for some $1 \leq b \leq a$ and additionally, $|C_G(g_i)| = p^{c_i}$ for some $1 \leq c_i < a$ (or such $C_G(g_i)$ don't exist for all $i$ in the case $a = 1$). This means $p \mid |N|$ and $p \mid \sum_{i=1}^{r} |G : C_G(g_i)|$ which forces $p \mid |N \cap Z(G)|$. Noting that $N \cap Z(G)$ is nonempty since $1 \in N \cap Z(G)$, we have $N \cap Z(G) \neq 1$.
\end{proof}

\end{document}