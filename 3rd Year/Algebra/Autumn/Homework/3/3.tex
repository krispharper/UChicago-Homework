\documentclass{article}
\usepackage{amsmath,amsthm,amssymb,amsfonts,fullpage,fancyhdr}

\input xy
\xyoption{all}

\pagestyle{fancy}
\renewcommand{\headheight}{50pt}
\renewcommand{\footskip}{40pt}
\renewcommand{\textheight}{590pt}
\renewcommand{\headrulewidth}{0pt}

\newtheorem{problem}{Problem}

\begin{document}

\rhead{Kris Harper\\MATH 25700\\October 16, 2009\\}
\chead{Homework 3\\}

\begin{problem}[2.1.3]
\label{d8subgroups}
Show that the following subsets of the dihedral group $D_8$ are actually subgroups:\\
(a) $\{1, r^2, s, sr^2\}$\\
(b) $\{1, r^2, sr, sr^3\}$.
\end{problem}
\begin{proof}
(a) Obviously $H = \{1, r^2, s, sr^2\}$ is nonempty. Since $H$ is finite, we need only check that $H$ is closed under multiplication. For $r^2$ we have $r^2 r^2 = r^4 = 1$, $r^2 s = sr^{-2} = sr^2 \in H$, and $r^2(sr^2) = (sr^{-2})r^2 = s \in H$. For $s$ we have $sr^2 \in H$, $s^2 = 1$ and $s(sr^2) = s^2r^2 = r^2 \in H$. Finally for $sr^2$ we have $sr^2 r^2 = sr^4 = s \in H$, $sr^2s = s^2s^{-2} = r^2 \in H$ and $sr^2sr^2 = s^2r^{-2}r^2 = 1$. Thus $H$ is a subgroup of $D_8$ since it's finite, nonempty and is closed under multiplication.

(b) By a similar argument as above, we only need to check that $H = \{1, r^2, sr, sr^3\}$ is closed under multiplication. For $r^2$ we have $r^2r^2 = 1$, $r^2sr = sr^{-2}r = sr^2r = sr^3 \in H$, and $r^2sr^3 = sr^{-2}r^3 = sr \in H$. For $sr$ we have $srr^2 = sr^3 \in H$, $srsr = ssr^{-1}r = 1 \in H$ and $srsr^3 = ssr^{-1}r^3 = ssr^2 = r^2 \in H$. Finally for $sr^3$ we have $sr^3r^2 = sr^5 = sr \in H$, $sr^3sr = ssr^{-3}r = ssr^2 = r^2 \in H$ and $sr^3sr^3 = ssr^{-3}r^{-3} = 1$. Thus $H$ is a subgroup of $D_8$ since it's finite, nonempty and is closed under multiplication.
\end{proof}

\begin{problem}[2.1.6]
Let $G$ be an abelian group. Prove that $\{g \in G \mid |g| < \infty\}$ is a subgroup of $G$ (called the \emph{torsion subgroup} of $G$). Give an explicit example where this set is not a subgroup when $G$ is non-abelian.
\end{problem}
\begin{proof}
Clearly $|1| = 1 < \infty$ so $H = \{g \in G \mid |g| < \infty\} \neq \emptyset$. Take $x, y \in H$ such that $|x| = n$ and $|y| = m$. Note that this implies $|y^{-1}| = m$ since $y^m = 1$ and we can multiply by $y^{-m} = (y^{-1})^m$ on both sides. Now since $G$ is abelian we have $(xy^{-1})^{nm} = x^{nm}(y^{-1})^{nm} = (x^n)^m ((y^{-1})^m)^n = 1$. Therefore $|xy^{-1}| \leq nm < \infty$. This shows that $H$ is a subgroup of $G$.

As an example, consider the group with presentation $\langle r, s \mid s^2 = 1, rs = sr^{-1} \rangle$. This is a nonabelian group with infinite order. Note that $|r| = \infty$, $|s| = 2$, $|sr| = 2$. But $(s)(sr) = s^2r = r$ so $|(s)(sr)| = \infty$.
\end{proof}

\begin{problem}[2.1.10]
(a) Prove that if $H$ and $K$ are subgroups of $G$ then so is their intersection $H \cap K$.\\
(b) Prove that the intersection of an arbitrary nonempty collection of subgroups of $G$ is again a subgroup of $G$ (do not assume the collection is countable).
\end{problem}
\begin{proof}
(a) Since $1 \in H$ and $1 \in K$ we know $H \cap K \neq \emptyset$. Now take $x,y \in H \cap K$. Note that this means $x,y \in H$ and $x, y \in K$. But since $H \leq G$ we know $xy^{-1} \in H$ and the same for $K$. Therefore $xy^{-1} \in H \cap K$ and we're done.

(b) Let $\mathcal{A}$ be a nonempty collection of subgroups of $G$ indexed by some set $I$. Note that $\bigcap_{i \in I} A_i \neq \emptyset$ since $1 \in A_i$ for all $i \in I$. Now consider $x, y \in \bigcap_{i \in I} A_i$. This means $x,y \in A_i$ for each $i \in I$. But since $A_i \leq G$ we have $xy^{-1} \in A_i$ for each $i \in I$. Then this means that $xy^{-1} \in \bigcap_{i \in I} A_i$. Therefore $\bigcap_{i \in I} A_i$ is a subgroup of $G$ by the subgroup criterion.
\end{proof}

\begin{problem}[2.1.15]
\label{unionofsubgroups}
Let $H_1 \leq H_2 \leq \dots$ be an ascending chain of subgroups of $G$. Prove that $\bigcup_{i=1}^{\infty} H_i$ is a subgroup of $G$.
\end{problem}
\begin{proof}
Clearly $\bigcup_{i=1}^{\infty} H_i \neq \emptyset$ since $1 \in H_1$. Let $x,y \in \bigcup_{i=1}^{\infty} H_i$. Then $x \in H_i$ and $y \in H_j$ for some $i,j$. Without loss of generality suppose $i \leq j$. Then we know $H_i \leq H_j$ and so $x,y \in H_j$. Since $H_j \leq G$, we know $xy^{-1} \in H_j$ and thus $xy^{-1} \in \bigcup_{i=1}^{\infty} H_i$. Therefore $\bigcup_{i=1}^{\infty} H_i \leq G$.
\end{proof}

\begin{problem}[2.1.16]
\label{uppertri}
Let $n \in \mathbb{Z}^+$ and let $F$ be a field. Prove that the set $\{(a_{ij}) \in GL_n(F) \mid a_{ij} = 0 \text{ for all } i > j\}$ is a subgroup of $GL_n(F)$ (called the group of upper triangular matrices).
\end{problem}
\begin{proof}
Let $H = \{(a_{ij}) \in GL_n(F) \mid a_{ij} = 0 \text{ for all } i > j\}$. The identity matrix $I \in H$ since $I_{ij} = 0$ for all $i \neq j$. Take $X,Y \in H$. Note that $Y^{-1} \in H$ because the inverse of an upper triangular matrix is an upper triangular matrix. This fact can be established with contradiction as multiplying an upper-triangular matrix by one which is not upper triangular will always result in an off-diagonal nonzero element. Furthermore, consider $(XY^{-1})_ij = \sum_{k = 1}^{n} X_{ik}Y_{kj}$. Note that $X_{ik} = 0$ whenever $i > k$, and $Y_{kj} = 0$ whenever $k > j$. Therefore, provided that $i > j$, this sum is always $0$. This then shows that $XY^{-1}$ is an upper triangular matrix. Therefore $XY^{-1} \in H$ and $H \leq GL_n(F)$.
\end{proof}

\begin{problem}[2.1.17]
Let $n \in \mathbb{Z}^+$ and let $F$ be a field. Prove that the set $\{(a_{ij}) \in GL_n(F) \mid a_{ij} = 0 \text{ for all } i > j \text{ and } a_ii = 1 \text{ for all } $i$\}$ is a subgroup of $GL_n(F)$
\end{problem}
\begin{proof}
As in Problem~\ref{uppertri} we see that $I \in H$. Let $X, Y \in H$. The fact that for $X,Y \in H$ we have $XY^{-1} \in H$ is the same proof as in Problem~\ref{uppertri}.
\end{proof}

\begin{problem}[2.2.6]
Let $H$ be a subgroup of the group $G$.\\
(a) Show that $H \leq N_G(H)$. Give an example to show that this is not necessarily true if $H$ is not a subgroup.
(b) Show that $H \leq C_G(H)$ if and only if $H$ is abelian.
\end{problem}
\begin{proof}
(a) Since $H \leq G$ we know that $H$ is closed under inverses and products and $H \neq \emptyset$. To show $H \leq N_G(H)$ we must show that $H \subseteq N_G(H)$. Let $x \in H$. Consider $xhx^{-1}$ for some $h \in H$. Since $H$ is closed under products and inverses, we know that $xhx^{-1} \in H$. Furthermore, for $h \in H$ we know $xhx^{-1} \in H$. We thus have $xHx^{-1} = H$, and therefore $x \in N_G(H)$. Hence, $H \subseteq N_G(H)$ and since it respects the group operation, $H \leq N_G(H)$.

As an example, take any set $H$ which doesn't contain the identity. Since $1 \cdot H \cdot 1^{-1} = H$ we know $1 \in N_G(H)$, but it's clear that $H \nleq N_G(H)$ since $1 \notin H$.

(b) Suppose $H \leq C_G(H)$. Then for $x,y \in H$ we know $xyx^{-1} = y$ which implies $xy = yx$. Conversely, suppose $H$ is abelian. Then for all $x,y \in H$ we have $xy = yx$ and so $xyx^{-1} = y$. But $C_G(H) = \{x \in G \mid xyx^{-1} = y \text{ for all } y \in H\}$. This shows that $H \subseteq C_G(H)$. Since $H \leq G$ we know $H \leq C_G(H)$.
\end{proof}

\begin{problem}[2.2.7]
\label{d2ncenter}
Let $n \in \mathbb{Z}$ with $n \geq 3$. Prove the following:\\
(a) $Z(D_{2n}) = 1$ if $n$ is odd.\\
(b) $Z(D_{2n}) = \{1, r^k\}$ if $n = 2k$.
\end{problem}
\begin{proof}
(a) We know that $s \notin Z(D_{2n})$ since $rs = sr^{-1}$. Take $r^k \in D_{2n}$ for $k \neq 0$. Note that $r^k \neq r^-k$, otherwise $r^{2k} = 1 = r^n$ and $n$ is even. Therefore $r^ks = sr^{-k}$ shows that $r^k$ doesn't commute with $s$. Thus the only element which commutes with all elements of $D_{2n}$ is $1$ and $Z(D_{2n}) = 1$.

(b) From Problem 1.2.4 we know that since $n = 2k$, $r^k$ is the only nonidentity element which commutes with every element of $D_{2n}$. Therefore $Z(D_{2n}) = \{1, r^k\}$.
\end{proof}

\begin{problem}[2.2.8]
Let $G = S_n$, fix an $i \in \{1, 2, \dots , n\}$ and let $G_i = \{\sigma \in G \mid \sigma(i) = i\}$ (the stabilizer of $i$ in $G$). Use group actions to prove that $G_i$ is a subgroup of $G$. Find $|G_i|$.
\end{problem}
\begin{proof}
Note that $\sigma$ is a group action acting on the set $\{1, \dots , n\}$ such that $\sigma . i = \sigma(i)$. It's easy to see that this is indeed a group action. Since the stabilizing set of a group action is always a subgroup of the acting group, we know that $G_i \leq G$. In any permutation of $\{1, 2, \dots , n\}$ there are $n$ possibilities for the location of $i$. Since $G_i$ is the set of permutations which fix $i$, we see that $|G_i| = |S_n|/n = (n-1)!$.
\end{proof}

\begin{problem}[2.2.9]
For any subgroup $H$ of $G$ and any nonempty subset $A$ of $G$ define $N_H(A)$ to be the set $\{h \in H \mid hAh^{-1} = A\}$. Show that $N_H(A) = N_G(A) \cap H$ and deduce that $N_H(A)$ is a subgroup of $H$ (note that $A$ need not be a subset of $H$).
\end{problem}
\begin{proof}
Let $h \in N_H(A)$. Then $hAh^{-1} = A$ and $h \in H$. But since $h \in G$ this means that $h \in N_G(A)$. Moreover, $h \in H$ as well which means $h \in N_G(A) \cap H$. For the other inclusion, suppose $h \in N_G(A) \cap H$. Then $hAh^{-1} = A$. But since $h \in H$ we know that $h \in N_H(A)$. Both inclusions show that $N_H(A) = N_G(A) \cap H$. Since $N_H(A) \subseteq N_G(A) \cap H$ it follows that $N_H(A) \subseteq H$. Because of this fact, we know that $N_H(A)$ is closed under inverses and products. Therefore $N_H(A) \leq H$.
\end{proof}

\begin{problem}[2.3.11]
Find all cyclic subgroups of $D_8$. Find a proper subgroup of $D_8$ which is not cyclic.
\end{problem}
\begin{proof}
We've shown that the groups $\langle r \rangle$ and $\langle s \rangle$ are cyclic subgroups of $D_{2n}$. Additionally, $\langle r^2 \rangle$ is a subgroup as per Problem~\ref{d2ncenter}. Consider the powers of $r$ multiplied by $s$. We have $(r^ks)(r^ks) = r^kr^{-k}s^2 = 1$. This covers all the possible cyclic groups so we are left with $\langle 1 \rangle$, $\langle r \rangle$, $\langle r^2 \rangle$, $\langle s \rangle$, $\langle rs \rangle$, $\langle r^2s \rangle$, $\langle r^3s \rangle$. Consider the subgroup $\{1, s, r^2, sr^2\}$. From Problem~\ref{d8subgroups} we know this is a subgroup.
\end{proof}

\begin{problem}[2.3.12]
Prove that the following groups are \emph{not} cyclic:\\
(a) $Z_2 \times Z_2$.\\
(b) $Z_2 \times \mathbb{Z}$.\\
(c) $\mathbb{Z} \times \mathbb{Z}$.
\end{problem}
\begin{proof}
(a) We can write this group as $\{1,a,b,c\}$ such that $a^2 = b^2 = c^2 = 1$ and $ab = c$, $ac = b$ and $bc = a$. To see the identification, take $1 = (0,0)$, $a = (1,0)$, $b = (0,1)$ and $c = (1,1)$. The necessary equalities hold. Note that each element has order $1$ or $2$, but $|Z_2 \times Z_2| = 4$ and therefore it is not cyclic.

(b) We can write $Z_2 \times \mathbb{Z} = \langle (0,1), (1,0) \rangle$. If we take some element $(a,b) \in Z_2 \times \mathbb{Z}$ then we can write this as $a(1,0) + b(0,1)$. This then implies that $Z_2 \times \mathbb{Z}$ is not cyclic.

(c) As in part (b) write $\mathbb{Z} \times \mathbb{Z} = \langle (0,1), (1,0) \rangle$. The same linear decomposition from (b) works here. Therefore $\mathbb{Z} \times \mathbb{Z}$ is not cyclic.
\end{proof}

\begin{problem}[2.3.16]
Assume $|x| = n$ and $|y| = m$. Suppose that $x$ and $y$ commute: $xy = yx$. Prove that $|xy|$ divides the least common multiple of $m$ and $n$. Need this be true if $x$ and $y$ do \emph{not} commute? Give an example of commuting elements $x$, $y$ such that the order of $xy$ is not equal to the least common multiple of $x$ and $y$.
\end{problem}
\begin{proof}
Let $l$ be the least common multiple of $n$ and $m$ such that $an = l$ and $bm = l$. Then $1 = (x^n)^a (y^m)^b = x^{an} y^{bm} = x^l y^l = (xy)^l$ since $x$ and $y$ commute. But we know that if $|xy| = k$ then $|(xy)^l| = k/(k,l)$. Since $(xy)^l = 1$ we have $k = (k,l)$ and in particular $k \mid l$.

In the case of $D_6$ we have $|r| = 3$ and $|s| = 2$. The least common multiple of $2$ and $3$ is $6$. But $(rs)(rs) = rr^{-1}s^2 = 1$, so $|rs| = 2$. Thus if $x$ and $y$ do not commute, the statement is false. In $D_6$ consider the commuting elements $r$ and $r^2$. We know $|r| = 6$ and $|r^2| = 3$ so the least common multiple is $6$. But $rr^2 = r^3$ and $|r^3| = 2$. Nevertheless, $2 \mid 6$.
\end{proof}

\begin{problem}[2.3.26]
Let $Z_n$ be a cyclic group of order $n$ and for each integer $a$ let
\[
\sigma_a : Z_n \to Z_n \text{ by } \sigma_a(x) = x^a \text{ for all } x \in Z_n.
\]
(a) Prove that $\sigma_a$ is an automorphism of $Z_n$ if and only if $a$ and $n$ are relatively prime.\\
(b) Prove that $\sigma_a = \sigma_b$ if and only if $a \equiv b \pmod{n}$.\\
(c) Prove that \emph{every} automorphism of $Z_n$ is equal to $\sigma_a$ for some integer $a$.\\
(d) Prove that $\sigma_a \circ \sigma_b = \sigma_{ab}$. Deduce that the map $\overline{a} \mapsto \sigma_a$ is an isomorphism of $(\mathbb{Z}/n\mathbb{Z})^{\times}$ onto the automorphism group of $Z_n$ (so $\textup{Aut}(Z_n)$ is an abelian group of order $\phi(n)$).
\end{problem}
\begin{proof}
(a) Note that if $|x| = n$ then $Z_n = \langle x^a \rangle$ if and only if $(a,n) = 1$. Suppose that $\sigma_a$ is an automorphism of $Z_n$. Then for each $y \in Z_n$ there exists $x \in Z_n$ such that $\sigma_a(x) = x^a = y$. There exists $y \in Z_n$ such that $|y| = n$ and so $Z_n = \langle y \rangle = \langle x^a \rangle$. But the above theorem states that $(a,n) = 1$. Conversely, suppose that $(a,n) = 1$. Then we know $Z_n = \langle x^a \rangle$. This shows that $\sigma_a$ is surjective. To show that it's injective, take $y^i,y^j \in Z_n$ with $y^i \neq y^j$. Then we have $\sigma_a(y^i) = (y^a)^i \neq (y^a)^j = \sigma_a(y^j)$. The fact that $\sigma_a$ is a homomorphism follows from $\sigma_a(xy) = (xy)^a = x^ay^a = \sigma_a(x)\sigma_a(y)$. Thus, $\sigma_a$ is an automorphism.

(b) Suppose $\sigma_a = \sigma_b$. Then for all $x \in Z_n$ we have $x^a = x^b$ and $x^{a-b} = 1$. For some $x$, $|x| = n$, so we have $n \mid (a-b)$ which means $a \equiv b \pmod{n}$. Conversely, suppose that $a \equiv b \pmod{n}$. Then $n \mid (a-b)$ so there exists $c$ such that $cn = a-b$. Then $1 = (x^n)^c = x^{cn} = x^{a-b}$. Therefore $x^a = x^b$ and so $\sigma_a = \sigma_b$.

(c) Let $\varphi : Z_n \to Z_n$ be an automorphism. Then since elements of $Z_n$ are of the form $x^k$ for $1 \leq k \leq n$ we know $\varphi(x) = x^k$ for some $k$. But then note that
\[
\varphi(x^j) = \varphi(x\cdot x\cdot \dots \cdot x) = \varphi(x)\varphi(x) \dots \varphi(x) = (\varphi(x))^j = (x^k)^j = (x^j)^k.
\]
Thus $\varphi = \sigma_k$.

(d) We have
\[
\sigma_a \circ \sigma_b(x) = \sigma_a(\sigma_b(x)) = \sigma_a(x^b) = (x^b)^a = x^{ab} = \sigma_{ab}(x).
\]
Let $\varphi : (\mathbb{Z}/n\mathbb{Z})^{\times} \to \text{Aut}(Z_n)$ be the map described. Part (b) shows injectivity of $\varphi$. Part (c) shows surjectivity. And the above calculation shows that the group structure is preserved. Thus, $\varphi$ is an isomorphism.
\end{proof}

\begin{problem}[2.4.14]
A group $H$ is called \emph{finitely generated} if there is a finite set $A$ such that $H = \langle A \rangle$.\\
(a) Prove that every finite group is finitely generated.\\
(b) Prove that $\mathbb{Z}$ is finitely generated.\\
(c) Prove that every finitely generated subgroup of the additive group $\mathbb{Q}$ is cyclic.
\end{problem}
\begin{proof}
(a) Let $H = \{1, a_1, a_2, \dots , a_n\}$ be a finite group. Then $H = \langle a_1, a_2, \dots , a_n \rangle$.

(b) Let $n \in \mathbb{Z}$. Then we can write $n = \pm 1 \cdot n$. Therefore $\mathbb{Z} = \langle 1 \rangle$.

(c) Let $H = \langle a_1/b_1, a_2/b_2, \dots , a_n/b_n \rangle$ be a finitely generated additive subgroup of $\mathbb{Q}$. Let $x = \prod_{i=1}^{n} b_i$. Now take $m_1/n_1 = \sum_{j=k}^{l} a_{i_j}/b_{i_j}$ and $m_2/n_2 = \sum_{j=k'}^{l'} a_{i_j}/b_{i_j}$. Note that $n_1 = \prod_{j=k}^{l}b_{i_j}$ and $n_2 = \prod_{j=k'}^{l'}b_{i_j}$. Now consider the sum $(m_1n_2 + m_2n_1)/(n_1n_2)$. Note that if any terms in the products $n_1$ and $n_2$ coincide, then we can undistribute them in the sum in the numerator and cancel them with the denominator. This shows that $n_1n_2 \mid x$. Letting $y = x/(n_1n_2)$ we have $1 \cdot (m_1/n_1 + m_2/n_2) = (y/y)(m_1/n_1 + m_2/n_2) = (y(m_1n_2 + m_2n_1))/(yn_1n_2) = (ym_1n_2 + ym_2n_1)/x$. We have thus shown that $H \leq \langle 1/x \rangle$ which proves that $H$ is cyclic.
\end{proof}

\begin{problem}[2.4.16]
A subgroup $M$ of a group $G$ is called a \emph{maximal subgroup} if $M \neq G$ and the only subgroups of $G$ which contain $M$ are $M$ and $G$.\\
(a) Prove that $H$ is a proper subgroup of the finite group $G$ then there is a maximal subgroup of $G$ containing $H$.\\
(b) Show that the subgroup of all rotations in a dihedral group is a maximal subgroup.\\
(c) Show that if $G = \langle x \rangle$ is a cyclic group of order $n \geq 1$ then a subgroup of $H$ is maximal if and only if $H = \langle x^p \rangle$ for some prime $p$ dividing $n$.
\end{problem}
\begin{proof}
(a) If $H$ is maximal then we're done. If $H$ is not maximal, then since $H \neq G$, there must be a subgroup $H_1$ of smallest order which contains $H$. If $H_1 = G$ then $H$ must have been maximal since $H$ and $G$ both contain $H$ and since $H_1$ is of smallest order, there are no other such subgroups. Otherwise, if $H_1$ is maximal then we're finished, and if not then there exists a subgroup $H_2$ of smallest order which contains $H_1$. Since $G$ is finite, this process must eventually stop so that $H_i = G$ for some $i$. Then $H_{i-1}$ is maximal since is a proper subgroup of $G$ for which the only subgroups which contain it are $H_{i-1}$ and $G$.

(b) Clearly $\langle r \rangle \neq D_{2n}$ since $s \notin \langle r \rangle$. Let $H \leq D_{2n}$ be a subgroup such that $\langle r \rangle \leq H$. Since all powers of $r$ are already in $H$, we must have $sr^{k} \in H$ or $r^{k}s \in H$. Note that if $sr^{k} \in H$ then $sr^{k}r^{-k} = s \in H$. A similar argument holds for $r^ks$. Therefore $s \in H$, and so $H = D_{2n}$. This shows that $\langle r \rangle$ is maximal in $D_{2n}$.

(c) Let $H$ be a maximal subgroup of $G$. Note that since $G$ is cyclic, there is a unique cyclic subgroup $\langle x^d \rangle$ of order $a$ for each $a \mid n$ where $ad = n$. Furthermore, $H$ is one of these subgroups. If $H = \langle x^a \rangle$ for $a = bc$ then $H = \langle x^{bc} \rangle = \langle (x^b)^c \rangle$. Therefore $H \leq \langle x^b \rangle$. But since $H$ is maximal, $H = \langle x^p \rangle$ where $p$ is not the product of two integers. Therefore $p$ is prime.

Conversely, suppose that $H = \langle x^p \rangle$ for some prime $p$ dividing $n$. Then let $K \leq G$ be a subgroup such that $H \leq K$. By the statement above, we know $K = \langle x^a \rangle$ where $a$ divides $n$. Since $H \leq K$ we know that $x^p = (x^a)^k = x^{ak}$ for some $k$. If $k \neq 1$ then $p$ is not prime, so $k = 1$. But then $p = a$ and $H = K$. Furthermore, we know $H$ has order $n/p$ and since $p \neq 1$ we see $H \neq G$. This shows that $H$ is maximal.
\end{proof}

\begin{problem}[2.4.17]
Let $G$ be a finitely generated group, say $G = \langle g_1, g_2, \dots , g_n \rangle$ and let $\mathcal{S}$ be the set of all proper subgroups of $G$. Then $\mathcal{S}$ is partially ordered by inclusion. Let $\mathcal{C}$ be a chain in $\mathcal{S}$.\\
(a) Prove that the union, $H$, of all the subgroups in $\mathcal{C}$ is a subgroup of $G$.\\
(b) Prove that $H$ is a \emph{proper} subgroup.\\
(c) Use Zorn's Lemma to show that $\mathcal{S}$ has a maximal element (which is, by definition, a maximal subgroup).
\end{problem}
\begin{proof}
(a) Since $\mathcal{S}$ is partially ordered by inclusion, this follows directly from Problem~\ref{unionofsubgroups}.

(b) Suppose that $H = G$. Then for each $i$, $g_i \in H$. But then each $g_i$ is in a proper subgroup of $G$ lying in $\mathcal{C}$. But since $\mathcal{C}$ is chain, each subgroup is contained in another in $\mathcal{C}$. Since there are only finitely many $g_i$, there must be one subgroup which contains all of them. But then this subgroup isn't proper. This is a contradiction and so $H$ must be a proper subgroup.

(c) Part (b) shows that $H \in \mathcal{S}$ and part (a) shows that $H$ is an upper bound for $\mathcal{C}$. Since $\mathcal{S}$ is nonempty ($\langle 1 \rangle \in \mathcal{S}$) and each chain has an upper bound, by Zorn's Lemma $\mathcal{S}$ must have a maximal element.
\end{proof}

\begin{problem}[2.5.7]
Find the center of $D_{16}$.
\end{problem}
\begin{proof}
From Problem~\ref{d2ncenter} we know that $Z(D_{16}) = \{1, r^4\}$.
\end{proof}

\begin{problem}[2.5.8]
In each of the following groups, find the normalizer of each subgroup:\\
(a) $S_3$.\\
(b) $Q_8$.
\end{problem}
\begin{proof}
(a) The subgroups of $S_3$ are $\langle (1 \; 2) \rangle$, $\langle (1 \; 3) \rangle$, $\langle (2 \; 2) \rangle$ and $\langle (1 \; 2 \; 3) \rangle$. We know $(1 \; 2 \; 3) \notin N_{S_3}(\langle (1 \; 2) \rangle)$ because $\langle (1 \; 2) \rangle = \{(1), (1 \; 2)\}$ and $(1 \; 2 \; 3)(1 \; 2)(1 \; 2 \; 3)^{-1} = (1 \; 2\; 3)^2(1 \; 2) \notin N_{S_3}(\langle (1 \; 2) \rangle)$. The same argument holds for $(1 \; 2 \; 3)^2$. Now consider $(1 \; 3)$. We see that $(1 \; 3)(1 \; 2)(1 \; 3)^{-1} = (2 \; 3) \notin \langle (1 \; 3) \rangle$. The same argument holds for $(2 \; 3)$. This shows that $N_{S_3}(\langle (1 \; 2) \rangle ) = \langle (1 \; 2) \rangle$. A similar statement holds for $\langle (1 \; 3) \rangle$ and $\langle (2 \; 3) \rangle$. Now note that $\langle (1 \; 2 \; 3) \rangle \leq N_{S_3}(\langle (1 \; 2 \; 3) \rangle) \leq S_3$. Since $S_3$ has order $6$ and $\langle (1 \; 2 \; 3) \rangle$ has order $3$, from Lagrange's Theorem we know that $N_{S_3}(\langle (1 \; 2 \; 3) \rangle)$ is either $\langle (1 \; 2 \; 3) \rangle$ or $S_3$. Noting that $(1 \; 2)(1 \; 2 \; 3)(1 \; 2) = (1 \; 2 \; 3)^{-1}$ and that a similar statement can be said for $(1 \; 2 \; 3)^2$, we see that $(1 \; 2) \in N_{S_3}(\langle (1 \; 2 \; 3) \rangle)$. Since $(1 \; 2) \notin \langle (1 \; 2 \; 3) \rangle$ we must have $N_{S_3}(\langle (1 \; 2 \; 3) \rangle ) = S_3$. A similar proof shows the same is true for $(1 \; 2 \; 3)^2$. This delineates all the normalizers for subgroups of $S_3$.

(b) From Problem~\ref{q8normalsubgroups} we know that every subgroup of $Q_8$ is normal. This is equivalent to saying $N_{Q_8}(H) = Q_8$ for $H \leq Q_8$.
\end{proof}

\begin{problem}[2.5.12]
The group $A = Z_2 \times Z_4 = \langle a,b \mid a^2 = b^4 = 1, ab = ba \rangle$ has order $8$ and has three subgroups of order $4$: $\langle a, b^2 \rangle \cong V_4$, $\langle b \rangle \cong Z_4$ and $\langle ab \rangle \cong Z_4$ and every proper subgroup is contained in one of these three. Draw the lattice of all subgroups of $A$, giving each subgroup in terms of at most two generators.
\end{problem}

\centerline{
\xymatrix{
& Z_2 \times Z_4 \ar@{-}[ld] \ar@{-}[d] \ar@{-}[rd] &\\
\langle b \rangle \ar@{-}[rd] & \langle a , b^2 \rangle \ar@{-}[ld] \ar@{-}[d] \ar@{-}[rd] & \langle ab \rangle \ar@{-}[ld]\\
\langle a \rangle \ar@{-}[rd] & \langle b^2 \rangle \ar@{-}[d] & \langle ab^2 \rangle \ar@{-}[ld]\\
& \langle 1 \rangle &
}
}

\begin{problem}[2.5.13]
The group $G = Z_2 \times Z_8 = \langle x,y \mid x^2 = y^8 = 1, xy = yx \rangle$ has order $16$ and has three subgroups of order $8$: $\langle x, y^2 \rangle \cong Z_2 \times Z_4$, $\langle y \rangle \cong Z_8$ and $\langle xy \rangle \cong Z_8$ and every proper subgroup is contained in one of these three. Draw the lattice of all subgroups of $G$, giving each subgroup in terms of at most two generators.
\end{problem}

\centerline{
\xymatrix{
& & Z_2 \times Z_8 \ar@{-}[ld] \ar@{-}[d] \ar@{-}[rd]&\\
& \langle y \rangle \ar@{-}[rd] & \langle x, y^2 \rangle \ar@{-}[ld] \ar@{-}[d] \ar@{-}[rd] & \langle xy \rangle \ar@{-}[ld]\\
& \langle x, y^4 \rangle \ar@{-}[ld] \ar@{-}[d] \ar@{-}[rd] & \langle y^2 \rangle \ar@{-}[d] & \langle xy^2 \rangle \ar@{-}[ld]\\
\langle xy^4 \rangle \ar@{-}[rrd] & \langle x \rangle \ar@{-}[rd] & \langle y^4 \rangle \ar@{-}[d] &\\
& & \langle 1 \rangle &\\
}
}

\begin{problem}[2.5.14]
Let $M$ be the group of order $16$ with the following presentation:
\[
\langle u, v \mid u^2 = v^8 = 1, uv = uv^5 \rangle
\]
(sometimes called the \emph{modular} group of order $16$). It has three subgroups of order $8$: $\langle u, v^2 \rangle$, $\langle v \rangle$ and $\langle uv \rangle$ and every proper subgroup is contained in one of these three. Prove that $\langle u, v^2 \rangle \cong Z_2 \times Z_4$, $\langle v \rangle \cong Z_8$ and $\langle uv \rangle \cong Z_8$. Show that the lattice of subgroups of $M$ is the same as the lattice of subgroups of $Z_2 \times Z_8$ but that these two groups are not isomorphic.
\end{problem}
\begin{proof}
Let $G = \langle u, v^2 \rangle$ and $H = Z_2 \times Z_4 = \langle a,b \mid a^2 = b^4 = 1, ab = ba \rangle$. Let $\varphi : H \to G$ be a function such that $\varphi(a) = u$ and $\varphi(b) = v^2$. Comparing the two sets quickly shows that $\varphi$ is a bijection (the two sets are identical with $v^2 = b$). Now take $\varphi(ab) = uv^2 = \varphi(a)\varphi(b)$. This shows that the two sets are isomorphic. It should be immediately obvious that $\langle v \rangle \cong Z_8$ since this \emph{is} the cyclic group of order $8$. To show $\langle uv \rangle \cong Z_8$ we need only show that $|uv| = 8$. But this is easily seen since $|v| = 8$ and $|u| \mid |v|$. Note that $M \ncong Z_2 \times Z_8$ because $M$ is not abelian. That is, if $\varphi : Z_2 \times Z_8 \to M$ is an isomorphism, then $\varphi(a) \varphi(b) = \varphi (ab) = \varphi(ba) = \phi(b)\phi(a)$. But this is not in general true for $M$. Therefore $M \ncong Z_2 \times Z_8$. The lattice for $M$ is

\centerline{
\xymatrix{
& & M \ar@{-}[ld] \ar@{-}[d] \ar@{-}[rd]&\\
& \langle v \rangle \ar@{-}[rd] & \langle u, v^2 \rangle \ar@{-}[ld] \ar@{-}[d] \ar@{-}[rd] & \langle uv \rangle \ar@{-}[ld]\\
& \langle u, v^4 \rangle \ar@{-}[ld] \ar@{-}[d] \ar@{-}[rd] & \langle v^2 \rangle \ar@{-}[d] & \langle uv^2 \rangle \ar@{-}[ld]\\
\langle uv^4 \rangle \ar@{-}[rrd] & \langle u \rangle \ar@{-}[rd] & \langle v^4 \rangle \ar@{-}[d] &\\
& & \langle 1 \rangle &\\
}
}
\end{proof}

\begin{problem}[3.1.3]
Let $A$ be an abelian group and let $B$ be a subgroup of $A$. Prove that $A/B$ is abelian. Give an example of a nonabelian group $G$ containing a proper normal subgroup $N$ such that $G/N$ is abelian.
\end{problem}
\begin{proof}
Let $xB, yB \in A/B$. Since $xy = yx$ we see that $xByB = (xy)B = (yx)B = yBxB$. Consider $G = Q_8/\langle i \rangle$. From Problem~\ref{q8normalsubgroups} we know that $G \cong \mathbb{Z}/2\mathbb{Z}$, which is abelian, while $\langle i \rangle \subsetneq Q_8$.
\end{proof}

\begin{problem}[3.1.22]
(a) Prove that if $H$ and $K$ are normal subgroups of a group $G$ then their intersection $H \cap K$ is also a normal subgroup of $G$.
\end{problem}
\begin{proof}
Let $g \in G$. Since $gHg^{-1} = H$ and $gKg^{-1} = K$, consider
\[
g(H \cap K)g^{-1} = \{gxg^{-1} \mid x \in H, x \in K\} = \{gxg^{-1} \mid x \in H\} \cap \{gxg^{-1} \mid x \in K\} = gHg^{-1} \cap gKg^{-1} = H \cap K.
\]
Since $H$ and $K$ are normal subgroups of $G$, we have $gHg^{-1} \subseteq H$.
\end{proof}

\begin{problem}[3.1.32]
\label{q8normalsubgroups}
Prove that every subgroup of $Q_8$ is normal. For each subgroup find the isomorphism type of its corresponding quotient.
\end{problem}
\begin{proof}
The subgroups of $Q_8$ are $\langle i \rangle$, $\langle j \rangle$, $\langle k \rangle$, $\langle -1 \rangle$, $\langle 1 \rangle$. Consider $\langle i \rangle = \{1, i, -1, -i\}$ and the coset $j\langle i \rangle = \{jx \mid x \in \langle i \rangle\}$. Note that $j(1) = (1)j$, $ji = -ij$, $j(-1) = (-1)j$ and $j(-i) = ij$. Therefore $j \langle i \rangle = \langle i \rangle j$. A similar argument holds for $\langle j \rangle$ and $\langle k \rangle$ since these groups have identical structures to $\langle i \rangle$. Also note that $i \langle -1 \rangle = \langle -1 \rangle$ since $i(1) = (1)i$ and $i(-1) = (-1)i$. A similar argument holds for $j \langle -1 \rangle$ and $k \langle -1 \rangle$. This shows that for every subgroup of $Q_8$, the left coset is also a right coset. Thus, every subgroup is normal.

For $Q_8/\langle i \rangle$ we have
\[
j\langle i \rangle = \{j, ji, -ji, -j\} = \{j, -k, k, -j\} = \{ki, -k, k, -ki\} = k\langle i \rangle.
\]
Also since every element $x \in \langle i \rangle$ appears as both $x$ and $-x$ we now know $\pm j \langle i \rangle = \pm k \langle i \rangle$. Likewise
\[
i \langle i \rangle = \{i, i^2, -i, -i^2\} = \{i, 1, -i, -1\} = 1\langle i \rangle
\]
and the same argument above shows that $\pm i \langle i \rangle = \pm 1 \langle i \rangle$. Thus, we can take $j\langle i \rangle$ and $\langle i \rangle$ to be the two elements of $Q_8/\langle i \rangle$. Note that $j\langle i \rangle \cdot j \langle i \rangle = -\langle i \rangle = \langle i \rangle$. Thus we have $Q_8/\langle i \rangle \cong \mathbb{Z}/2\mathbb{Z}$. This same argument holds for $Q_8/\langle j \rangle$ and $Q_8/\langle j \rangle$ as well.

For $Q_8/\langle -1 \rangle$ we have $\pm \langle -1 \rangle = \{1, -1\}$. Thus for $x \in \{i, j, k\}$ we also have $\pm x \langle -1 \rangle = \{x, -x\}$. Therefore $(\pm i\langle -1 \rangle)^2 = (\pm j\langle -1 \rangle)^2 = (\pm k\langle -1 \rangle)^2 = \pm \langle -1 \rangle$. Furthermore, $i\langle -1 \rangle \cdot j \langle -1 \rangle = -k \langle -1 \rangle = k \langle -1 \rangle$. Since similar statements can be said about $jk$ and $ki$, we see that $Q_8/\langle -1 \rangle \cong V_8$, the Klein-4 group.
\end{proof}

\begin{problem}[3.2.1]
Which of the following are permissible orders for subgroups of a group of order $120$: $1$, $2$, $5$, $7$, $9$, $15$, $60$, $240$? For each permissible order, give the corresponding index.
\end{problem}
\begin{proof}
By Lagrange's Theorem we know that permissible orders are those which divide $120$. Therefore, the permissible orders are $1$, $2$, $5$, $15$, and $60$ with indices $120$, $60$, $24$, $8$ and $2$ respectively.
\end{proof}

\begin{problem}[3.2.5]
Let $H$ be a subgroup of $G$ and fix some element $g \in G$.\\
(a) Prove that $g H g^{-1}$ is a subgroup of $G$ of the same order as $H$.\\
(b) Deduce that if $n \in \mathbb{Z}^+$ and $H$ is the unique subgroup of $G$ of order $n$, then $H \unlhd G$.
\end{problem}
\begin{proof}
(a) Since $1 \in H$ we know $1 = g \cdot 1 \cdot g^{-1} \in gHg^{-1}$ and so the set is nonempty. Take $x,y \in gHg^{-1}$. Then $x = gh_1g^{-1}$ and $y = gh_2g^{-1}$ for $h_1, h_2 \in H$. Also $y^{-1} = (g^{-1})^{-1}(gh_2)^{-1} = gh_2^{-1}g^{-1}$. Therefore $xy^{-1} = (gh_1g^{-1})(gh_2g^{-1}) = gh_1h_2^{-1}g^{-1}$. Since $H$ is a subgroup of $G$ we know $h_1h_2^{-1} \in H$ and thus $xy^{-1} \in gHg^{-1}$. This shows that $gHg^{-1} \leq G$. Now suppose that $x = y$ so that $gh_1g^{-1} = gh_2g^{-1}$. Then $h_1 = h_2$. Also, if $x \in gHg^{-1}$ then there's clearly $h_1 \in H$ for which $x = ghg^{-1}$. Thus the map $\phi : H \to gHg^{-1}$ where $h \mapsto ghg^{-1}$ is a bijection. Thus $|H| = |gHg^{-1}|$.

(b) If $H$ is the unique subgroup of order $n$ of $G$ then from part (a) we know that $gHg^{-1} = H$ for all $g \in G$. This shows that $H \unlhd G$.
\end{proof}

\begin{problem}[3.2.8]
Prove that if $H$ and $K$ are finite subgroups of $G$ whose orders are relatively prime then $H \cap K = 1$.
\end{problem}
\begin{proof}
Suppose to the contrary that $x \in H \cap K$ with $x \neq 1$. Then $\langle x \rangle \leq H$ and $\langle x \rangle \leq K$. But since $x \neq 1$ we know that $|\langle x \rangle| = k$ for some $k \neq 1$. Therefore each of $H$ and $K$ have a subgroup of order $k$. By Lagrange's Theorem we have $k \mid |H|$ and $k \mid |K|$ contradicting the fact that $(|H|, |K|) = 1$.
\end{proof}

\end{document}