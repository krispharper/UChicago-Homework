\documentclass{article}
\usepackage{amsmath,amsthm,amssymb,amsfonts,fullpage,fancyhdr}

\pagestyle{fancy}
\renewcommand{\headheight}{50pt}
\renewcommand{\footskip}{10pt}
\renewcommand{\textheight}{609pt}
\renewcommand{\headrulewidth}{0pt}

\newtheorem{problem}{Problem}

\begin{document}

\rhead{Kris Harper\\MATH 25700\\October 2, 2009\\}
\chead{Homework 1\\}

\begin{problem}
Determine whether the following functions $f$ are well-defined:\\
(a) $f : \mathbb{Q} \to \mathbb{Z}$ defined by $f(a/b) = a$.\\
(b) $f : \mathbb{Q} \to \mathbb{Q}$ defined by $f(a/b) = a^2/b^2$.
\end{problem}
\begin{proof}
(a) Here $f$ is not well defined. Note that $f(1/2) = 1$ and $f(2/4) = 2$ yet $1/2 = 2/4$.\newline

(b) Now $f$ is well defined. Let $a/b, c/d \in \mathbb{Q}$ such that $a/b = c/d$. We wish to show that $f(a/b) = f(c/d)$. Note that since $a/b = c/d$, squaring both sides gives $a^2/b^2 = c^2/d^2$ which is the desired equality.
\end{proof}

\begin{problem}
Let $f : A \to B$ be a surjective map of sets. Prove that the relation
\[
a \sim b \textup{ if and only if } f(a) = f(b)
\]
is an equivalence relation whose equivalence classes are the fibers of $f$.
\end{problem}
\begin{proof}
Clearly $\sim$ is reflexive since $f(a) = f(a)$ for all $a \in A$. Similarly if $a \sim b$ for $a,b \in A$ then $f(a) = f(b)$ and so $f(b) = f(a)$. Thus $b \sim a$ and $\sim$ is symmetric. Finally if $a \sim b$ and $b \sim c$ for $a,b,c \in A$, then $f(a) = f(b)$ and $f(b) = f(c)$. But then $f(a) = f(c)$ and so $a \sim c$. Thus $\sim$ is an equivalence relation.\newline

Consider $\overline{a}$, the equivalence class of $a$, and let $b \in \overline{a}$. Then $f(b) = f(a)$ and so $b \in f^{-1}(a)$. Thus $\overline{a} \subseteq f^{-1}(a)$. Conversely, let $b \in f^{-1}(a)$. Then $f(b) = f(a)$ and $b \sim a$. Thus $b \in \overline{a}$ and $f^{-1}(a) \subseteq \overline{a}$. Therefore the equivalence classes of $\sim$ are precisely the fibers of $f$.
\end{proof}

\begin{problem}
For each of the following pairs of integers $a$ and $b$, determine their greatest common divisor, their least common multiple, and write the greatest common divisor in the form $ax + by$ for some integers $x$ and $y$.\\
(a) $a = 20$, $b=13$.\\
(b) $a = 69$, $b=372$.
\end{problem}

(a) $(20, 13) = 1$. The least common multiple of $20$ and $13$ is $260$. $1 = (2)20 + (-3)13$.\newline

(b) $(69, 372) = 3$. The least common multiple of $69$ and $372$ is $8556$. $3 = (27)69 + (-5)372$.


\begin{problem}
Prove that if $n$ is composite, then there are integers $a$ and $b$ such that $n$ divides $ab$ but $n$ does not divide either $a$ or $b$.
\end{problem}
\begin{proof}
Let $n$ be composite. Then $n = p_1^{q_1}p_2^{q_2}\dots p_s^{q_s}$ for primes $p_1, \dots , p_s$ such that there exists $i$ and $j$ where $q_i \geq 1$ and $q_j \geq 1$ (if $i = j$ then $q_i > 1$). We can assume $i \leq j$. If $i < j$ let $a = p_1^{q_1}p_2^{q_2}\dots p_i^{q_i}$ and $b = p_{i+1}^{q_{i+1}}p_{i+2}^{q_{i+2}}\dots p_s^{q_s}$. Otherwise if $i =j$ let $a = p_1^{q_1}p_2^{q_2}\dots p_i^{q_i-1}$ and $b = p_{i}p_{i+1}^{q_{i+1}}p_{i+2}^{q_{i+2}}\dots p_s^{q_s}$. Note that since $a$ and $b$ are multiples of prime numbers, both are greater than $1$. Clearly $n \mid ab$ since $n$ will divide itself. But since $n = ab$ and $a > 1$, $b > 1$ and $n > 1$, it cannot be that $n \mid a$ or $n \mid b$.
\end{proof}

\begin{problem}
Determine the value $\varphi(n)$ for each integer $n \leq 30$ where $\phi$ denotes the Euler $\varphi$-function.
\end{problem}

$\varphi(1) = 1$, $\varphi(2) = 1$, $\varphi(3) = 2$, $\varphi(4) = 2$, $\varphi(5) = 4$, $\varphi(6) = 2$, $\varphi(7) = 6$, $\varphi(8) = 4$, $\varphi(9) = 6$, $\varphi(10) = 4$, $\varphi(11) = 10$, $\varphi(12) = 4$, $\varphi(13) = 12$, $\varphi(14) = 6$, $\varphi(15) = 8$, $\varphi(16) = 8$, $\varphi(17) = 16$, $\varphi(18) = 6$, $\varphi(19) = 18$, $\varphi(20) = 8$, $\varphi(21) = 12$, $\varphi(22) = 10$, $\varphi(23) = 22$, $\varphi(24) = 8$, $\varphi(25) = 20$, $\varphi(26) = 12$, $\varphi(27) = 18$, $\varphi(28) = 12$, $\varphi(29) = 28$, $\varphi(30) = 8$.


\begin{problem}
If $p$ is a prime prove there do not exist nonzero integers $a$ and $b$ such that $a^2 = pb^2$ (i.e. $\sqrt{p}$ is not a rational number).
\end{problem}
\begin{proof}
Let $p$ be prime and assume nonzero integers $a$ and $b$ exist such that $a^2 = pb^2$. Then $p \mid a^2$ or equivalently $p \mid a \cdot a$. Thus $p \mid a$ and so there exists $c \in \mathbb{Z}$ such that $pc = a$. Then $p^2c^2 = a^2 = pb^2$ and $pc^2 = b^2$. Consequently $p \mid b^2$ and so $p \mid b$ as well. Then $a_1 = a/p$ and $b_1 = b/p$ are both integers and we have $a_1^2 = pb_1^2$. But the same argument holds and so $p \mid a_1$ and $p \mid b_1$. We can let the integers $a_2 = a_1/p$ and $b_2 = b_1/p$ and continue in this fashion until $b_n^2 = 1$. This forces $a_n^2 = p$, but $p$ is prime and clearly not a perfect square. This is a contradiction and so $a$ and $b$ cannot exist.
\end{proof}

\begin{problem}
Let $f : A \to B$. The map $f$ is injective if and only if $f$ has a left inverse.
\end{problem}
\begin{proof}
Suppose that $f$ is injective. Let $g : B \to A$ be the function such that for $b \in f(A) \subseteq B$, $g(b) = a$ where $a$ is the unique element of $A$ such that $f(a) = b$. We know $a$ is unique because $f$ is injective and that such an $a$ exists because $b$ is in the image of $A$. Define $g(c)$ for $c \in B \backslash f(A)$ to be anything. Then for $a \in A$ we have $g \circ f(a) = g(f(a)) = a$.\newline

Conversely suppose that $f$ has a left inverse. Then there exists $g : B \to A$ such that $g \circ f : A \to A$ is the identity. Consider $x, y \in A$ such that $x \neq y$. Then $g \circ f(x) = g(f(x)) \neq g(f(y)) = g \circ f(y)$ which implies $f(x) \neq f(y)$ (otherwise $g(f(x))$ would equal $g(f(y))$). Thus $f$ is injective.
\end{proof}

\begin{problem}
If $A$ and $B$ are finite sets with the same number of elements then $f : A \to B$ is bijective if and only if $f$ is injective if and only if $f$ is surjective.
\end{problem}
\begin{proof}
Suppose $f$ is bijective, then it is clearly injective. Now suppose $f$ is injective. Let $|A| = |B| = n$. Since $f$ is injective, two distinct elements of $A$ are mapped by $f$ to two distinct elements of $B$. There are $n$ distinct elements of $A$ and so $|f(A)| = n$. But $|B| = n$ as well and so $f$ is surjective. Finally suppose that $f$ is surjective. Then for each $b \in B$ there exists $a \in A$ such that $f(a) = b$. Since $|B| = n$ there must be at least $n$ distinct elements of $A$ which map to unique values of $B$. But $|A| = n$, therefore if $x \neq y$ in $A$, then $f(x) \neq f(y)$ in $B$. Thus $f$ is both surjective and injective and so $f$ is a bijection.
\end{proof}

\begin{problem}
Write out the multiplication table for $D_6$.
\end{problem}
\begin{tabular}{|c|c|c|c|c|c|c|}
\hline
$\mathbf{\times}$ & $\mathbf{I}$ & $\mathbf{R_{120}}$ & $\mathbf{R_{240}}$ & $\mathbf{F_T}$ & $\mathbf{F_L}$ & $\mathbf{F_R}$\\
\hline
$\mathbf{I}$ & $I$ & $R_{120}$ & $R_{240}$ & $F_T$ & $F_L$ & $F_R$\\
\hline
$\mathbf{R_{120}}$ & $R_{120}$ & $R_{240}$ & $I$ & $F_R$ & $F_T$ & $F_L$\\
\hline
$\mathbf{R_{240}}$ & $R_{240}$ & $I$ & $R_{120}$ & $F_L$ & $F_R$ & $F_T$\\
\hline
$\mathbf{F_T}$ & $F_T$ & $F_L$ & $F_R$ & $I$ & $R_{120}$ & $R_{240}$\\
\hline
$\mathbf{F_L}$ & $F_L$ & $F_R$ & $F_T$ & $R_{240}$ & $I$ & $R_{120}$\\
\hline
$\mathbf{F_R}$ & $F_R$ & $F_T$ & $F_L$ & $R_{120}$ & $R_{240}$ & $I$\\
\hline
\end{tabular}

\begin{problem}
Write out the multiplication table for $D_8$.
\end{problem}
\begin{tabular}{|c|c|c|c|c|c|c|c|c|}
\hline
$\mathbf{\times}$ & $\mathbf{I}$ & $\mathbf{R_{90}}$ & $\mathbf{R_{180}}$ & $\mathbf{R_{270}}$ & $\mathbf{V}$ & $\mathbf{H}$ & $\mathbf{D_L}$ & $\mathbf{D_R}$\\
\hline
$\mathbf{I}$ & $I$ & $R_{90}$ & $R_{180}$ & $R_{270}$ & $V$ & $H$ & $D_L$ & $D_R$\\
\hline
$\mathbf{R_{90}}$ & $R_{90}$ & $R_{180}$ & $R_{270}$ & $I$ & $D_L$ & $D_R$ & $H$ & $V$\\
\hline
$\mathbf{R_{180}}$ & $R_{180}$ & $R_{270}$ & $I$ & $R_{90}$ & $H$ & $V$ & $D_R$ & $D_L$\\
\hline
$\mathbf{R_{270}}$ & $R_{270}$ & $I$ & $R_{90}$ & $R_{180}$ & $D_R$ & $D_L$ & $V$ & $H$\\
\hline
$\mathbf{V}$ & $V$ & $D_R$ & $H$ & $D_L$ & $I$ & $R_{180}$ & $R_{270}$ & $R_{90}$\\
\hline
$\mathbf{H}$ & $H$ & $D_L$ & $V$ & $D_R$ & $R_{180}$ & $I$ & $R_{90}$ & $R_{270}$\\
\hline
$\mathbf{D_L}$ & $D_L$ & $V$ & $D_R$ & $H$ & $R_{90}$ & $R_{270}$ & $I$ & $R_{180}$\\
\hline
$\mathbf{D_R}$ & $D_R$ & $H$ & $D_L$ & $V$ & $R_{270}$ & $R_{90}$ & $R_{180}$ & $I$\\
\hline
\end{tabular}

\end{document}