\documentclass{article}
\usepackage{amsmath,amsthm,amssymb,amsfonts,fullpage,fancyhdr}

\pagestyle{fancy}
\renewcommand{\headheight}{50pt}
\renewcommand{\footskip}{10pt}
\renewcommand{\textheight}{609pt}
\renewcommand{\headrulewidth}{0pt}

\newtheorem{problem}{Problem}

\begin{document}

\rhead{Kris Harper\\MATH 25900\\April 16, 2010\\}
\chead{Homework 2\\}

\begin{problem}[13.1.3]
Show that $x^3 + x + 1$ is irreducible over $\mathbb{F}_2$ and let $\theta$ be a root. Compute the powers of $\theta$ in $\mathbb{F}_2(\theta)$.
\end{problem}
\begin{proof}
Note that $0 + 0 + 1 = 1 = 1 + 1 + 1$ so neither $0$ or $1$ is a root of $x^3 + x + 1$. Thus this polynomial is irreducible. We know that $\mathbb{F}_2(\theta) \cong \mathbb{F}_2[x]/(x^3 + x + 1)$. Thus we have
\[
\theta^3 = - \theta - 1 = \theta + 1.
\]
Then
\[
\theta^4 = \theta^2 + \theta,
\]
\[
\theta^5 = \theta^3 + \theta^2 = \theta^2 + \theta + 1,
\]
\[
\theta^6 = \theta^3 + \theta^2 + \theta = \theta^2 + 1
\]
\[
\theta^7 = \theta^3 + \theta = 1.
\]
Taken together with $0$, $1$, $\theta$ and $\theta^2$ we see that these powers of $\theta$ form all $8$ elements of $\mathbb{F}_2(\theta)$. Moreover, $(\mathbb{F}_2(\theta))^{\times} = \langle \theta \rangle$.
\end{proof}

\begin{problem}[13.1.5]
Suppose $\alpha$ is a rational root of a monic polynomial in $\mathbb{Z}[x]$. Prove that $\alpha$ is an integer.
\end{problem}
\begin{proof}
Let $\alpha = a/b$ and take $(a,b) = 1$ with $b > 0$. There exists a polynomial $p(x) \in \mathbb{Z}[x]$ such that $p(a/b) = (a/b)^m + c_1(a/b)^{m-1} + \dots + c_m = 0$. Multiply this equation by $b^m$ so we have $a^m + c_1ba^{m-1} + \dots + b^mc = 0$. Since $b$ divides each term following $a^m$ and it divides the right hand side we see that $b \mid a^m$. Since $(a,b) = 1$ we must have $b = 1$ so that $a/b \in \mathbb{Z}$.
\end{proof}

\begin{problem}[13.1.8]
Prove that $x^5 - ax - 1 \in \mathbb{Z}[x]$ is irreducible unless $a = 0$, $2$ or $-1$. The first two correspond to linear factors, the third corresponds to the factorization $(x^2 - x + 1)(x^3 + x^2 - 1)$.
\end{problem}
\begin{proof}
From the rational root theorem we know that roots of this polynomial which lie in $\mathbb{Q}$ can only be $\pm 1$. Putting these in gives $-a = 0$ and $-2 + a = 0$ so if $a = 0$ or $a = 2$ then we have a linear factorization. If $a = -1$ then we have the above factorization so these three cases do indeed imply $x^5 - ax - 1$ is reducible. By the rational root theorem we've exhausted all the possibilities of $x^5 - ax - 1$ having a linear factor. Thus it can only factor into two polynomials of degree $2$ and $3$.

Suppose $x^5 - ax - 1 = (x^2 + bx + c)(x^3 + dx^2 + ex + f)$. Multiplying this out gives us the equations $b + d = 0$, $c + bd + e = 0$, $cd + be + f = 0$, $ce + bf = -a$ and $cf = -1$. Thus $b = -d$ and $c = \pm 1$. If $c = 1$ then $f = -1$ and we have $1 - b^2 + e = 0$, $-b + be - 1 = 0$ and $e-b = -a$. The second of these equations gives us $b(e-1) = 1$ so $e = 0$ or $e = 2$. If $e = 2$ then $b = 1$ and the first equation gives $0 = 1 - 1 + 2 = 2$. If $e = 0$ then $b = -1$ and by the third equation $a = -1$ as we had earlier.

Now we consider $c = -1$. Then $f = 1$ and we now have $-1 - b^2 + e = 0$, $b + be + 1 = 0$ and $b-e = -a$. Thus $b(e+1) = -1$ so $b = \pm 1$. By the first equation $0 = -1 - 1 + e$ so $e = 2$. But then by the second equation again we have $b = -1/3$ which is not an integer. So $c \neq -1$ and we see that $x^5 - ax - 1$ can only be factored into a quadratic and a cubic if $a = -1$.
\end{proof}

\begin{problem}[13.2.3]
Determine the minimal polynomial over $\mathbb{Q}$ for $1 + i$.
\end{problem}
\begin{proof}
Note that $(1+i)^2 - 2(1 + i) + 2 = 2i - 2 - 2i + 2 = 0$ so $1 + i$ is a root of $x^2 - 2x + 2$. But this polynomial is irreducible over $\mathbb{Q}$ using Eisenstein. Thus it must be the minimal polynomial for $1 + i$.
\end{proof}

\begin{problem}[13.2.4]
Determine the degree over $\mathbb{Q}$ of $2 + \sqrt{3}$ and of $1 + \sqrt[3]{2} + \sqrt[3]{4}$.
\end{problem}
\begin{proof}
Note that $(2 + \sqrt{3})^2 - 4(2 + \sqrt{3}) + 1 = 0$ so $2 + \sqrt{3}$ is a root for $x^2 - 4x + 1$. Moreover this polynomial is irreducible over $\mathbb{Q}$ by the rational root theorem so this must be the minimal polynomial for $2 + \sqrt{3}$. Therefore $\deg (2 + \sqrt{3}) = 2$.

Note that $(1 + \sqrt[3]{2} + \sqrt[3]{4})^3 - 3(1 + \sqrt[3]{2} + \sqrt[3]{4})^2 - 3(1 + \sqrt[3]{2} + \sqrt[3]{4}) - 1 = 0$ so $1 + \sqrt[3]{2} + \sqrt[3]{4}$ is a root of $x^3 - 3x^2 - 3x - 1$. Furthermore by the rational root theorem this is irreducible in $\mathbb{Q}$ so this must be the minimal polynomial. Thus $\deg (1 + \sqrt[3]{2} + \sqrt[3]{4}) = 3$.
\end{proof}

\begin{problem}[13.2.5]
Let $F = \mathbb{Q}(i)$. Prove that $x^3 - 2$ and $x^3 - 3$ are irreducible over $F$.
\end{problem}
\begin{proof}
There are no rational roots to either of these polynomials by the rational root theorem. Suppose we have a root of the form $a+ib$ where $a,b \in \mathbb{Q}$. Then $(a + ib)^3 - 2 = (a^3-3ab^2-2) + i(3a^2b-b^3) = 0$. This gives the two equations $a^3 - 3ab^2 - 2 = 0$ and $3a^2b - b^3 = 0$. Clearly $a \neq 0$ otherwise we get $-2 = 0$. If $b = 0$ then $a^3 - 2 = 0$ which we know has no solutions in $\mathbb{Q}$. The rational root theorem tells us that solutions to $a^3 - 3ab^2 - 2 = 0$ are either $a = \pm 2$ or $a = \pm 1/2$. If $a = 2$ then $b^2 = 1$ so $b = \pm 1$. In either case the second equation tells us that $a^2 = 1/3$ which has no solutions in $\mathbb{Q}$. Thus $a \neq 2$. If $a = -2$ then $b^2 = 5/3$ which has no solutions in $\mathbb{Q}$. If $a = 1/2$ then $b^2 = -5/4$ which has no solutions in $\mathbb{Q}$. If $a = -1/2$ then $b^2 = 17/12$ which has no solutions in $\mathbb{Q}$. All possibilities have been exhausted so there are no possible roots to $x^3 - 2$ of the form $a + ib$.

For the second polynomial we have similar equations $a^3 - 3ab^2 - 3 = 0$ and $3a^2b - b^3 = 0$. Now $a = \pm 3$ or $\pm 1/3$. If $a = 3$ then $b^2 = 8/3$. If $a = -3$ then $b^2 = 10/3$. If $a = 1/3$ then $b^2 = -80/27$ and if $a = -1/3$ then $b^2 = 82/27$. None of these have solutions in $\mathbb{Q}$ so there are no roots to $x^3 - 3$ of the form $a + ib$. Since each of these is cubic we see that they must be irreducible over $F$.
\end{proof}

\begin{problem}[13.2.7]
Prove that $\mathbb{Q}(\sqrt{2} + \sqrt{3}) = \mathbb{Q}(\sqrt{2},\sqrt{3})$. Conclude that $[\mathbb{Q}(\sqrt{2} + \sqrt{3}) : \mathbb{Q}] = 4$. Find an irreducible polynomial satisfied by $\sqrt{2} + \sqrt{3}$.
\end{problem}
\begin{proof}
The field $\mathbb{Q}(\sqrt{2},\sqrt{3})$ clearly contains $\mathbb{Q}$ and the element $\sqrt{2} + \sqrt{3}$ so we must have containment $\mathbb{Q}(\sqrt{2} + \sqrt{3}) \subseteq \mathbb{Q}(\sqrt{2}, \sqrt{3})$. Now note that
\[
\frac{ \left ( \sqrt{2} + \sqrt{3} \right )^3 - 9 \left ( \sqrt{2} + \sqrt{3} \right ) }{2} = \sqrt{2}
\]
and
\[
\frac{ \left ( \sqrt{2} + \sqrt{3} \right )^3 - 11 \left ( \sqrt{2} + \sqrt{3} \right ) }{-2} = \sqrt{3}.
\]
Thus $\sqrt{2}$ and $\sqrt{3}$ are both contained in $\mathbb{Q}(\sqrt{2} + \sqrt{3})$ so we must have the second inclusion as well. Since $[\mathbb{Q}(\sqrt{2}, \sqrt{3}) : \mathbb{Q}] = 4$ we must also have $[\mathbb{Q}(\sqrt{2} + \sqrt{3}) : \mathbb{Q}] = 4$. Note that $(\sqrt{2} + \sqrt{3})^4 - 10(\sqrt{2}+\sqrt{3})^2 + 1 = 0$ so $\sqrt{2} + \sqrt{3}$ satisfies $x^4 - 10x + 1$. The rational root theorem tells us that this is irreducible.
\end{proof}

\begin{problem}[13.2.10]
Determine the degree of the extension $\mathbb{Q}(\sqrt{3 + 2\sqrt{2}})$ over $\mathbb{Q}$.
\end{problem}
\begin{proof}
If we write $3 + 2 \sqrt{2} = 1 + 2 \sqrt{2} + 2 = (1 + \sqrt{2})^2$ we see that $\sqrt{3 + 2\sqrt{2}} = 1 + \sqrt{2}$. Using this it's easy to see that $(1+\sqrt{2})^2 - 2 (1 + \sqrt{2}) - 1 = 0$ so $\sqrt{3 + 2 \sqrt{2}}$ is a root for $x^2 - 2x - 1$. By the rational root theorem we know this is irreducible and thus $[\mathbb{Q}(\sqrt{3 - 2\sqrt{2}}) : \mathbb{Q}] = 2$.
\end{proof}

\begin{problem}[13.2.11]
(a) Let $\sqrt{3 + 4i}$ denote the square root of the complex number $3 + 4i$ that lies in the first quadrant and let $\sqrt{3 - 4i}$ denote the square root of $3 - 4i$ that lies in the fourth quadrant. Prove that $[\mathbb{Q}(\sqrt{3 + 4i} + \sqrt{3-4i}) : \mathbb{Q}] = 1$.\\
(b) Determine the degree of the extension $\mathbb{Q}(\sqrt{1 + \sqrt{-3}} + \sqrt{1 - \sqrt{-3}})$ over $\mathbb{Q}$.
\end{problem}
\begin{proof}
(a) Write $3 + 4i = 4 + 4i - 1 = (2+i)^2$ so that $\sqrt{3+4i} = 2+i$. Likewise $\sqrt{3-4i} = 2-i$. Summing these we get $4$ so $\sqrt{3 + 4i} + \sqrt{3 - 4i} - 4 = 0$ and this is a root of $x - 4$. Thus $[\mathbb{Q}(\sqrt{3 + 4i} + \sqrt{3-4i}) : \mathbb{Q}] = 1$.

(b) We see that
\[
\left (\sqrt{1 + \sqrt{-3}} + \sqrt{1 - \sqrt{-3}} \right )^4 - 4 \left (\sqrt{1 + \sqrt{-3}} + \sqrt{1 - \sqrt{-3}} \right )^2 - 36 = 0
\]
so that $\sqrt{1 + \sqrt{-3}} + \sqrt{1 - \sqrt{-3}}$ is a root of $x^4 - 4x^2 - 36$. Eisenstein will tell us that this is irreducible over $\mathbb{Q}$ so $[\mathbb{Q}(\sqrt{1 + \sqrt{-3}} + \sqrt{1 - \sqrt{-3}}) : \mathbb{Q}] = 4$.
\end{proof}

\begin{problem}[13.2.13]
Suppose $F = \mathbb{Q}(\alpha_1, \alpha_2, \dots , \alpha_n)$ where $\alpha_i^2 \in \mathbb{Q}$ for $i = 1, 2, \dots , n$. Prove that $\sqrt[3]{2} \notin F$.
\end{problem}
\begin{proof}
Since $\alpha_i^2 \in \mathbb{Q}$ for $i = 1, 2, \dots , n$ we see that $\deg \alpha_i \leq 2$. Furthermore, since degree extensions are multiplicative we have that $[F : \mathbb{Q}] = 2^k$ for some $k \leq n$. But $[\mathbb{Q}(\sqrt[3]{2}) : \mathbb{Q}] = 3$ and $3 \nmid 2^k$ so $\sqrt[3]{2} \notin F$.
\end{proof}

\begin{problem}[13.2.21]
Let $K = \mathbb{Q}(\sqrt{D})$ for some squarefree integer $D$. Let $\alpha = a + b \sqrt{D}$ be an element of $K$ use the basis $1, \sqrt{D}$ for $K$ as a vector space over $\mathbb{Q}$ and show that the matrix of the linear transformation ``multiplication by $\alpha$'' on $K$ considered in the previous exercises has the matrix $\left ( \begin{array}{cc} a & bD\\ b & a \end{array} \right )$. Prove directly that the map $a + b \sqrt{D} \mapsto \left ( \begin{array}{cc} a & bD\\ b & a \end{array} \right )$ is an isomorphism of the field $K$ with a subfield of the ring of $2 \times 2$ matrices with coefficients in $\mathbb{Q}$.
\end{problem}
\begin{proof}
Let $\beta = c + d \sqrt{D} \in K$. Then $\alpha \beta = (ac + bdD) + (ad + bc) \sqrt{D}$. But also
\[
\left (
\begin{array}{cc}
a & bD\\
b & a
\end{array}
\right )
\left (
\begin{array}{c}
c\\
d
\end{array}
\right )
=
\left (
\begin{array}{c}
ac + bdD\\
bc + ad
\end{array}
\right)
\]
so this matrix is precisely the transformation ``multiplication by $\alpha$''. Now note that
\[
\alpha \beta = (ac + bdD) + (ad + bc) \sqrt{D} \mapsto \left (
\begin{array}{cc}
ac + bdD & (ad + bc) D\\
ad + bc & ac + bdD
\end{array}
\right )
=
\left (
\begin{array}{cc}
a & bD\\
b & a
\end{array}
\right )
\left (
\begin{array}{cc}
c & dD\\
d & c
\end{array}
\right )
\]
so this is a homomorphism. It's clearly injective because $K$ is a field and this is not the $0$ map ($1$ is sent to the identity matrix). Thus $K$ is isomorphic to its image under this map and so it's image is a subfield of $2 \times 2$ matrices with coefficients in $\mathbb{Q}$.
\end{proof}

\end{document}