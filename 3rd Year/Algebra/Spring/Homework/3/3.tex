\documentclass{article}
\usepackage{amsmath,amsthm,amssymb,amsfonts,fullpage,fancyhdr}

\pagestyle{fancy}
\renewcommand{\headheight}{50pt}
\renewcommand{\footskip}{10pt}
\renewcommand{\textheight}{609pt}
\renewcommand{\headrulewidth}{0pt}

\newcommand{\aut}{\textup{Aut}}

\newtheorem{problem}{Problem}

\begin{document}

\rhead{Kris Harper\\MATH 25900\\April 23, 2010\\}
\chead{Homework 3\\}

\begin{problem}[13.4.1]
\label{four}
Determine the splitting field and its degree over $\mathbb{Q}$ for $x^4 - 2$.
\end{problem}
\begin{proof}
If $\alpha$ is a root of this polynomial then $(\zeta \alpha)^{4} = 2$ where $\zeta$ is fourth root of unity so the four solutions are $\zeta^n \sqrt[4]{2}$ where $1 \leq n \leq 4$. Note that the splitting field for this polynomial must then contain $\mathbb{Q}(\zeta, \sqrt[4]{2})$ and this extension also contains all the roots, so the splitting field must be $\mathbb{Q}(\zeta, \sqrt[4]{2})$. This field contains the extension $\mathbb{Q}(\zeta)$ and is generated over it by $\sqrt[4]{2}$ so we must have $[\mathbb{Q}(\zeta, \sqrt[4]{2}), \mathbb{Q}] = 4\phi(4) = 8$.
\end{proof}

\begin{problem}[13.4.2]
Determine the splitting field and its degree over $\mathbb{Q}$ for $x^4 + 2$.
\end{problem}
\begin{proof}
This follows the exact same argument as Problem~\ref{four} but with $\sqrt[4]{-2}$ in place of $\sqrt[4]{2}$. We then have the splitting field $\mathbb{Q}(\zeta, \sqrt[4]{-2})$ with extension degree $8$.
\end{proof}

\begin{problem}[13.4.6]
Let $K_1$ and $K_2$ be finite extensions of $F$ contained in the field $K$, and assume both are splitting fields over $F$.\\
(a) Prove that their composite $K_1K_2$ is a splitting field over $F$.\\
(b) Prove that $K_1 \cap K_2$ is a splitting field over $F$.
\end{problem}
\begin{proof}
(a) Let $f(x)$ and $g(x)$ be the polynomials for which $K_1$ and $K_2$ are respectively splitting fields. Note that $f(x)g(x)$ splits completely in $K_1K_2$ and this is the smallest field extension containing both $K_1$ and $K_2$. Since $K_1$ and $K_2$ are the smallest fields for which $f(x)$ and $g(x)$ split completely it follows that $K_1K_2$ must be the splitting field for $f(x)g(x)$.

(b) Let $f(x)$ be an irreducible polynomial in $F[x]$ with a root in $K_1 \cap K_2$. Then $f(x)$ has a root in $K_1$ and a root in $K_2$ so it splits completely in these two fields since they are splitting fields. But then the factors $f(x)$ splits into are contained in both $K_1$ and $K_2$ so $f(x)$ splits completely in $K_1 \cap K_2$ as well. Thus $K_1 \cap K_2$ is a splitting field for $F$.
\end{proof}

\begin{problem}[13.5.2]
Find all irreducible polynomials of degrees $1$, $2$ and $4$ over $\mathbb{F}_2$ and prove that their product is $x^{16}-x$.
\end{problem}
\begin{proof}
Degrees $1$ and $2$ are easily taken care of. The polynomials $x$ and $x+1$ are irreducible of degree $1$ and the only such. The polynomial $x^2 + x + 1$ is the only irreducible polynomial of degree $2$ as $x^2 + x = x(x+1)$, $x^2 = xx$ and $x^2 + 1 = (x+1)^2$.

There are $16$ polynomials of degree $4$ in $\mathbb{F}_2[x]$. If the constant term is $0$ then we can factor $x$ out so this reduces the number to $8$. We now have
\begin{align*}
x^4 + x^3 + x^2 + 1 &= (x+1)(x^3 + x + 1)\\
x^4 + x^3 + x + 1 &= (x+1)^2(x^2 + x + 1)\\
x^4 + x^2 + x + 1 &= (x+1)(x^3 + x^2 + 1)\\
x^4 + x^2 + 1 &= (x^2 + x + 1)^2\\
x^4 + 1 &= (x+1)^4
\end{align*}
We are left with the three polynomials $x^4 + x + 1$, $x^4 + x^3 + 1$ and $x^4 + x^3 + x^2 + x + 1$. Putting in $0$ and $1$ immediately shows that none of these polynomials has a linear factor. Consider the product of the two quadratics $(x^2 + ax + 1)(x^2 + bx + 1)$. But now note that the cases where $a = b = 1$, $a = 0$, $b = 1$ and $a = b = 0$ are all covered in the factorizations above. Thus these three polynomials are irreducible. Taking the product of these three polynomials as well as $x^2 + x + 1$, $x - 1$ and $x$ gives $x^{16} - x$.
\end{proof}

\begin{problem}[13.5.3]
Prove that $d$ divides $n$ if and only if $x^d - 1$ divides $x^n - 1$.
\end{problem}
\begin{proof}
Suppose $d \mid n$ and let $d' = n/d$. Then $(x^d - 1)(1 + x^d + x^{2d} + \dots + x^{(d'-1)d}) = x^{dd'} - 1 = x^n - 1$. Conversely suppose $x^d - 1 \mid x^n - 1$ so that $x^n - 1 = (x^d - 1)f(x)$. Write $n = qd + r$ so that $(x^d - 1)f(x) = x^n - 1 = (x^{qd + r} - x^r) + (x^r - 1) = x^r(x^{qd} - 1) + (x^r - 1)$. Since $x^d-1$ divides the left hand of this equation and divides $x^{qd-1} - 1$ (since $d \mid qd$) we see that it must also divide $x^r - 1$. But $r < d$ so $r = 0$ and $n = qd$.
\end{proof}

\begin{problem}[13.5.5]
For any prime $p$ and any nonzero $a \in \mathbb{F}_p$ prove that $x^p - x + a$ is irreducible and separable over $\mathbb{F}_p$.
\end{problem}
\begin{proof}
Suppose $x^p - x + a = (x^n + \dots + b)(x^m + \dots + c)$ with $n$ and $m$ nonzero. Using the product rule for derivatives, the derivative of the right side is $(nx^{n-1} + \dots + b')(x^m + \dots + c) + (mx^{m-1} + \dots + c')(x^n + \dots + b)$. Since $m$ and $n$ are both positive and nonzero we see that this polynomial must have degree $x^{m+n-1}$, but the derivative of the left hand side is $-1$. This is a contradiction and so $x^p - x + a$ must be irreducible. Since irreducible polynomials in a finite field are separable we must have $x^p - x + a$ is also separable.
\end{proof}

\begin{problem}[13.5.6]
Prove that $x^{p^n-1} - 1 = \prod_{\alpha \in \mathbb{F}_{p^n}^{\times}} (x - \alpha)$. Conclude that $\prod_{\alpha \in \mathbb{F}_{p^n}^{\times}} \alpha = (-1)^{p^n}$ so the product of the nonzero elements of a finite field is $+1$ if $p = 2$ and $-1$ if $p$ is odd. For $p$ odd and $n = 1$ derive \emph{Wilson's Theorem}: $(p-1)! \equiv -1 \pmod{p}$.
\end{problem}
\begin{proof}
Note that if $\alpha \in \mathbb{F}_{p^n}^{\times}$ then $\alpha^{p^n-1} = 1$. Consider the polynomial $f(x) = (x^{p^n-1} - 1) - \prod_{\alpha \in \mathbb{F}_{p^n}^{\times}} (x - \alpha)$. This has $p^n - 1$ roots, but degree less than $p^n - 1$, thus it must be identically $0$. If we set $x = 0$ in the above formula then we have $-1 = \prod_{\alpha \in \mathbb{F}_{p^n}^{\times}} (- \alpha) = (-1)^{p^n-1} \prod_{\alpha \in \mathbb{F}_{p^n}^{\times}} \alpha$ so $\prod_{\alpha \in \mathbb{F}_{p^n}^{\times}} \alpha = -(-1)^{-p^n+1} = -((-1)^{p^n-1})^{-1} = (-1)^{p^n}$. When $n = 1$ we have $\prod_{\alpha \in \mathbb{F}_p^{\times}} \alpha = (-1)^p$. Reducing this equation modulo $p$ we get $(p-1)! \equiv -1 \pmod{p}$. Note that the right hand side is clearly $-1$ for $p$ odd, and is easily verified for $p = 2$ since $1 \equiv -1 \pmod{2}$.
\end{proof}

\begin{problem}[13.6.2]
Let $\zeta_n$ be a primitive $n^{\textup{th}}$ root of unity and let $d$ be a divisor of $n$. Prove that $\zeta_n^d$ is a primitive $(n/d)^{\textup{th}}$ root of unity.
\end{problem}
\begin{proof}
Let $m$ be the order of $\zeta_n^d$ so that $(\zeta_n^d)^m = \zeta_n^{md} = 1$. Since $\zeta_n$ is an $n^{\textup{th}}$ root of unity we see that $n \mid md$ and since $m$ is minimal we must have $m = n/d$. Therefore $\zeta_n^d$ is a primitive $(n/d)^{\textup{th}}$ root of unity.
\end{proof}

\begin{problem}[13.6.7]
Use the M\"{o}bius Inversion formula indicated in Section 14.3 to prove
\[
\Phi_m(x) = \prod_{d \mid m} (x^d - 1)^{\mu(m/d)}.
\]
\end{problem}
\begin{proof}
We know $x^m - 1 = \prod_{d \mid m} \Phi_d(x)$. The M\"{o}bius inversion formula tells us that we can recover $\Phi_m(x)$ as $\Phi_m(x) = \prod_{d \mid m} (x^{m/d} - 1)^{\mu(d)}$. Changing the index to $d' = m/d$ gives us the desired result.
\end{proof}

\begin{problem}[13.6.8]
Let $\ell$ be a prime and let $\Phi_{\ell}(x) = \frac{x^{\ell}-1}{x-1} = x^{\ell-1} + x^{\ell-2} + \dots + x + 1 \in \mathbb{Z}[x]$ be the $\ell^{\textup{th}}$ cyclotomic polynomial, which is irreducible over $\mathbb{Z}$ by Theorem 41. This exercise determines the factorization of $\Phi_{\ell}(x)$ modulo $p$ for any prime $p$. Let $\zeta$ denote any fixed primitive $\ell^{\textup{th}}$ root of unity.\\
(a) Show that if $p = \ell$ then $\Phi_{\ell}(x) = (x-1)^{\ell-1} \in \mathbb{F}_{\ell}[x]$.\\
(b) Suppose $p \neq \ell$ and let $f$ denote the order of $p$ mod $\ell$, i.e., $f$ is the smallest power of $p$ with $p^f \equiv 1 \pmod{\ell}$. Use the fact that $\mathbb{F}_{p^n}^{\times}$ is a cyclic group to show that $n = f$ is the smallest power $p^n$ of $p$ with $\zeta \in \mathbb{F}_{p^n}$. Conclude that the minimal polynomial of $\zeta$ over $\mathbb{F}_p$ has degree $f$.\\
(c) Show that $\mathbb{F}_p(\zeta) = \mathbb{F}_p(\zeta^a)$ for any integer $a$ not divisible by $\ell$. Conclude using (b) that, in $\mathbb{F}_p[x]$, $\Phi_{\ell}(x)$ is the product of $\frac{\ell-1}{f}$ distinct irreducible polynomials of degree $f$.\\
(d) In particular, prove that, viewed in $\mathbb{F}_p[x]$, $\Phi_7(x) = x^6 + x^5 + \dots + x + 1$ is $(x-1)^6$ for $p = 7$, a product of distinct linear factors for $p \equiv 1 \pmod{7}$, a product of $3$ irreducible quadratics for $p \equiv 6 \pmod{7}$, a product of $2$ irreducible cubics for $p \equiv 2,4 \pmod{7}$, and is irreducible for $p \equiv 3,5 \pmod{7}$.
\end{problem}
\begin{proof}
(a) Over $\mathbb{F}_{\ell}$ we know $x^{\ell} - 1 = (x-1)^{\ell}$ so $\Phi_{\ell}(x) = (x-1)^{\ell}/(x-1) = (x-1)^{\ell-1}$.

(b) Suppose $\zeta$ is an $\ell^{\textup{th}}$ root of unity in the extension $\mathbb{F}_{p^n}$. Then $\zeta$ has order $\ell$ in $\mathbb{F}_{p^n}$ and $\ell \mid p^n - 1$. Thus $p^n \equiv 1 \pmod{\ell}$. On the other hand, suppose $p^n \equiv 1 \pmod{\ell}$ so that $\ell \mid p^n - 1$. Since $\mathbb{F}_{p^n}^{\times}$ is cyclic there exists some element of order $\ell$ in $\mathbb{F}_{p^n}$. Thus $\zeta \in \mathbb{F}_{p^n}$ if and only if $p^n \equiv 1 \pmod{l}$ which is true if and only if $f \mid n$. Thus $f$ is the smallest such integer for which this is true. The smallest extension over $\mathbb{F}_p$ containing $\zeta$ then must have degree $f$ so this is the degree of it's minimal polynomial.

(c) It's clear that $\mathbb{F}_p(\zeta^a) \subseteq \mathbb{F}_p(\zeta)$ since fields are closed under multiplication. Noting that $\zeta = (\zeta^a)^b$ with $ab \equiv 1 \pmod{\ell}$ we see that $\mathbb{F}_p(\zeta) \subseteq \mathbb{F}_p(\zeta^a)$.

For $(a,\ell) = 1$ we know $\zeta^a$ encompasses all the primitive roots modulo $\ell$ so $\mathbb{F}_p(\zeta) = \mathbb{F}_{p^f}$ is the unique extension of $\mathbb{F}_p$ of degree $f$ which contains all primitive roots modulo $\ell$. We then know the minimal polynomial for $\zeta^a$ is of degree $f$ as well so $\Phi_{\ell}(x) = m_1(x) \dots m_k(x)$ where $k = (\ell-1)/f$ since each $m_i(x)$ has degree $f$. We know each $m_i$ is distinct because $\Phi_{\ell}(x)$ is separable over $\mathbb{F}_p$ for $p \neq \ell$.

(d) Part (a) tells us that $\Phi_7(x) = (x-1)^6$ in $\mathbb{F}_7$. We can now compute $f$ for the various cases. If $p \equiv 1 \pmod{7}$ then $f = 1$ so $\Phi_7(x)$ splits into $(7-1)/1 = 6$ linear factors. If $p \equiv 6 \pmod{7}$ then $f = 2$ so $\Phi_7(x)$ splits into $(7-1)/2 = 3$ quadratics. If $p \equiv 2 \pmod{7}$ or $p \equiv 4 \pmod{7}$ then $f = 3$ so $\Phi_7(x)$ splits into $(7-1)/3 = 2$ cubics. And if $p \equiv 3 \pmod{7}$ or $p \equiv 5 \pmod{7}$ then $f = 6$ so $\Phi_7(x)$ is irreducible.
\end{proof}

\begin{problem}[14.1.4]
Prove that $\mathbb{Q}(\sqrt{2})$ and $\mathbb{Q}(\sqrt{3})$ are not isomorphic.
\end{problem}
\begin{proof}
We know $[\mathbb{Q}(\sqrt{2},\sqrt{3}) : \mathbb{Q}] = 4$ so the Galois group for this extension cannot have order larger than $4$. But the automorphisms taking $\sqrt{2}$ to $-\sqrt{2}$ and $\sqrt{3}$ to $-\sqrt{3}$ already enumerate $4$ maps. Thus there is no automorphism taking $\sqrt{2}$ to $\sqrt{3}$ so the two fields cannot be isomorphic.
\end{proof}

\begin{problem}[14.1.6]
Let $k$ be a field.\\
(a) Show that the mapping $\varphi : k[t] \to k[t]$ defined by $\varphi(f(t)) = f(at + b)$ for fixed $a,b \in k$, $a \neq 0$ is an automorphism of $k[t]$ which is the identity on $k$.\\
(b) Conversely, let $\varphi$ be an automorphism of $k[t]$ which is the identity on $k$. Prove that there exist $a,b \in k$ with $a \neq 0$ such that $\varphi(f(t)) = f(at + b)$ as in (a).
\end{problem}
\begin{proof}
(a) Let $f(t), g(t) \in k[t]$. Then $\varphi(f(t) + g(t)) = f(at+b) + g(at+b) = \varphi(f(t)) + \varphi(g(t))$ and $\varphi(f(t)g(t)) = f(at+b)g(at+b) = \varphi(f(t))\varphi(g(t))$ so $\varphi$ is a homomorphism. Note that $\varphi$ clearly fixes $k$ because if $f(t)$ is a constant function then $f(at+b)$ is the same constant function.

Now suppose $p(t) = c_n t^n + \dots + c_0$ and $q(t) = d_m t^m + \dots + d_0$ distinct elements of $k[t]$. If $p(x)$ and $q(x)$ have different degrees then their images are clearly distinct. Otherwise let $i$ be the largest index such that $c_i \neq d_i$. Then $\varphi(p(t))$ has the term $c_i (at+b)^i$ while $\varphi(q(t))$ has the term $d_i (at + b)^i$. Since the polynomials are identical for indices greater than $i$ there is no cancellation of the terms $c_iat^i \neq d_iat^i$. Thus $\varphi(p(t)) \neq \varphi(q(t))$ and $\varphi$ is injective. 

Finally consider the polynomial $c_n' (at + b)^n + \dots + c_0'$. If we expand this out and compare degrees with $p(t)$ we can recursively solve for the $c_i'$ in terms of the $c_i$. That is, first solve for $c_n'$ in terms of $c_n$, $a$ and $b$, then solve for $c_{n-1}'$ in terms of $c_n$, $c_{n-1}$, $a$ and $b$. Continue in this way until we can rewrite $c_n' (at + b)^n + \dots + c_0$ in terms of $c_i$, $a$ and $b$. But then applying $\varphi$ to this polynomial will give back $p(t)$ so $\varphi$ is surjective as well.

(b) Now suppose $\varphi$ is an automorphism of $k[t]$ fixing $k$. Let $f(t) = c_n t^n + \dots + c_0$ be any element of $k[t]$. Note that $\varphi(f(t)) = \varphi(c_nt^n) + \dots + \varphi(c_0) = c_n \varphi(t^n) + \dots + c_0$ since $\varphi$ is a homomorphism fixing $k$. Thus $\varphi$ is completely determined by which polynomial it sends $t$ to. Note that $\deg \varphi(t) \leq 1$ since, for example, we cannot have $\varphi(t) = t^2 = \varphi(t) \varphi(t)$ so that $t^2 = 1$. On the other hand $\varphi(t) \neq c$ some constant because then $\varphi$ is clearly not surjective onto $k[t]$. Thus $\varphi(t) = at + b$ with $a \neq 0$ so that $\varphi(f(t)) = f(at + b)$ for some $a,b \in k$.
\end{proof}

\begin{problem}[14.1.10]
Let $K$ be an extension of the field $F$. Let $\varphi : K \to K'$ be an isomorphism of $K$ with a field $K'$ which maps $F$ to the subfield $F'$ of $K'$. Prove that the map $\sigma \mapsto \varphi \sigma \varphi^{-1}$ defines a group isomorphism $\aut(K/F) \to \aut(K'/F')$.
\end{problem}
\begin{proof}
Let $\sigma \in \aut(K/F)$ and $x,y \in K'$. Then $\varphi(\sigma(\varphi^{-1}(x + y))) = \varphi(\sigma(\varphi^{-1}(x) + \varphi^{-1}(y))) = \varphi(\sigma(\varphi^{-1}(x)) + \sigma(\varphi^{-1}(y))) = \varphi(\sigma(\varphi^{-1}(x))) + \varphi(\sigma(\varphi^{-1}(y)))$ so this map is a homomorphism. Furthermore the map is injective and surjective since it's the composition of injective and surjective maps. Thus $\varphi \sigma \varphi^{-1}$ is an element of $\aut(K'/F')$.

Let $\sigma, \sigma' \in \aut(K/F)$ so that $\varphi (\sigma \sigma') \varphi^{-1} = \varphi \sigma \varphi^{-1} \varphi \sigma' \varphi^{-1}$ and the map is a homomorphism. Suppose $\sigma \neq \sigma'$ such that $\sigma(x) \neq \sigma'(x)$ for some $x \in K$. Let $y = \varphi(x)$. Then since $\varphi$ is injective $\varphi \sigma \varphi^{-1}(y) = \varphi \sigma (x) \neq \varphi \sigma'(x) = \varphi \sigma' \varphi^{-1} (y)$ so the map is injective.

Let $\tau \in \aut(K'/F')$. Then we've already seen $\varphi^{-1} \tau \varphi$ gives an element $\sigma \in \aut(K/F)$. But then multiplying on the left by $\varphi$ and on the right by $\varphi^{-1}$ gives $\varphi \sigma \varphi^{-1} = \tau$ so the map is surjective and thus an isomorphism.
\end{proof}

\end{document}