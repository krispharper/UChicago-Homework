\documentclass{article}
\usepackage{amsmath,amsthm,amssymb,amsfonts,fullpage,fancyhdr,longtable}

\pagestyle{fancy}
\renewcommand{\headheight}{50pt}
\renewcommand{\footskip}{10pt}
\renewcommand{\textheight}{600pt}
\renewcommand{\headrulewidth}{0pt}

\newcommand{\gal}{\textup{Gal}}
\newcommand{\norm}{\textup{N}}
\newcommand{\tr}{\textup{Tr}}

\newtheorem{problem}{Problem}

\begin{document}

\rhead{Kris Harper\\MATH 25900\\May 5, 2010\\}
\chead{Quiz 3\\}

Consider $K = \mathbb{Q}(\sqrt{2}, \sqrt{3})$. Referring to \#17 and \#18 in D\&F page 582, compute each of the following and explain your answers briefly:

\begin{problem}
(a) $\norm_{K/\mathbb{Q}} (\sqrt{2})$.\\
(b) $\norm_{K/\mathbb{Q}} (\sqrt{6})$.\\
(c) $\norm_{K/\mathbb{Q}} (\sqrt{2} + \sqrt{3})$.\\
(d) $\norm_{K/\mathbb{Q}} (2)$.
\end{problem}
\begin{proof}
(a) The subgroup we're interested in is simply all of $\gal(K/\mathbb{Q}) = \{1, \sigma, \tau, \sigma \tau\}$ where $\sigma : \sqrt{2} \mapsto -\sqrt{2}$ and $\tau : \sqrt{3} \mapsto -\sqrt{3}$. Then
\[
\norm_{K/\mathbb{Q}} (\sqrt{2}) = \sqrt{2} \sigma(\sqrt{2}) \tau(\sqrt{2}) \sigma(\tau(\sqrt{2})) = (\sqrt{2})(-\sqrt{2})(\sqrt{2})(-\sqrt{2}) = 4.
\]

(b) Note $\sqrt{6} = \sqrt{2} \sqrt{3}$ so
\[
\norm_{K/\mathbb{Q}} (\sqrt{6}) = \sqrt{6} \sigma(\sqrt{6}) \tau(\sqrt{6}) \sigma(\tau(\sqrt{6})) = (\sqrt{6}) (-\sqrt{2} \sqrt{3}) (-\sqrt{3} \sqrt{2}) (-\sqrt{2})(-\sqrt{3}) = 36.
\]

(c) We have
\begin{align*}
\norm_{K/\mathbb{Q}} (\sqrt{2} + \sqrt{3})
&= (\sqrt{2} + \sqrt{3}) (\sigma(\sqrt{2} + \sqrt{3})) (\tau(\sqrt{2} + \sqrt{3})) (\sigma(\tau(\sqrt{2} + \sqrt{3})))\\
&= (\sqrt{2} + \sqrt{3}) (-\sqrt{2} + \sqrt{3}) (\sqrt{2} - \sqrt{3}) (-\sqrt{2} - \sqrt{3}) = 1.
\end{align*}

(d) Note $2 \in \mathbb{Q}$ so it's fixed by all the elements of $\gal(K/\mathbb{Q})$. Thus $\norm_{K/\mathbb{Q}} (2) = 16$.
\end{proof}

\begin{problem}
(a) $\tr_{K/\mathbb{Q}} (\sqrt{2})$.\\
(b) $\tr_{K/\mathbb{Q}} (\sqrt{6})$.\\
(c) $\tr_{K/\mathbb{Q}} (\sqrt{2} + \sqrt{3})$.\\
(d) $\tr_{K/\mathbb{Q}} (2)$.
\end{problem}
\begin{proof}
(a) We have
\[
\tr_{K/\mathbb{Q}} (\sqrt{2}) = \sqrt{2} + \sigma(\sqrt{2}) + \tau(\sqrt{2}) + \sigma(\tau(\sqrt{2})) = \sqrt{2} - \sqrt{2} + \sqrt{2} - \sqrt{2} = 0.
\]

(b) Note $\sqrt{6} = \sqrt{2} \sqrt{3}$ so
\[
\tr_{K/\mathbb{Q}} (\sqrt{6}) = \sqrt{6} + \sigma(\sqrt{6}) + \tau(\sqrt{6}) + \sigma(\tau(\sqrt{6})) = \sqrt{6} - \sqrt{2} \sqrt{3} - \sqrt{3} \sqrt{2} + \sqrt{2} \sqrt{3} = 0.
\]

(c) We have
\begin{align*}
\tr_{K/\mathbb{Q}} (\sqrt{2} + \sqrt{3})
&= (\sqrt{2} + \sqrt{3}) + \sigma(\sqrt{2} + \sqrt{3}) + \tau(\sqrt{2} + \sqrt{3}) + \sigma(\tau(\sqrt{2} + \sqrt{3}))\\
&= (\sqrt{2} + \sqrt{3}) + (-\sqrt{2} + \sqrt{3}) + (\sqrt{2} - \sqrt{3}) + (-\sqrt{2} - \sqrt{3}) = 0.
\end{align*}

(d) Note $2 \in \mathbb{Q}$ so it's fixed by all the elements of $\gal(K/\mathbb{Q})$. Thus $\tr_{K/\mathbb{Q}} (2) = 8$.
\end{proof}

\end{document}