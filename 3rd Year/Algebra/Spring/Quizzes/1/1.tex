\documentclass{article}
\usepackage{amsmath,amsthm,amssymb,amsfonts,fullpage,fancyhdr,longtable}

\pagestyle{fancy}
\renewcommand{\headheight}{50pt}
\renewcommand{\footskip}{10pt}
\renewcommand{\textheight}{600pt}
\renewcommand{\headrulewidth}{0pt}

\newtheorem{problem}{Problem}

\begin{document}

\rhead{Kris Harper\\MATH 25900\\April 7, 2010\\}
\chead{Quiz 1\\}

\begin{problem}
Find the rational canonical form of
\[
R_{\theta} =
\left (
\begin{array}{cc}
\cos \theta & -\sin \theta\\
\sin \theta & \cos \theta
\end{array}
\right ).
\]
Show $R_{\theta}$ is similar to $R_{\phi}$ in $M_2(\mathbb{R})$ iff $\theta = \pm \phi$. Find eigenvalues of $R_{\theta}$.
\end{problem}
\begin{proof}
First we restrict the values of $\theta$ to the interval $[-\pi,\pi)$. We find the characteristic polynomial of $R_{\theta}$, $c_{R_{\theta}}(x) = (x-\cos \theta)^2 + \sin^2 \theta = x^2 - 2x\cos \theta + \cos^2 \theta + \sin^2 \theta = x^2 - 2x\cos \theta + 1$. Using the quadratic formula we find that the solutions to $c_{R_{\theta}}(x)$ are $\cos \theta \pm \sqrt{\cos^2 \theta - 1}$. These are then the eigenvalues for $R_{\theta}$. Note that they're only real-valued if $\cos^2 \theta = 1$ so $\theta \in \{-\pi,0\}$ and in this case the eigenvalues simplify to either $-1$ or $1$.

Note that $(R_{\theta} - (\cos \theta \pm \sqrt{\cos^2 \theta - 1})I) \neq 0$ so the minimal polynomial $m_{R_{\theta}}(x) = c_{R_{\theta}}(x)$. Thus the rational canonical form is simply the $2 \times 2$ companion matrix for $c_{R_{\theta}}(x)$
\[
\left (
\begin{array}{cc}
0 & -1\\
1 & 2\cos \theta
\end{array}
\right).
\]
Finally note that $R_{\theta}$ is similar to $R_{\phi}$ if and only if $R_{\theta}$ and $R_{\phi}$ have the same rational canonical form. Thus, they're similar if and only if $2\cos \theta = 2 \cos \phi$ which is true if and only if $\theta = \pm \phi$ since $\theta,\phi \in [-\pi,\pi)$.
\end{proof}

\begin{problem}
Find all possible Jordan forms for all $8 \times 8$ matrices having $x^2(x-1)^3$ as a minimal polynomial.
\end{problem}
\begin{proof}
If $x^2(x-1)^3$ is the minimal polynomial for a matrix then the elementary divisors are powers of $x$ and $(x-1)$ such that $x^2$ and $(x-3)^3$ appear at least once and the product of all the elementary divisors must be an eighth degree polynomial. This generates the following possible lists of elementary divisors. We have
\[
\begin{tabular}{cccc}
$x$, $x$, $x$, $x^2$, $(x-1)^3$ & $x$, $x^2$, $x^2$, $(x-1)^3$ & $x$, $x$, $x^2$, $(x-1)$, $(x-1)^3$\\
$x^2$, $x^2$, $(x-1)$, $(x-1)^3$ & $x$, $x^2$, $(x-1)$, $(x-1)$, $(x-1)^3$ & $x$, $x^2$, $(x-1)^2$, $(x-1)^3$\\
$x^2$, $(x-1)$, $(x-1)$, $(x-1)$, $(x-1)^3$ & $x^2$, $(x-1)$, $(x-1)^2$, $(x-1)^3$ & $x^2$, $(x-1)^3$, $(x-1)^3$
\end{tabular}
\]
These respectively have the following Jordan forms up to a permutation in their Jordan blocks. We have
\begin{longtable}{cc}
$
\left (
\begin{array}{ccccc}
\begin{array}{ccc|}
1 & 1 & 0\\
0 & 1 & 1\\
0 & 0 & 1\\ \hline
\end{array}
&
\begin{array}{cc}
0 & 0\\
0 & 0\\
0 & 0\\ \hline
\end{array}
&
\begin{array}{c}
0\\
0\\
0
\end{array}
&
\begin{array}{c}
0\\
0\\
0
\end{array}
&
\begin{array}{c}
0\\
0\\
0
\end{array}\\
\begin{array}{ccc}
0 & 0 & 0\\
0 & 0 & 0
\end{array}
&
\begin{array}{|cc|}
0 & 1\\
0 & 0\\ \hline
\end{array}
&
\begin{array}{c}
0\\
0\\ \hline
\end{array}
&
\begin{array}{c}
0\\
0
\end{array}
&
\begin{array}{c}
0\\
0
\end{array}\\
\begin{array}{ccc}
0 & 0 & 0
\end{array}
&
\begin{array}{cc}
0 & 0
\end{array}
&
\begin{array}{|c|}
0\\ \hline
\end{array}
&
\begin{array}{c}
0\\ \hline
\end{array}
&
\begin{array}{c}
0
\end{array}\\
\begin{array}{ccc}
0 & 0 & 0
\end{array}
&
\begin{array}{cc}
0 & 0
\end{array}
&
\begin{array}{c}
0
\end{array}
&
\begin{array}{|c|}
0\\ \hline
\end{array}
&
\begin{array}{c}
0\\ \hline
\end{array}\\
\begin{array}{ccc}
0 & 0 & 0
\end{array}
&
\begin{array}{cc}
0 & 0
\end{array}
&
\begin{array}{c}
0
\end{array}
&
\begin{array}{c}
0
\end{array}
&
\begin{array}{|c}
0
\end{array}
\end{array}
\right )
$
&
$
\left (
\begin{array}{cccc}
\begin{array}{ccc|}
1 & 1 & 0\\
0 & 1 & 1\\
0 & 0 & 1\\ \hline
\end{array}
&
\begin{array}{cc}
0 & 0\\
0 & 0\\
0 & 0\\ \hline
\end{array}
&
\begin{array}{cc}
0 & 0\\
0 & 0\\
0 & 0
\end{array}
&
\begin{array}{c}
0\\
0\\
0
\end{array}\\
\begin{array}{ccc}
0 & 0 & 0\\
0 & 0 & 0
\end{array}
&
\begin{array}{|cc|}
0 & 1\\
0 & 0\\ \hline
\end{array}
&
\begin{array}{cc}
0 & 0\\
0 & 0\\ \hline
\end{array}
&
\begin{array}{c}
0\\
0
\end{array}\\
\begin{array}{ccc}
0 & 0 & 0\\
0 & 0 & 0
\end{array}
&
\begin{array}{cc}
0 & 0\\
0 & 0
\end{array}
&
\begin{array}{|cc|}
0 & 1\\
0 & 0\\ \hline
\end{array}
&
\begin{array}{c}
0\\
0\\ \hline
\end{array}\\
\begin{array}{ccc}
0 & 0 & 0
\end{array}
&
\begin{array}{cc}
0 & 0
\end{array}
&
\begin{array}{cc}
0
\end{array}
&
\begin{array}{|c}
0
\end{array}\\
\end{array}
\right )
$
\\\\
$
\left (
\begin{array}{ccccc}
\begin{array}{ccc|}
1 & 1 & 0\\
0 & 1 & 1\\
0 & 0 & 1\\ \hline
\end{array}
&
\begin{array}{cc}
0 & 0\\
0 & 0\\
0 & 0\\ \hline
\end{array}
&
\begin{array}{c}
0\\
0\\
0
\end{array}
&
\begin{array}{c}
0\\
0\\
0
\end{array}
&
\begin{array}{c}
0\\
0\\
0
\end{array}\\
\begin{array}{ccc}
0 & 0 & 0\\
0 & 0 & 0
\end{array}
&
\begin{array}{|cc|}
0 & 1\\
0 & 0\\ \hline
\end{array}
&
\begin{array}{c}
0\\
0\\ \hline
\end{array}
&
\begin{array}{c}
0\\
0
\end{array}
&
\begin{array}{c}
0\\
0
\end{array}\\
\begin{array}{ccc}
0 & 0 & 0
\end{array}
&
\begin{array}{cc}
0 & 0
\end{array}
&
\begin{array}{|c|}
1\\ \hline
\end{array}
&
\begin{array}{c}
0\\ \hline
\end{array}
&
\begin{array}{c}
0
\end{array}\\
\begin{array}{ccc}
0 & 0 & 0
\end{array}
&
\begin{array}{cc}
0 & 0
\end{array}
&
\begin{array}{c}
0
\end{array}
&
\begin{array}{|c|}
0\\ \hline
\end{array}
&
\begin{array}{c}
0\\ \hline
\end{array}\\
\begin{array}{ccc}
0 & 0 & 0
\end{array}
&
\begin{array}{cc}
0 & 0
\end{array}
&
\begin{array}{c}
0
\end{array}
&
\begin{array}{c}
0
\end{array}
&
\begin{array}{|c}
0
\end{array}
\end{array}
\right )
$
&
$\left (
\begin{array}{cccc}
\begin{array}{ccc|}
1 & 1 & 0\\
0 & 1 & 1\\
0 & 0 & 1\\ \hline
\end{array}
&
\begin{array}{cc}
0 & 0\\
0 & 0\\
0 & 0\\ \hline
\end{array}
&
\begin{array}{cc}
0 & 0\\
0 & 0\\
0 & 0
\end{array}
&
\begin{array}{c}
0\\
0\\
0
\end{array}\\
\begin{array}{ccc}
0 & 0 & 0\\
0 & 0 & 0
\end{array}
&
\begin{array}{|cc|}
0 & 1\\
0 & 0\\ \hline
\end{array}
&
\begin{array}{cc}
0 & 0\\
0 & 0\\ \hline
\end{array}
&
\begin{array}{c}
0\\
0
\end{array}\\
\begin{array}{ccc}
0 & 0 & 0\\
0 & 0 & 0
\end{array}
&
\begin{array}{cc}
0 & 0\\
0 & 0
\end{array}
&
\begin{array}{|cc|}
0 & 1\\
0 & 0\\ \hline
\end{array}
&
\begin{array}{c}
0\\
0\\ \hline
\end{array}\\
\begin{array}{ccc}
0 & 0 & 0
\end{array}
&
\begin{array}{cc}
0 & 0
\end{array}
&
\begin{array}{cc}
0
\end{array}
&
\begin{array}{|c}
1
\end{array}\\
\end{array}
\right )
$
\\\\
$
\left (
\begin{array}{ccccc}
\begin{array}{ccc|}
1 & 1 & 0\\
0 & 1 & 1\\
0 & 0 & 1\\ \hline
\end{array}
&
\begin{array}{cc}
0 & 0\\
0 & 0\\
0 & 0\\ \hline
\end{array}
&
\begin{array}{c}
0\\
0\\
0
\end{array}
&
\begin{array}{c}
0\\
0\\
0
\end{array}
&
\begin{array}{c}
0\\
0\\
0
\end{array}\\
\begin{array}{ccc}
0 & 0 & 0\\
0 & 0 & 0
\end{array}
&
\begin{array}{|cc|}
0 & 1\\
0 & 0\\ \hline
\end{array}
&
\begin{array}{c}
0\\
0\\ \hline
\end{array}
&
\begin{array}{c}
0\\
0
\end{array}
&
\begin{array}{c}
0\\
0
\end{array}\\
\begin{array}{ccc}
0 & 0 & 0
\end{array}
&
\begin{array}{cc}
0 & 0
\end{array}
&
\begin{array}{|c|}
1\\ \hline
\end{array}
&
\begin{array}{c}
0\\ \hline
\end{array}
&
\begin{array}{c}
0
\end{array}\\
\begin{array}{ccc}
0 & 0 & 0
\end{array}
&
\begin{array}{cc}
0 & 0
\end{array}
&
\begin{array}{c}
0
\end{array}
&
\begin{array}{|c|}
1\\ \hline
\end{array}
&
\begin{array}{c}
0\\ \hline
\end{array}\\
\begin{array}{ccc}
0 & 0 & 0
\end{array}
&
\begin{array}{cc}
0 & 0
\end{array}
&
\begin{array}{c}
0
\end{array}
&
\begin{array}{c}
0
\end{array}
&
\begin{array}{|c}
0
\end{array}
\end{array}
\right )
$
&
$
\left (
\begin{array}{cccc}
\begin{array}{ccc|}
1 & 1 & 0\\
0 & 1 & 1\\
0 & 0 & 1\\ \hline
\end{array}
&
\begin{array}{cc}
0 & 0\\
0 & 0\\
0 & 0\\ \hline
\end{array}
&
\begin{array}{cc}
0 & 0\\
0 & 0\\
0 & 0
\end{array}
&
\begin{array}{c}
0\\
0\\
0
\end{array}\\
\begin{array}{ccc}
0 & 0 & 0\\
0 & 0 & 0
\end{array}
&
\begin{array}{|cc|}
0 & 1\\
0 & 0\\ \hline
\end{array}
&
\begin{array}{cc}
0 & 0\\
0 & 0\\ \hline
\end{array}
&
\begin{array}{c}
0\\
0
\end{array}\\
\begin{array}{ccc}
0 & 0 & 0\\
0 & 0 & 0
\end{array}
&
\begin{array}{cc}
0 & 0\\
0 & 0
\end{array}
&
\begin{array}{|cc|}
1 & 1\\
0 & 1\\ \hline
\end{array}
&
\begin{array}{c}
0\\
0\\ \hline
\end{array}\\
\begin{array}{ccc}
0 & 0 & 0
\end{array}
&
\begin{array}{cc}
0 & 0
\end{array}
&
\begin{array}{cc}
0
\end{array}
&
\begin{array}{|c}
0
\end{array}\\
\end{array}
\right )
$
\\\\
$
\left (
\begin{array}{ccccc}
\begin{array}{ccc|}
1 & 1 & 0\\
0 & 1 & 1\\
0 & 0 & 1\\ \hline
\end{array}
&
\begin{array}{cc}
0 & 0\\
0 & 0\\
0 & 0\\ \hline
\end{array}
&
\begin{array}{c}
0\\
0\\
0
\end{array}
&
\begin{array}{c}
0\\
0\\
0
\end{array}
&
\begin{array}{c}
0\\
0\\
0
\end{array}\\
\begin{array}{ccc}
0 & 0 & 0\\
0 & 0 & 0
\end{array}
&
\begin{array}{|cc|}
0 & 1\\
0 & 0\\ \hline
\end{array}
&
\begin{array}{c}
0\\
0\\ \hline
\end{array}
&
\begin{array}{c}
0\\
0
\end{array}
&
\begin{array}{c}
0\\
0
\end{array}\\
\begin{array}{ccc}
0 & 0 & 0
\end{array}
&
\begin{array}{cc}
0 & 0
\end{array}
&
\begin{array}{|c|}
1\\ \hline
\end{array}
&
\begin{array}{c}
0\\ \hline
\end{array}
&
\begin{array}{c}
0
\end{array}\\
\begin{array}{ccc}
0 & 0 & 0
\end{array}
&
\begin{array}{cc}
0 & 0
\end{array}
&
\begin{array}{c}
0
\end{array}
&
\begin{array}{|c|}
1\\ \hline
\end{array}
&
\begin{array}{c}
0\\ \hline
\end{array}\\
\begin{array}{ccc}
0 & 0 & 0
\end{array}
&
\begin{array}{cc}
0 & 0
\end{array}
&
\begin{array}{c}
0
\end{array}
&
\begin{array}{c}
0
\end{array}
&
\begin{array}{|c}
1
\end{array}
\end{array}
\right )
$
&
$
\left (
\begin{array}{cccc}
\begin{array}{ccc|}
1 & 1 & 0\\
0 & 1 & 1\\
0 & 0 & 1\\ \hline
\end{array}
&
\begin{array}{cc}
0 & 0\\
0 & 0\\
0 & 0\\ \hline
\end{array}
&
\begin{array}{cc}
0 & 0\\
0 & 0\\
0 & 0
\end{array}
&
\begin{array}{c}
0\\
0\\
0
\end{array}\\
\begin{array}{ccc}
0 & 0 & 0\\
0 & 0 & 0
\end{array}
&
\begin{array}{|cc|}
0 & 1\\
0 & 0\\ \hline
\end{array}
&
\begin{array}{cc}
0 & 0\\
0 & 0\\ \hline
\end{array}
&
\begin{array}{c}
0\\
0
\end{array}\\
\begin{array}{ccc}
0 & 0 & 0\\
0 & 0 & 0
\end{array}
&
\begin{array}{cc}
0 & 0\\
0 & 0
\end{array}
&
\begin{array}{|cc|}
1 & 1\\
0 & 1\\ \hline
\end{array}
&
\begin{array}{c}
0\\
0\\ \hline
\end{array}\\
\begin{array}{ccc}
0 & 0 & 0
\end{array}
&
\begin{array}{cc}
0 & 0
\end{array}
&
\begin{array}{cc}
0
\end{array}
&
\begin{array}{|c}
1
\end{array}\\
\end{array}
\right )
$
\\\\
\multicolumn{2}{c}{
$
\left (
\begin{array}{ccc}
\begin{array}{ccc|}
1 & 1 & 0\\
0 & 1 & 1\\
0 & 0 & 1\\ \hline
\end{array}
&
\begin{array}{ccc}
0 & 0 & 0\\
0 & 0 & 0\\
0 & 0 & 0\\ \hline
\end{array}
&
\begin{array}{cc}
0 & 0\\
0 & 0\\
0 & 0
\end{array}\\
\begin{array}{ccc}
0 & 0 & 0\\
0 & 0 & 0\\
0 & 0 & 0
\end{array}
&
\begin{array}{|ccc|}
1 & 1 & 0\\
0 & 1 & 1\\
0 & 0 & 1\\ \hline
\end{array}
&
\begin{array}{cc}
0 & 0\\
0 & 0\\
0 & 0\\ \hline
\end{array}\\
\begin{array}{ccc}
0 & 0 & 0\\
0 & 0 & 0
\end{array}
&
\begin{array}{ccc}
0 & 0 & 0\\
0 & 0 & 0
\end{array}
&
\begin{array}{|cc}
0 & 1\\
0 & 0
\end{array}
\end{array}
\right )
$
}
\end{longtable}
\end{proof}

\begin{problem}
Show that the matrix $A$ has only one Jordan block of size $3$ iff $ab \neq 0$, where
\[
A =
\left (
\begin{array}{ccc}
1 & a & c\\
0 & 1 & b\\
0 & 0 & 1
\end{array}
\right ).
\]
\end{problem}
\begin{proof}
Note that the characteristic polynomial is $c_A(x) = (x-1)^3$. We will produce a Jordan block of size $3$ if and only if the minimal polynomial is $c_A(x)$ since this ensures that there will only be one invariant factor, $(x-1)^3$, and thus one elementary divisor. But this will happen if and only if $(A-I)^2 \neq 0$. Note that
\[
(A-I)^2 =
\left (
\begin{array}{ccc}
0 & a & c\\
0 & 0 & b\\
0 & 0 & 0
\end{array}
\right )
\left (
\begin{array}{ccc}
0 & a & c\\
0 & 0 & b\\
0 & 0 & 0
\end{array}
\right )
=
\left (
\begin{array}{ccc}
0 & 0 & ab\\
0 & 0 & 0\\
0 & 0 & 0
\end{array}
\right )
\]
So $(A-I)^2 \neq 0$ if and only if $ab \neq 0$.
\end{proof}

\begin{problem}
Find characteristic polynomial, minimal polynomial, invariant factors, elementary divisors, rational canonical form and Jordan canonical form form the following matrix over $\mathbb{Q}$:
\[
A =
\left (
\begin{array}{cccc}
0 & 3 & 0 & 0\\
1 & 2 & 0 & 0\\
0 & 0 & 0 & 0\\
0 & 0 & 0 & 2
\end{array}
\right ).
\]
\end{problem}
\begin{proof}
Note that $A$ is already in rational canonical form It has a companion matrix of size $2$ in the upper left corner and two companion matrices of size $1$ following. The corresponding invariant factors are $x^2-2x-3$, $x$ and $x-2$. The characteristic polynomial is then $c_A(x) = (x^2-2x-3)x(x-2)$. As a check we can take the determinant of $xI-A$. This is
\begin{align*}
c_A(x) = \det (xI-A)
&= \det \left (
\left (
\begin{array}{cccc}
x & -3 & 0 & 0\\
-1 & x-2 & 0 & 0\\
0 & 0 & x & 0\\
0 & 0 & 0 & x-2
\end{array}
\right ) \right )\\
&= x^4 - 4x^3 + x^2 + 6x = (x^2-2x-3)x(x-2)\\
&= x(x+1)(x-2)(x-3).
\end{align*}
Since the characteristic polynomial is composed of relatively prime factors, the minimal polynomial must be equal to the characteristic polynomial so $m_A(x) = c_A(x) = (x^2-2x-3)x(x-2)$. As stated earlier, the invariant factors are $x^2-2x-3$, $x$ and $x-2$. The elementary divisors are the prime powers of these factors, so they must be $x$, $(x+1)$, $(x-2)$ and $(x-3)$. We've already stated that $A$ is in rational canonical form as given, and the list of elementary divisors shows that the Jordan form is
\[
\left (
\begin{array}{cccc}
3 & 0 & 0 & 0\\
0 & 2 & 0 & 0\\
0 & 0 & -1 & 0\\
0 & 0 & 0 & 0\\
\end{array}
\right ).
\]
\end{proof}

\end{document}