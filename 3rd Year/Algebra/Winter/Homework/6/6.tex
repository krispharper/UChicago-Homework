\documentclass{article}
\usepackage{amsmath,amsthm,amssymb,amsfonts,fullpage,fancyhdr}

\pagestyle{fancy}
\renewcommand{\headheight}{50pt}
\renewcommand{\footskip}{10pt}
\renewcommand{\textheight}{609pt}
\renewcommand{\headrulewidth}{0pt}

\newcommand{\ann}{\textup{Ann}}

\newtheorem{problem}{Problem}

\begin{document}

\rhead{Kris Harper\\MATH 25800\\March 5, 2010\\}
\chead{Homework 6\\}

\begin{problem}[11.3.3]
Let $S$ be any subset of $V^*$ for some finite dimensional space $V$. Define $\ann(S) = \{v \in V \mid f(v) = 0 \textup{ for all } f \in S\}$. ($\ann(S)$ is called the \emph{annihilator} of $S$ in $V$).\\
(a) Prove that $\ann(S)$ is a subspace of $V$.\\
(b) Let $W_1$ and $W_2$ be subspaces of $V^*$. Prove that $\ann(W_1 + W_2) = \ann(W_1) \cap \ann(W_2)$ and $\ann(W_1 \cap W_2) = \ann(W_1) + \ann(W_2)$.\\
(c) Let $W_1$ and $W_2$ be subspaces of $V^*$. Prove that $W_1 = W_2$ if and only if $\ann(W_1) = \ann(W_2)$.
\end{problem}

\begin{proof}
(a) Note that $0 \in \ann(S)$ so $\ann(S) \neq \emptyset$. Let $v,u \in \ann(S)$ and let $r \in F$. Let $f \in V^*$. Then $f(rv + u) = rf(v) + f(u) = 0$ because $f$ is linear. Thus $rv + u \in \ann(S)$ as well and $\ann(S)$ is a subspace of $V$.

(b) Let $v \in \ann(W_1 + W_2)$. Then for each $f \in W_1$ and $g \in W_2$ we have $0 = (f+g)(v) = f(v) + g(v)$. In particular, if $g$ is the zero function, then $f(v) = 0$ necessarily. The same is true for $g$ and so $f(v) = g(v) = 0$. Thus $v \in \ann(W_1) \cap \ann(W_2)$. Conversely, suppose $v \in \ann(W_1) \cap \ann(W_2)$. Then for each $f \in W_1$ and $g \in W_2$ we have $f(v) = g(v) = 0$. But then $f(v) + g(v) = (f+g)(v) = 0$ and $v \in \ann(W_1 + W_2)$.

Now note that
\begin{align*}
\ann(W_1 \cap W_2)
&= \{v \in V \mid \text{$f(v) = 0$ for all $f \in W_1 \cap W_2$}\}\\
&= \{v \in V \mid \text{$(f+g)(v) = 0$ for all $f \in W_1$ and $g \in W_2$}\}\\
&= \{v + u \in V \mid \text{$f(v) = 0$ for all $f \in W_1$ and $g(u) = 0$ for all $g \in W_2$}\}\\
&= \ann(W_1) + \ann(W_2).
\end{align*}

(c) Suppose $W_1 = W_2$. Let $v \in \ann(W_1)$ and let $f \in W_2$. Since $W_1 = W_2$, $f \in W_1$ as well, so $f(v) = 0$. Since $f$ was arbitrary, $v \in \ann(W_2)$. The second inclusion holds similarly. Conversely, suppose $\ann(W_1) = \ann(W_2)$. Let $f \in W_1$. Since $W_1$ and $W_2$ are subspace of $V^*$ and $f$ agrees with every function of $W_2$ on the vectors in $\ann(W_1)$, it follows that $f \in W_2$ as well. The second inclusion follows similarly.
\end{proof}

\begin{problem}
(a) Prove that the elementary row operations have the following effect on determinants:\\
(i) Interchanging two rows changes the sign of the determinant.\\
(ii) Add a multiple of a row to another does not alter the determinant.\\
(iii) Multiplying any row by a nonzero element $u$ from $F$ multiplies the determinant by $u$.\\
(b) Prove that $\det A$ is nonzero if and only if $A$ is row equivalent to the $n \times n$ identity matrix. Suppose $A$ can be row reduced to the identity matrix using a total of $s$ row interchanges as in (i) and by multiplying rows by the nonzero elements $u_1, u_2, \dots , u_t$ as in (iii). Prove that $\det A = (-1)^s(u_1u_2 \dots u_t)^{-1}$.
\end{problem}
\begin{proof}
(a) These all follow from the fact that $\det$ is alternating and multilinear and that $\det(A) = \det(A^t)$. In particular, if $A$ is a matrix with columns $A_1, \dots A_n$, then we know
\[
\det(A_1, \dots , A_i, \dots , A_j, \dots , A_n) = -\det(A_1, \dots , A_j, \dots , A_i, \dots , A_n).
\]
Also,
\[
\det(A_1, \dots , A_i + kA_j, \dots , A_n) = \det(A_1, \dots , A_i, \dots , A_n) + k\det(A_1, \dots , A_j, \dots , A_j, \dots , A_n) = \det(A) + 0.
\]
Finally, note
\[
\det(A_1, \dots , kA_i, \dots , A_n) = k \det(A_1, \dots , A_i, \dots , A_n).
\]
Looking at $A^t$ instead of $A$ will translate all of these statements about columns to rows.

(b) Suppose $A$ is row equivalent to the identity. Then a finite number of the operations (i), (ii) and (iii) will result in the identity matrix. Part (a) states that the determinant of $A$ will then be a finite number of constants multiplied together with a possible sign change. Conversely, if $\det A$ is nonzero then the rows $A_1, \dots , A_n$ must be linearly independent. This means we can perform row operations on them until we end up with the identity matrix. In particular, if $A$ is reduced to the identity in $s$ row interchanges, then part (a) tells us that $\det(A)$ will change by $(-1)^s$ from $\det(I)$. If $A$ is reduced by multiplying by the elements $u_1, \dots , u_t$, then part (a) tells us $\det(A)$ will change by $u_1 \dots u_t$ from $\det(I)$. Thus, in total $\det(A) = (-1)^s(u_1 \dots u_t)^{-1} \det(I) = (-1)^s(u_1 \dots u_t)^{-1}$.
\end{proof}

\begin{problem}
Compute the determinants of the following matrices using row reduction.
\end{problem}
\begin{proof}
First we consider $A$. Interchange the first and third rows and add the second row to the first.
\[
\left (
\begin{array}{ccc}
1 & 4 & 0\\
-2 & 0 & 2\\
5 & 4 & -6\\
\end{array}
\right )
\]
Add twice the first row to the second row and $-5$ times the first row to the third row.
\[
\left (
\begin{array}{ccc}
1 & 4 & 0\\
0 & 8 & 2\\
0 & -16 & -6
\end{array}
\right )
\]
Add twice the second row to the third row. Multiply the third row by $-1/2$. Add $-2$ times the third row to the second row.
\[
\left (
\begin{array}{ccc}
1 & 4 & 0\\
0 & 8 & 0\\
0 & 0 & 1
\end{array}
\right )
\]
Multiply the second row by $1/8$ and add $-4$ times this row to the first row.
\[
\left (
\begin{array}{ccc}
1 & 0 & 0\\
0 & 1 & 0\\
0 & 0 & 1
\end{array}
\right )
\]
In all, we've interchanged rows once ance multiplied by $-1/2$ and $1/8$. Thus, $\det(A) = 16$. Now consider $B$. Interchange the first and third rows. Add $-2$ times the first row to the second row and $-1$ times the first row to the third row.
\[
\left (
\begin{array}{cccc}
1 & 0 & 1 & -2\\
0 & -1 & 2 & -4\\
0 & 2 & -5 & 6\\
0 & 1 & -2 & 3
\end{array}
\right )
\]
Add the second row to the fourth row and two times the second row to the third row. Multiply the second row by $-1$.
\[
\left (
\begin{array}{cccc}
1 & 0 & 1 & -2\\
0 & 1 & -2 & 4\\
0 & 0 & -1 & -2\\
0 & 0 & 0 & -1
\end{array}
\right )
\]
Multiply the last row by $-1$, add twice this to the third row. Multiply the third row by $-1$. Add twice this to the second row and $-4$ times the last row to the second row. Add $-1$ times the third row and two times the fourth row to the first row.
\[
\left (
\begin{array}{cccc}
1 & 0 & 0 & 0\\
0 & 1 & 0 & 0\\
0 & 0 & 1 & 0\\
0 & 0 & 0 & 1
\end{array}
\right )
\]
In all, we've interchanged rows once and multiplied by $-1$ three times. Thus, $\det(B) = 1$.
\end{proof}

\end{document}