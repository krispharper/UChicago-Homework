\documentclass{article}
\usepackage{amsmath,amsthm,amssymb,amsfonts,fullpage,fancyhdr}

\pagestyle{fancy}
\renewcommand{\headheight}{50pt}
\renewcommand{\footskip}{10pt}
\renewcommand{\textheight}{609pt}
\renewcommand{\headrulewidth}{0pt}

\newtheorem{problem}{Problem}

\begin{document}

\rhead{Kris Harper\\MATH 25800\\February 5, 2010\\}
\chead{Homework 3\\}

\begin{problem}[9.1.4]
Prove that the ideals $(x)$ and $(x,y)$ are prime ideals in $\mathbb{Q}[x,y]$ but only the latter ideal is a maximal ideal.
\end{problem}
\begin{proof}
We've seen that $\mathbb{Q}[x,y]/(x,y) \cong \mathbb{Q}$ using the homomorphism $p(x,y) \in \mathbb{Q}[x,y]$ maps to its constant term and the First Isomorphism Theorem. Since $\mathbb{Q}$ is an integral domain, $(x,y)$ must be prime in $\mathbb{Q}[x]$. Furthermore, $\mathbb{Q}[x,y]/(x) = \mathbb{Q}[x][y]/(x) \cong (\mathbb{Q}[x]/(x))[y] \cong \mathbb{Q}[y]$. Since $\mathbb{Q}$ is a field, $\mathbb{Q}[y]$ is an integral domain which means $\mathbb{Q}[x,y]/(x)$ is an integral domain. It follows that $(x)$ must be prime in $\mathbb{Q}[x,y]$.

We see that $(x,y)$ is maximal in $\mathbb{Q}[x,y]$ since $\mathbb{Q}[x,y]/(x,y) \cong \mathbb{Q}$ and $\mathbb{Q}$ is a field. On the other hand, $(x)$ can't be maximal because $(x,y)$ is a proper ideal which contains it.
\end{proof}

\begin{problem}[9.1.6]
Prove that $(x,y)$ is not a principal ideal in $\mathbb{Q}[x,y]$.
\end{problem}
\begin{proof}
Suppose that $(x,y) = (a(x,y))$ for some polynomial $a(x,y)$. Note that for some polynomial $p(x,y)$, $a(x,y)p(x,y) = x$ and since degrees add when multiplying, we must have that the degree of $a(x,y)$ is either $0$ or $1$. But $a(x,y)$ can't be constant since there are no constant terms in $(x,y)$. Thus $a(x,y) = px + qy + r$ for some $p,q,r \in \mathbb{Q}$. We still have $a(x,y)p(x,y) = x$ and it follows that the degree of $p(x,y)$ must be $0$. It's easy to see that $r = 0$. But this forces $q = 0$ and $p(x,y) = 1/p$. Now it's impossible that $a(x,y)q(x,y) = y$ for some $q(x,y)$ since every term in this product will contain a factor of $x$. This is a contradiction and so $(x,y)$ can't be principal.
\end{proof}

\begin{problem}[9.2.2]
Let $F$ be a finite field of order $q$ and let $f(x)$ be a polynomial in $F[x]$ of degree $n \geq 1$. Prove that $F[x]/(f(x))$ has $q^n$ elements.
\end{problem}
\begin{proof}
Let $\overline{g(x)} \in F[x]/(f(x))$. If the degree of $g(x)$ is greater than or equal to $n$, then write $g(x) = f(x)q(x) + r(x)$ using the division algorithm in $F[x]$ where the degree of $r(x)$ is less than $n$. Then note that $\overline{g(x)} = \overline{r(x)}$ in $F[x]/(f(x))$ so every polynomial of $F[x]/(f(x))$ can be written as a polynomial of degree less than $n$. This shows that the polynomials $\overline{1}, \overline{x}, \dots , \overline{x^{n-1}}$ form a basis for the vector space $F[x]/(f(x))$ with coefficients from $F$. In particular, if $F$ has $q$ elements and every polynomial $\overline{g(x)}$ can be written as a linear combination of $\overline{1}, \overline{x}, \dots , \overline{x^{n-1}}$, then there are only $q^n$ distinct polynomials since there are $q$ choices for each coefficient and $n$ terms. This shows that $F[x]/(f(x))$ has $q^n$ elements.
\end{proof}

\begin{problem}[9.2.3]
\label{field}
Let $f(x)$ be a polynomial in $F[x]$. Prove that $F[x]/(f(x))$ is a field if and only if $f(x)$ is irreducible.
\end{problem}
\begin{proof}
Note that $f(x)$ being irreducible implies that $f(x)$ is prime since $F[x]$ is a Euclidean Domain and therefore a Principal Ideal Domain. But this also means that $(f(x))$ is prime and therefore maximal. It then follows that $F[x]/(f(x))$ is a field. Conversely, suppose that $F[x]/(f(x))$ is field. Then $(f(x))$ is maximal and thus prime which shows that $f(x)$ is prime and therefore irreducible.
\end{proof}

\begin{problem}[9.2.5]
Exhibit \emph{all} the ideals in ring $F[x]/(p(x))$, where $F$ is a field and $p(x)$ is a polynomial in $F[x]$ (describe them in terms of the factorization of $p(x)$).
\end{problem}
\begin{proof}
From the fourth Isomorphism Theorem, we know that there is a bijective correspondence between the ideals of $F[x]/(p(x))$ and the ideals of $F[x]$ which contain $(p(x))$. Furthermore, $F[x]$ is a Principal Ideal Domain so all ideals of $F[x]$ containing $(p(x))$ are of the form $(q(x))$ where $q(x) \mid p(x)$. But these are precisely the factors of $p(x)$. So all ideals of $F[x]/(p(x))$ are of the form $(q(x))/(p(x))$ where $q(x)$ is a factor of $p(x)$. In particular, if $p(x)$ is irreducible, then the only ideals of $F[x]/(p(x))$ are $(p(x))/(p(x)) = 0$ and $(1)/(p(x)) = F[x]/(p(x))$, so $F[x]/(p(x))$ is a field as in Problem~\ref{field}.
\end{proof}

\begin{problem}[9.3.4]
Let $R = \mathbb{Z} + x\mathbb{Q}[x] \subseteq \mathbb{Q}[x]$ be the set of polynomials in $x$ with rational coefficients whose constant term is an integer.\\
(a) Prove that $R$ is an integral domain and it's units are $\pm 1$.\\
(b) Show that the irreducibles in $R$ are $\pm p$ where $p$ is a prime in $\mathbb{Z}$ and the polynomials $f(x)$ that are irreducible in $\mathbb{Q}[x]$ and have constant term $\pm 1$. Prove that these irreducibles are prime in $R$.
(c) Show that $x$ cannot be written as a product of irreducibles in $R$ (in particular, $x$ is not irreducible) and conclude that $R$ is not a U.F.D.\\
(d) Show that $x$ is not a prime in $R$ and describe the quotient ring $R/(x)$.
\end{problem}
\begin{proof}
(a) Note that a subring of an integral domain is an integral domain since if two nonzero elements multiply to $0$ in the subring, they also multiply to $0$ in the ring. Since $\mathbb{Q}[x]$ is a Euclidean Domain, it suffices to prove that $R$ is a subring of $\mathbb{Q}[x]$. This is easily verified as the difference of two polynomials in $R$ will have as a constant term the difference of two integers, also an integer. Likewise, the product of these two polynomials will have as a constant term the product to two integers, also an integer. Thus $R$ is a subring of $\mathbb{Q}[x]$ and also an integral domain.

Additionally, suppose that $p(x)q(x) = 1$. Since degrees add under multiplication, we must have the degree of each $p(x)$ and $q(x)$ is $0$ so that they're both integers. But the only units in the integers are $\pm 1$. Thus, these are the only units in $R$.

(b) Suppose that $p = q(x)q'(x)$ for some prime $p \in \mathbb{Z}$. By the same argument as in part (a), $q(x)$ and $q'(x)$ are both constants which means they're both integers. Therefore $p$ is irreducible in $R$ since it's irreducible in $\mathbb{Z}$. Likewise, it follows that if $f(x)$ with constant term $1$ is irreducible in $\mathbb{Q}[x]$ then it's irreducible $R$. Now suppose $p(x)$ is any polynomial in $R$ which is not of this form. If $p(x)$ is constant, then it's some composite integer and so it factors in $R$ as it factors in $\mathbb{Z}$. Otherwise, suppose $p(x)$ is nonconstant and has a constant term $a \neq \pm 1$. Then $p(x) = aq(x)$ where $q(x)$ has a constant term of $\pm 1$ and coefficients $1/a$ times the coefficients of $p(x)$. Since $a \in \mathbb{Z}$ and is not a unit, $p(x)$ is reducible.

Suppose now that $p(x)$ is an irreducible in $R$ and $p(x) \mid a(x)b(x)$ with $a(x) = \sum_{i=1}^n a_ix^i$ and $b(x) = \sum_{i=1}^m b_ix^i$. Then there exists $c(x) \in R$ such that $p(x)c(x) = a(x)b(x)$. Suppose first that $p(x) = p$ a prime. Then $p$ divides every coefficient in the product $a(x)b(x)$. In particular, for each $0 \leq k \leq n+m$, $p$ divides $\left ( \sum_{i=0}^k a_ib_{k-i} \right ) x^k$ so $p$ divides each term in this sum. Note though that it must be the case that $p$ divides all the $a_i$ or all the $b_i$ because if it doesn't then one of these sums will contain the product $a_ib_j$ for two coefficients which $p$ doesn't divide. Since $p$ is prime in $\mathbb{Z}$, it must divide one of the two, a contradiction.

Now consider the case that $p(x)$ is an irreducible polynomial in $\mathbb{Q}$ with constant term $\pm 1$. Note that the constant terms of $a(x)$ and $b(x)$ must also be $\pm 1$. This then means that $p(x) \mid a(x)$ or $p(x) \mid b(x)$ so $p(x)$ is prime.

(c) This follows directly from part (b). The only irreducibles are primes $\pm p$ and $f(x)$ which has constant term $\pm 1$. A product of two primes will clearly not produce $x$, and a product of two polynomials with constant terms $\pm 1$ will still have a nonzero constant term. Moreover, a product of $p$ with $f(x)$ will also have a nonzero constant term $\pm p$. Since $x$ is not the product of any pair of irreducibles, it follows readily from induction that $x$ is not the product of any number. It follows that $R$ is not a U.F.D since we can factor $x$ as, for example $2 \cdot (1/2)x$ and $3 \cdot (1/3)x$ where none of the terms involved are not units since they aren't $\pm 1$.

(d) Note that $(x)$ is all the polynomials with rational coefficients which have no constant term and an integer coefficient for $x$. Then $(2/3x + (x))(3/2x + (x)) = x + (x) = 0$ are two zero divisors in $(x)$, so $x$ can't be prime as the quotient ring isn't an integral domain. The ring $R/(x)$ isn't an integral domain. Polynomials in $R$ are $0$ in the quotient ring if and only if they have no constant term and an integer coefficient for $x$.
\end{proof}

\end{document}