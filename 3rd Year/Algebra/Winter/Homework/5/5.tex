\documentclass{article}
\usepackage{amsmath,amsthm,amssymb,amsfonts,fullpage,fancyhdr}

\pagestyle{fancy}
\renewcommand{\headheight}{50pt}
\renewcommand{\footskip}{10pt}
\renewcommand{\textheight}{609pt}
\renewcommand{\headrulewidth}{0pt}

\newcommand{\tor}{\textup{Tor}}
\newcommand{\Hom}{\textup{Hom}}

\input xy
\xyoption{all}

\newtheorem{problem}{Problem}

\begin{document}

\rhead{Kris Harper\\MATH 25800\\February 19, 2010\\}
\chead{Homework 5\\}

\begin{problem}[10.3.1]
Prove that if $A$ and $B$ are sets of the same cardinality, then the free modules $F(A)$ and $F(B)$ are isomorphic.
\end{problem}
\begin{proof}
Since $A$ and $B$ have the same cardinality there exists a bijection $f : A \to B$. Let $\varphi : F(A) \to F(B)$ be given by $\varphi(r_1a_1 + \dots + r_na_n) = r_1f(a_1) + \dots + r_nf(a_n)$. Note that $\varphi$ is surjective since given the element $r_1b_1 + \dots + r_nb_n \in F(B)$ we know $r_1f^{-1}(b_1) + \dots + r_nf^{-1}(b_n)$ is mapped to it by $\varphi$. It's also injective in that given two distinct elements $r_1a_1 + \dots + r_na_n \neq r_1'a_1' + \dots + r_m'a_m'$ we see that there must exist some $i$ such that $r_ia_i \neq r_i'a_i'$. Then in $F(A)$ we have $\varphi(r_1a_1 + \dots + r_na_n) = r_1f(a_1) + \dots + r_nf(a_n)$ and since $f$ is injective, $r_if(a_i) \neq r_i'f(a_i')$. Therefore the images of the two elements are distinct and $\varphi$ is injective.

If $r_1a_1 + \dots + r_na_n$ and $r_1'a_1' + \dots + r_m'a_m'$ are two elements of $F(A)$ then
\begin{align*}
\varphi((r_1a_1 + \dots + r_na_n) + (r_1'a_1' + \dots + r_m'a_m'))
&= r_1f(a_1) + \dots + r_nf(a_n) + r_1'f(a_1') + \dots + r_m'f(a_m')\\
&= \varphi(r_1a_1 + \dots + r_na_n) + \varphi(r_1'a_1' + \dots + r_m'a_m')
\end{align*}
so $\varphi$ is additive. Finally, let $r \in R$ so we have
\begin{align*}
r\varphi(r_1a_1 + \dots + r_na_n)
&= r(r_1f(a_1) + \dots + r_nf(a_n))\\
&= rr_1f(a_1) + \dots + rr_nf(a_n)\\
&= \varphi(rr_1a_1 + \dots + rr_na_n)\\
&= \varphi(r(r_1a_1 + \dots + r_na_n))
\end{align*}
and $\varphi$ is scalar multiplicative. Therefore $\varphi$ is an $R$-module isomorphism between $F(A)$ and $F(B)$.
\end{proof}

\begin{problem}[10.3.4]
An $R$-module $M$ is called a \emph{torsion} module if for each $m \in M$ there is a nonzero element $r \in R$ such that $rm = 0$, where $r$ may depend on $m$ (i.e., $M = \tor(M)$ in the notation of Exercise 8 of Section 1). Prove that every finite abelian group is a torsion $\mathbb{Z}$-module. Give an example of an infinite abelian group that is a torsion $\mathbb{Z}$-module.
\end{problem}
\begin{proof}
Let $A$ be a finite abelian group of order $n$. Then for each $a \in A$ we have $na = 0$. Therefore $A$ is a torsion $\mathbb{Z}$ module. As an example of an infinite abelian group, consider $\mathbb{Q}/\mathbb{Z}$. Every element of this group has finite order ($a/b + \mathbb{Z}$ has order at most $b$), so for each element we can find an element of $\mathbb{Z}$ which sends it to $0$. Therefore $\mathbb{Q}/\mathbb{Z}$ is a torsion module.
\end{proof}

\begin{problem}[10.3.6]
Prove that if $M$ is a finitely generated $R$-module that is generated by $n$ elements then every quotient of $M$ may be generated by $n$ (or fewer) elements. Deduce that quotients of cyclic modules are cyclic.
\end{problem}
\begin{proof}
Suppose $M$ is generated by the set $A$ with $|A| = n$. Let $N$ be a submodule of $M$ and consider the projection map $\pi : M \to M/N$. Let $\overline{m} \in M/N$ and let $m' \in \pi^{-1}(\overline{m})$. Then $m' = r_1a_1 + \dots + r_1a_n$ and $\pi(m') = \overline{m}$. But then $\pi(m') = \pi(r_1a_1 + \dots + r_na_n) = r_1\pi(a_1) + \dots + r_n\pi(a_n)$. Thus, every element $\overline{m} \in M/N$ can be written as a finite linear combination of elements of the set $\pi(A)$ and $M/N$ is finitely generated. A cyclic module only has one generator and we've shown that a quotient of such a module will have one or fewer generators. Thus, it must also be cyclic.
\end{proof}

\begin{problem}[10.3.9]
An $R$-module $M$ is called \emph{irreducible} if $M \neq 0$ and if $0$ and $M$ are the only submodules of $M$. Show that $M$ is irreducible if and only if $M \neq 0$ and $M$ is a cyclic module with any nonzero element as a generator. Determine all the irreducible $\mathbb{Z}$-modules.
\end{problem}
\begin{proof}
Suppose that $M$ is irreducible and that $M$ requires at least two generators, $a \neq b$. Then $Ra \neq Rb$ (since $R$ has $1$). But note that $R\{a,b\}$ contains both $Ra$ and $Rb$ since it contains $a$ and $b$. Therefore $Ra$ is a nonzero submodule of $M$ which is properly contained in $M$, a contradiction. Therefore $M = Ra$ for some $a$. Conversely, suppose $M \neq 0$ and $M$ is cyclic with generator $a$. Suppose $N$ is a submodule of $M$. Note that $N \subseteq M$ so for each nonzero $n \in N$ we have $n = ra$ for some $r \in R$. Therefore $N$ contains $a$ and thus also contains $Ra$. But then $N = M$ and so $M$ is irreducible.

The $\mathbb{Z}$ modules are the same as abelian groups, so the irreducible $\mathbb{Z}$-modules are all finitely generated abelian groups with $1$ generator.
\end{proof}

\begin{problem}[10.4.11]
Let $\{e_1, e_2\}$ be a basis of $V = \mathbb{R}^2$. Show that the element $e_1 \otimes e_2 + e_2 \otimes e_1$ in $V \otimes_{\mathbb{R}} V$ cannot be written as a simple tensor $v \otimes w$ for any $v,w \in \mathbb{R}^2$.
\end{problem}
\begin{proof}
Given the basis elements $e_1$ and $e_2$ of $V$, we know $e_1 \otimes e_1$, $e_2 \otimes e_2$, $e_1 \otimes e_2$ and $e_2 \otimes e_1$ form a basis for $V \otimes_{\mathbb{R}} V$. Thus, $\{e_1 \otimes e_2, e_2 \otimes e_1\}$ is a linearly independent set which means $e_1 \otimes e_2 + e_2 \otimes e_1$ cannot equal a simple tensor $v \otimes w$.
\end{proof}

\begin{problem}[10.4.12]
Let $V$ be a vector space over the field $F$ and let $v$, $v'$ be nonzero elements of $V$. Prove that $v \otimes v' = v' \otimes v$ in $V \otimes_F V$ if and only if $v = av'$ for some $a \in F$.
\end{problem}
\begin{proof}
Suppose there exists $a \in F$ such that $v = av'$. Then $v \otimes v' = av' \otimes v' = v' \otimes av' = v' \otimes v$. Conversely, suppose $v \otimes v' = v' \otimes v$. Then $v \otimes v' - v' \otimes v = 0$. Since $v$ and $v'$ are nonzero, these two simple tensors are linearly dependent so there exists $a \in F$ such that $v \otimes v' = a(v' \otimes v) = av' \otimes v = v' \otimes av$. This is only possible if $v = av'$.
\end{proof}

\begin{problem}[10.5.14]
Let $\xymatrix{0 \ar[r] & L \ar[r]^{\psi} & M \ar[r]^{\varphi} & N \ar[r] & 0}$ be a sequence of $R$-modules.\\
(a) Prove that the associated sequence
\[
\xymatrix{
0 \ar[r] & \Hom_R(D,L) \ar[r]^{\psi'} & \Hom_R(D,M) \ar[r]^{\varphi'} & \Hom_R(D,N) \ar[r] & 0
}
\]
is a short exact sequence of abelian groups for all $R$-modules $D$ if and only if the original sequence is a split short exact sequence.\\
(b) Prove that the associated sequence
\[
\xymatrix{
0 \ar[r] & \Hom_R(N,D) \ar[r]^{\psi'} & \Hom_R(M,D) \ar[r]^{\varphi'} & \Hom_R(L,D) \ar[r] & 0
}
\]
is a short exact sequence of abelian groups for all $R$-modules $D$ if and only if the original sequence is a split short exact sequence.
\end{problem}
\begin{proof}
(a) Suppose the original sequence splits and let $D$ be an $R$-module. Note then that we can write $M \cong L \oplus N$. But now we know $\Hom_R(D, M) \cong \Hom_R(D, L \oplus N) \cong \Hom_R(D, L) \oplus \Hom_R(D, N)$. Then the associated sequence also splits and is an exact sequence. Conversely, suppose that the associated sequence is a short exact sequence. Let $D = N$ and let $f \in \Hom_R(N,N)$ be the identity. This lifts to some map $f' \in \Hom_R(N,M)$ such that $\varphi \circ f' = f$. But since $f$ is the identity on $N$, we see that $f'$ is a splitting homomorphism for $\varphi$. Thus the original sequence must be exact.

(b) If the original sequence is exact the associated sequence is a short exact sequence using the same proof as in part (a). Namely, using the fact that $\Hom_R(M,D) \cong \Hom_R(L \oplus N, D) \cong \Hom_R(L, D) \oplus \Hom_R(N, D)$. Conversely, if the associated sequence is exact, then let $D = L$ and let $f \in \Hom_R(L, L)$ be the identity. Then $f$ lifts into an element $f' \in \Hom_R(M, L)$ such that $f' \circ \psi = f$. But since $f$ is the identity on $L$ $f'$ is a splitting homomorphism for $\psi$ and the original sequence is short exact.
\end{proof}

\end{document}