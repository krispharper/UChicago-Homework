\documentclass{article}
\usepackage{amsmath,amssymb,amsfonts,amsthm,fullpage}

\newtheorem{**}{** Problem}
\newtheorem{problem}{Problem}
\newtheorem{lemma}{Lemma}

\newcommand{\slope}{\text{slope }}

\begin{document}
\begin{flushright}
Kris Harper\\

MATH 20800\\

February 23, 2009
\end{flushright}

\begin{center}
Homework 8
\end{center}

\begin{flushleft}

\begin{**}
Let
\[
f(x) =
\begin{cases}
2x^2 \sin \frac{1}{x} & x \neq 0\\
0 & x = 0.
\end{cases}
\]
Is $f$ increasing or decreasing on any neighborhood of the origin?
\end{**}

No.
\begin{proof}
The derivative of $f$ is $Df(x) = 4x \sin (1/x) - 2 \cos (1/x)$ for all $x \neq 0$. But there are infinitely many zeros of this function in any neighborhood of the origin. Thus, $Df$ is never strictly greater or less than $0$ for any neighborhood of the origin and therefore $f$ is neither increasing or decreasing for any neighborhood of the origin.
\end{proof}

\begin{**}
Let
\[
f(x) =
\begin{cases}
e^{-\frac{1}{x^2}} & x \neq 0\\
0 & x = 0.
\end{cases}
\]
Show $f$ is differentiable infinitely many times at $0$ and the $k$th derivative at $0$ is $0$ for all $k$.
\end{**}
\begin{proof}
Note that
\[
\lim_{x \rightarrow 0} e^{-\frac{1}{x^2}} = 0
\]
and so this function is continuous and thus differentiable at $x = 0$. For $x \neq 0$, using the chain rule we have
\[
Df(x) = \frac{a_1 e^{-\frac{1}{x^2}}}{x^3}
\]
where $a_1$ is some integer constant. Now suppose that the $k$th derivative for $x \neq 0$ is
\[
D^kf(x) = \frac{a_1 e^{-\frac{1}{x^2}}}{x^{(k+2)}} + \frac{a_2 e^{-\frac{1}{x^2}}}{x^{(k+4)}} + \dots + \frac{a_k e^{-\frac{1}{x^2}}}{x^{(k+2k)}}
\]
where $a_1, \dots , a_k$ are integer constants. Using the chain rule and the product rule, we can differentiate again to obtain
\[
D^{k+1}f(x) = \frac{a_1 e^{-\frac{1}{x^2}}}{x^{(k+3)}} + \frac{a_2 e^{-\frac{1}{x^2}}}{x^{(k+5)}} + \dots + \frac{a_k e^{-\frac{1}{x^2}}}{x^{(k+1+2(k))}} + \frac{a_{k+1} e^{-\frac{1}{x^2}}}{x^{(k+1+2(k+1))}}
\]
for all $x \neq 0$ where $a_1, \dots , a_k$ are different integer constants. Thus, by induction, the is the $k$th nonzero derivative. To show that each derivative is continuous at $0$, note that the first derivative for $x \neq 0$ is
\[
Df(x) = \frac{a_1 e^{-\frac{1}{x^2}}}{x^3}.
\]
Taking $\lim_{x \rightarrow 0} Df(x)$ we see that l'Hopital's Rule applies, and we end up with $\lim_{x \rightarrow 0} Df(x) = 0$. We can assume inductively that the $k$th derivative is continuous at $0$, and then use that fact and l'Hopital's Rule to show the $k+1$st derivative is continuous at $0$. Thus $D^kf(0)$ exists for all $k$ and $D^kf(0) = 0$ for all $k$.
\end{proof}

\begin{**}
If $\alpha \notin \mathbb{Q}$ and $r \leq 2$, show $f_r$ is not differentiable at $\alpha$.
\end{**}
\begin{proof}
Let $\alpha \notin \mathbb{Q}$ and let $r \leq 2$. Suppose that $Df_r(\alpha)$ exists. Then we have
\[
Df_r(\alpha) = \lim_{x \rightarrow \alpha} \left | \frac{f_r(x) - f_r(\alpha)}{x-\alpha} \right | = \lim_{x \rightarrow \alpha} \left | \frac{f_r(x)}{x-\alpha} \right |.
\]
For $x \notin \mathbb{Q}$ we have $f_r(x) = 0$, which means that $Df_r(\alpha) = 0$ on the irrationals. But also note that there are infinitely many $x \in \mathbb{Q}$ with $x = p/q$ such that $|\alpha - x| < 1/q^2 \leq |f_r(x)|$. Thus there are infinitely many $x \in \mathbb{Q}$ such that
\[
1 < \left | \frac{f_r(x)}{\alpha - x} \right |.
\]
But then $Df_r(\alpha)$ has different values on the rationals and irrationals. This is a contradiction and so $Df_r(\alpha)$ does not exist.
\end{proof}

\begin{**}
Let
\[
f_1 =
\begin{cases}
x & 0 \leq x \leq \frac{1}{2}\\
1-x & \frac{1}{2} \leq x \leq 1.
\end{cases}
\]
Extend $f_1(x) = f_1(x+1)$. Let $f_n(x) = \frac{1}{2} f_{n-1} (2x)$. Let
\[
f(x) = \sum_{n=1}^{\infty} f_n(x).
\]
Show that $f$ is continuous on $\mathbb{R}$, but is differentiable nowhere.
\end{**}
\begin{proof}
Each $f_n$ is piecewise linear and therefore continuous. Additionally, $(f_n)$ uniformly converges and so $f$ must be continuous. To see that $f$ is not differentiable, let $a \in \mathbb{R}$. Then consider
\[
\lim_{x \rightarrow a} \left |\frac{f(x) - f(a)}{x-a} \right | = \lim_{x \rightarrow a} \left |\frac{\sum_{n=1}^{\infty} f_n(x) - f_n(a)}{x-a} \right |.
\]
But between every two points in $\mathbb{R}$ there exists a ``spike'' of $f$ which separates the points. Thus, as $x$ approaches $a$, $f(x)$ and $f(a)$ will have opposite slopes, forcing the limit to not exist.
\end{proof}

\begin{**}
What is the differentiability of $f_r$ at $0$?
\end{**}
\begin{proof}
By definition we have $f_r(0) = 0$. Taking the difference quotient we have
\[
\lim_{h \rightarrow 0} \left | \frac{f_r(0 + h) - f_r(0) - Df_r(0)}{h} \right | = \lim_{h \rightarrow 0} \left | \frac{f_r(h) - Df_r(0)}{h} \right |.
\]
For $h \notin \mathbb{Q}$ we have $f_r(h) = 0$. For $h \in \mathbb{Q}$ we consider $\lim_{h \rightarrow 0} f_r(h)$. But this limit is $0$ since the denominator of $h$ will tend toward infinity as $h$ tends toward $0$. Thus, it must be the case that $Df_r(0) = 0$.
\end{proof}

\begin{**}
Let $\alpha$ be a real, algebraic number of degree $n \geq 2$. Then there exists a constant $c(\alpha)$ depending only on $\alpha$ such that for all $p/q \in \mathbb{Q}$ we have
\[
\left | \alpha - \frac{p}{q} \right | > \frac{c(\alpha)}{q^n}.
\]
\end{**}
\begin{proof}
Let $\alpha \in \mathbb{A}_{\mathbb{R}}$ such that $f(\alpha) = a_n \alpha^n + a_{n-1} \alpha^{n-1} + \dots + a_0 = 0$ where each $a_i \in \mathbb{Z}$. Then choose
\[
M > \max_{\alpha - 1 \leq x \leq \alpha + 1} |f'(x)|.
\]
Note that $M$ is entirely determined by $\alpha$. Suppose $p/q \in \mathbb{Q}$ such that $p/q \in (\alpha - 1 ; \alpha + 1)$ and $f(p/q) \neq 0$. We can do this because $n \geq 2$. Then we have
\[
\left | f \left (\frac{p}{q} \right ) \right | = \frac{|a_np^n + a_{n-1}p^{n-1}q + \dots + a_1pq^{n-1} + a_0 q|}{q^n} \geq \frac{1}{q^n}
\]
because the numerator in a nonzero integer. Then by the mean value theorem there exists $x$ such that
\[
\frac{1}{q^n} \leq \left |f \left ( \frac{p}{q} \right ) - f(\alpha) \right | = \left | \left ( \frac{p}{q} - \alpha \right ) f'(x) \right | < M \left |\frac{p}{q} - \alpha \right |.
\]
Taking $c(\alpha) = 1/M$ completes the proof.
\end{proof}

\begin{**}
Let $A \subseteq \mathbb{R}$ and suppose there exists a countable collection of intervals $\{I_i\}_{i \in \mathbb{N}}$ such that\\
1)
\[
A \subseteq \bigcup_i I_i
\]\\
2)
\[
\sum_i Vol(I_i)
\]
is finite.\\
3) If $a \in A$ then $a \in I_i$ for infinitely many $i \in \mathbb{N}$.\\
Show that this is equivalent to $A$ having measure $0$.
\end{**}
\begin{proof}
Suppose the above hypotheses are true for a set $A \subseteq \mathbb{R}$. Let $\varepsilon > 0$. If $\sum_i Vol (I_i) < \varepsilon$, we're finished. If $\sum_i Vol (I_i) \geq \varepsilon$ then we remove one interval, $I_{j_1}$ from the collection. Since $a \in A$ implies $a \in I_i$ for infinitely many $i$, it follows that $A \subseteq \bigcup_i I_i \backslash I_{j_1}$. Continue in this way until $\sum_i Vol(I_i) - \sum_k Vol(I_{j_k}) < \varepsilon$.\newline

Now let $A$ be a set with measure $0$ and let $\varepsilon > 0$. Suppose that there exists $a \in A$ such that $a \in I_i$ for only finitely many $i$, say $I_{j_1}, \dots , I_{j_n}$. But then this is a contradiction because $\varepsilon$ is finite and so we can simply choose a smaller $\varepsilon$ to exclude $a$. Thus $a$ must be in infinitely many intervals.
\end{proof}

\begin{**}
Show that, if $r \geq 2$, then $f_r$ is differentiable except on a set of measure $0$.
\end{**}
\begin{proof}
Use the method of ** Problem 9 to show that $f$ is not differentiable on a thick set, then $r < 2$.
\end{proof}

\begin{lemma}
Every thick subset of a complete metric space, $X$, is dense in $X$. If $X$ is the union of countably many closed sets, at least one has nonempty interior.
\end{lemma}
\begin{proof}
A thick subset of $X$ is a countable intersection $G = \bigcap G_n$ of open dense subsets of $X$. A thin subset of $X$ is a countable union $H = \bigcup H_n$ of closed nowhere dense subsets of $X$. If $X = \emptyset$ then we're finished, so assume $X \neq \emptyset$. Let $G$ be a thick subset of $X$ and let $p_0 \in X$ and $\varepsilon > 0$. Choose $r_n < 1/n$ and create a sequence $(p_n) \in X$ by
\[
B_{2r_1} (p_1) \subseteq B_{\varepsilon} (p_0)
\]
\[
B_{2r_2} (p_2) \subseteq B_{r_1} (p_1) \cap G_1
\]
\[
\dots
\]
\[
B_{2r_{n+1}} (p_{n+1}) \subseteq B_{r_n} (p_n) \cap G_1 \cap \dots \cap G_n.
\]
Since the $r_n \rightarrow 0$ as $n \rightarrow \infty$, $(p_n)$ is a Cauchy sequence in $X$, which converges to $p \in X$ since $X$ is complete. Since $p \in \overline{B_{r_n} (p_n)}$ we also have $p \in G_n$ and thus $p \in G \cap B_{\varepsilon} (p_0)$. Thus $G$ is dense in $X$.\newline

Now suppose that $X = H = \bigcup H_n$ with each $H$ a nowhere dense closed set. Then each $G_n = ^c H_n$ is open dense and
\[
G = \bigcap G_n = ^c \left ( \bigcup K_n \right ) = \emptyset
\]
which contradicts the density of $G$.
\end{proof}

\begin{lemma}
The set $PL$ of piecewise linear functions is dense in the set $C^0$ of all continuous linear functions from $\mathbb{R}$ to $\mathbb{R}$.
\end{lemma}
\begin{proof}
Let $f \in C^0$ and $\varepsilon > 0$. Since $[a,b]$ is compact, $f$ is uniformly continuous on this interval and so there exists $\delta > 0$ such that $|t-s| < \delta$ implies $|f(t) - f(s)| < \varepsilon$ for all $s,t \in [a,b]$. Choose $n > (b-a)/\delta$ and partition $[a,b]$ into $n$ subintervals $I_i = [x_i-1, x_i]$, each of length less than $\delta$. Now create a piecewise linear function $\phi : [a,b] \rightarrow \mathbb{R}$ which connects each of the points $(x_{i-1}, f(x_{i-1}))$ and $(x_i, f(x_i))$ for all $0 \leq i \leq n$. Then $\phi(t)$ for $t \in I_i$ is between $f(x_{i-1})$ and $f(x_i)$, each of which differ from $f(t)$ by less than $\varepsilon$. Thus for all $t \in [a,b]$ we have $|f(t) - \phi(t)| < \varepsilon$.
\end{proof}

\begin{lemma}
If $\phi \in PL$ and $\varepsilon > 0$, then there exists a sawtooth function $\sigma$ such that $||\sigma|| \leq \varepsilon$, $\sigma$ has period less than or equal to $\varepsilon$, and
\[
\min (|\slope \sigma|) > \max (|\slope \phi|) + \frac{1}{\varepsilon}.
\]
\end{lemma}
\begin{proof}
Choose $c \in \mathbb{R}$. The compressed sawtooth function $\sigma (x) = \varepsilon \sigma_0 (cx)$ has $||\sigma|| = \varepsilon$, period $1/c$, and slope $\pm \varepsilon c$. Choosing $c$ arbitrarily large will give the desired results.
\end{proof}

\begin{**}
Let $f : [a, b] \rightarrow \mathbb{R}$. If $f$ is monotonically increasing on $[a,b]$, then $f$ is differentiable almost everywhere on $[a,b]$.
\end{**}
\begin{proof}
We use the contrapositive. For $n \in \mathbb{N}$ define
\[
R_n = \{f \in C^0 \mid \forall x \in [a, b - \frac{1}{n}] \exists h > 0 \text{ such that } \left | \frac{f(x+h) - f(x)}{h} \right | > n\}
\]
\[
L_n = \{f \in C^0 \mid \forall x \in [a + \frac{1}{n}], b \exists h < 0 \text{ such that } \left | \frac{f(x+h) - f(x)}{h} \right | > n\}
\]
\[
G_n = \{f \in C^0 \mid \text{$f$ restricted to any interval of length $\frac{1}{n}$ is non-monotone}\}.
\]
We must show that each of these sets is open dense in $C^0$. Lemma 3 shows that the closure of $PL$ is $C^0$, so we must only show that the closure of each set contains $PL$. Let $\phi \in PL$ and $\varepsilon > 0$. By Lemma 3 there exists a sawtooth function $\sigma$ such that $||\sigma|| \leq \varepsilon$, $\sigma$ has period less than $1/n$ and
\[
\min (|\slope \sigma|) > \max (|\slope \phi|) + n.
\]
Consider $f = \phi + \sigma$ and note that $f \in PL$. It has slopes that are dominated by $\sigma$ and so they alternate in sign with period $1/2n$. At any $x \in [a, b + 1/n]$ there is a rightward slope either greater than $n$ or less than $-n$. Thus $f \in R_n$ and a similar argument shows $f \in L_n$. Since any interval of length $1/n$ contains a minimum or maximum of $\sigma$, it must contain a subinterval where $f$ is strictly increasing, and one where $f$ is strictly decreasing. Thus $f \in G_n$. Since $||f-\phi|| = \varepsilon$ we have $R_n$, $L_n$ and $G_n$ are dense in $C^0$.\newline

Now suppose that $f \in R_n$. The for all $x \in [a, b - 1/n]$ there exists $h > 0$ such that
\[
\left | \frac{f(x+h) - f(x)}{h} \right | > n.
\]
Since $f$ is continuous, there exists some neighborhood $A_x \subseteq [a,b]$ of $x$ and a constant $y > 0$ such that
\[
\left | \frac{f(t+h) - f(t)}{h} \right | > n +y
\]
for all $t \in A_x$. Since $[a, b - 1/n]$ is compact, we only need finitely many of these intervals, $A_{x_1}, \dots , A_{x_m}$, to cover it. Again, since $f$ is continuous for all $t \in A_{x_i}$ we have
\[
\left | \frac{f(t+h_i) - f(t)}{h_i} \right | > n + y_i.
\]
Since there are only finitely many inequalities we can replace $f$ by a function $g$ with the distance between $f$ and $g$ small enough so that we have
\[
\left | \frac{g(t+h_i) - g(t)}{h_i} \right | > n.
\]
Thus $g \in R_n$ and $R_n$ is open in $C^0$. A similar proof holds for $L_n$ open in $C^0$.\newline

Let $(f_k)$ be a sequence of functions in $^c G_n$ and such that $f_k$ uniformly converges to $f$. We must show that $f \in ^c G_n$. Each $f_k$ is monotone on some interval of $1/n$. Taking these intervals as a sequence, there is a subsequence which converges to some interval $I$. This interval must have length $1/n$ and by uniform convergence, $f$ is monotone on $I$. Thus $^cG_n$ is closed and therefore $G_n$ is open. Hence all of $R_n$, $L_n$ and $G_n$ are open dense in $C^0$.\newline

Finally, suppose that $f$ belongs to the thick set $\bigcup_n R_n \cap L_n \cap G_n$. Then for each $x \in [a,b]$ there exists a sequence of nonzero $h_n$ such that
\[
\left | \frac{f(x + h_n) - f(x)}{h_n} \right | < n.
\]
The numerator is at most $2 ||f||$ and so $h_n \rightarrow 0$ as $n \rightarrow \infty$. This shows that $f$ is not differentiable at $x$. Moreover, since $f \in G_n$, $f$ is not monotone on any interval of length $1/n$ and since any interval $I$ contains an interval of length $1/n$ for $n$ large enough, $f$ is not monotone on $I$.
\end{proof}

\end{flushleft}
\end{document}