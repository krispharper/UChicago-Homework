\documentclass{article}
\usepackage{amsmath,amssymb,amsfonts,amsthm,fullpage}

\newtheorem{**}{** Problem}
\newtheorem{problem}{Problem}
\newtheorem{lemma}{Lemma}

\begin{document}
\begin{flushright}
Kris Harper\\

MATH 20800\\

January 12, 2009
\end{flushright}

\begin{center}
Homework 2
\end{center}

\begin{flushleft}

\begin{**}
1) Suppose $U$ is an open set in a metric space $X$. Show that $\overline{U} \backslash U$ is nowhere dense in $X$.\\
2) Suppose $F$ is a closed subset of a metric space $X$. Show that $F \backslash F^{\circ}$ is nowhere dense in $X$.\\
3) Show that a countable union of sets of the first category is of of the first category.
\end{**}
\begin{proof}
1) Suppose to the contrary that $(\overline{U} \backslash U)^{\circ} \neq \emptyset$. Then since this set is open there exists $x \in (\overline{U} \backslash U)^{\circ}$ and some open set $V \subseteq (\overline{U} \backslash U)^{\circ}$ such that $x \in V$. Note that since $U$ is open, we must have $(\overline{U} \backslash U)^{\circ} \cap U = \emptyset$. But then since $x \notin U$, $x$ is an accumulation point of $U$ and so $V \cap U \neq \emptyset$. But this is a contradiction. Therefore $(\overline{U} \backslash U)^{\circ} = \emptyset$ and is thus nowhere dense in $X$.\newline

2) Suppose to the contrary that $(F \backslash F^{\circ})^{\circ} \neq \emptyset$. Suppose there exists some nonempty open subset $U \subseteq F \backslash F^{\circ}$. Then $U \cup F^{\circ} \subseteq F$ and $U \cup F^{\circ}$ is a larger open set than $F^{\circ}$ contained in $F$. This is a contradiction and so the largest open set contained in $F \backslash F^{\circ}$ must be $\emptyset$. Therefore $F \backslash F^{\circ}$ is nowhere dense.\newline

3) Consider some countable union of sets
\[
S = \bigcup_{n=1}^{\infty} A_n
\]
where
\[
A_n = \bigcup_{m=1}^{\infty} B_n
\]
where $B_n$ is nowhere dense. But then we simply have a countable union of countable unions of sets, which is countable (this follows from the same proof to show $\mathbb{Q}$ is countable). Thus $S$ is still a countable union of nowhere dense sets.
\end{proof}

\begin{**}
For any $\alpha$, with $0 < \alpha < 1$, show that $C_{\alpha}$ is a nowhere dense perfect set in $[0,1]$ with the usual metric.
\end{**}
\begin{proof}
Consider some point $x \in C_{\alpha}$ and an interval $(x - \varepsilon, x + \varepsilon)$. Note that $x$ must be an endpoint of a closed interval that remains after deleting some open interval. Then consider the step in the process of making $C_{\alpha}$ in which a closed interval is formed which is entirely contained in $(x - \varepsilon, x + \varepsilon)$. This must occur at some point since the remaining closed intervals are continually getting smaller. Then after the next step, there remains another endpoint of a closed interval which is in the neighborhood around $x$. This shows that $x$ is an accumulation point of $C_{\alpha}$. Since every point of $C_{\alpha}$ is an accumulation point, $\overline{C_{\alpha}} = C_{\alpha}$. We must then show that $^c C_{\alpha}$ is dense in $[0,1]$. Consider some point in $x \in C_{\alpha}$ and a neighborhood around $x$. This neighborhood must contain points from $^c C_{\alpha}$ by the same argument as above. Thus $x$ is an accumulation point of $^c C_{\alpha}$. But then $C_{\alpha} \cup ^c C_{\alpha} = [0,1]$ must be the closure of $^c C_{\alpha}$ since it is the whole space. Therefore $C_{\alpha}$ is nowhere dense.
\end{proof}

\begin{**}
1) Show that the irrational numbers are of the second category in $\mathbb{R}$ with the usual metric.\\
2) Determine if each of the following sets is of the first or second category in $\mathbb{R}^n$ with the usual metric:\\
i) $A = \{(x_1, x_2, \dots , x_n) \mid \text{$x_j \in \mathbb{Q}$ for $j = 1, 2, \dots , n$}\}$.\\
ii) $A = \{(x_1, x_2, \dots , x_n) \mid \text{at least $k$ of the coordinates are rational}\}$ for a fixed $k$.\\
iii) $A = \{(x_1, x_2, \dots , x_n) \mid \text{at least $1$ coordinate is irrational}\}$.
\end{**}
\begin{proof}
1) Since $\mathbb{R}$ is complete, it must be of the second category. Also, a point in $\mathbb{R}$ is nowhere dense since there are no isolated points. Then since $\mathbb{Q}$ is countable, it is a countable union of nowhere dense sets, and thus of the first category. But then $\mathbb{R} \backslash \mathbb{Q}$ must be of the second category, otherwise $\mathbb{R}$ would be a union of two sets of the first category.\newline

2) i) The same proof as in Part 1) is used to show that $\mathbb{Q}^n$ is of the first category in $\mathbb{R}^n$.\newline

ii) If $k=n$ or $k=0$ then the problem is reduced to parts i) or iii) respectively. Suppose that $0 < k < n$. Pick $k$ rational numbers and consider the set, $S$, of points where the $k$ rational coordinates correspond to those rational numbers and the other coordinates can vary. To see that this set is closed, we take a point $x \notin S$ and draw a ball centered at $x$ with radius less than the smallest difference between coordinates of $x$ and the $k$ fixed rational coordinates. Then the ball has empty intersection with $A$ and so $x$ is not an accumulation point. Now consider some nonempty open set $B \subseteq S$ and let $y \in B$. Then there exists a ball $B_r (y) \subseteq B$. But then it is certainly possible to choose a point in $\mathbb{R}^n$ which is in $B_r (y)$ but not in $B$ by varying one of the rational coordinates of $y$ by a small enough amount. This is a contradiction and so $B$ must be empty. This shows that $S$ is nowhere dense. But since $\mathbb{Q}^k$ is countable, we must have a countable union of nowhere dense sets and so the set of the first category.\newline

iii) Note that this is the complement of the set in part i). Since that set is of the first category, then this set cannot be, otherwise $\mathbb{R}^n$ would be of the first category, which is impossible since $\mathbb{R}^n$ is complete.
\end{proof}

\begin{lemma}
Let $A \subseteq X$. If $^c A$ is of the first category, then $A$ is dense in $X$.
\end{lemma}
\begin{proof}
Since $^c A$ is of the first category, then $^c A = \bigcup_{n=1}^{\infty} V_n \subseteq \bigcup_{n=1}^{\infty} \overline{V_n}$ where $V_n$ is nowhere dense. But then we must have $A \supseteq \bigcap_{n=1}^{\infty} {^c \overline{V_n}}$. Since $V_n$ is nowhere dense in $X$, each $^c \overline{V_n}$ is open and dense in $X$ and by the Baire Category Theorem, their intersection is dense. Therefore, $A$ is dense in $X$.
\end{proof}

\begin{**}
Let $X$ be a complete metric space and $(F_n)$ a countable collection of closed sets in $X$ such that $X = \bigcup_{n=1}^{\infty} F_n$. Show that $U = \bigcup_{n=1}^{\infty} F_n^{\circ}$ is dense.
\end{**}
\begin{proof}
Define $A_n = F_n \backslash F_n^{\circ}$. Problem 1 Part 2) shows that $A_n$ is nowhere dense. Then define $A = \bigcup_{n=1}^{\infty} A_n$. Note that $^c U = X \backslash U \subseteq A$. But then the previous lemma shows that $U$ is dense in $X$.
\end{proof}

\begin{**}
Graph $x^3 + 3x^2 + 2x = y^2 + y$.
\end{**}

\vspace{5 cm}

\begin{**}
Consider $\mathcal{C} ([a,b], \mathbb{C})$ where
\[
||f|| = \left ( \int_a^b |f(x)|^p dx \right )^{\frac{1}{p}}
\]
with $1 \leq p < \infty$. Show that this is a normed linear space.
\end{**}
\begin{proof}
Let $f \in \mathcal{C}([a,b], \mathbb{C})$. It is clear that $||f|| \geq 0$ since $|f(x)| \geq 0$ for all $x \in [a,b]$. Moreover, if $f = 0$ then $|f(x)| = 0$ for all $x \in [a,b]$ and then $||f||=0$. Suppose that $||f|| = 0$, but that $f \neq 0$. Then consider some $x \in [a,b]$ such that $|f(x)| > 0$. Since $f$ is continuous, there exists some neighborhood around $x$, $(x - \varepsilon, x + \varepsilon)$ such that $|f| > 0$ for all points in that neighborhood. But then
\[
\int_a^{x - \varepsilon} |f(x)|^p dx + \int_{x - \varepsilon}^{x + \varepsilon} |f(x)|^p dx + \int_{x + \varepsilon}^b |f(x)|^p dx = \int_a^b |f(x)^p dx.
\]
All terms are greater than or equal to $0$, but the middle term on the left is strictly greater than $0$ and the term on the right is equal to $0$, we have a contradiction. Therefore $f = 0$.\newline

Consider some $\alpha \in \mathbb{R}$. Then
\[
||\alpha f|| = \left ( \int_a^b |\alpha f(x)|^p dx \right)^{\frac{1}{p}} = \left ( \int_a^b |\alpha|^p |f(x)|^p dx \right )^{\frac{1}{p}} = \left ( |\alpha|^p \int_a^b |f(x)|^p dx \right )^{\frac{1}{p}} = |\alpha| \left ( \int_a^b |f(x)|^p dx \right )^{\frac{1}{p}} = |\alpha| ||f||.
\]\newline

We have shown earlier that for $x,y \in \mathbb{R}$ with $a,b > 0$ and $1 \leq p,q \leq \infty$ with $1/p + 1/q = 1$ we have
\[
xy \leq \frac{x^p}{p} + \frac{y^q}{q}.
\]
Then note that
\[
\frac{||fg||_1}{||f||_p \cdot ||g||_q} = \int_a^b \frac{|f(x)|}{||f||_p} \frac{|g(x)|}{||g||_q} dx \leq \frac{1}{p} \int_a^b \left ( \frac{|f(x)|}{||f||_p} \right )^p dx + \frac{1}{q} \int_a^b \left ( \frac{|g(x)|}{||g||_q} \right )^q dx = \frac{1}{p} + \frac{1}{q} = 1.
\]
This proves the H\"{o}lder inequality for integrals. The proof of the triangle inequality follows in the same way as for vectors in $\mathbb{R}^n$. Since all three conditions are met, $\mathcal{C} ([a,b], \mathbb{C})$ is a normed linear space.
\end{proof}

\begin{**}
Which of the following spaces are finite dimensional?\\
1) $(\mathbb{R}^n, ||\cdot||_2)$\\
2) $(\mathbb{C}^n, ||\cdot||_2)$\\
3) $(F^n, ||\cdot||_p)$, $1 \leq p \leq \infty$.\\
4) $\mathcal{BC}(X, F)$ with $||f|| = \sup_{x \in X} |f(x)|$.\\
5) $\mathcal{C}([a,b], \mathbb{C})$ with $||f|| = \left ( \int_a^b |f(x)|^p dx \right)^{\frac{1}{p}}$, $1 \leq p \leq < \infty$.
\end{**}
\begin{proof}
1) This is finite dimensional. The set $\{\mathbf{e}_j \mid 1 \leq j \leq n\}$, where $\mathbf{e}_j$ is the vector with a $1$ in the $j$th coordinate and $0$s elsewhere, acts as a basis.\newline

2) The same basis as in Part 1) works to show this is finite dimensional.\newline

3) The same basis as in Part 1) works to show this is finite dimensional.\newline

4) If $X$ has finitely many points, then $\mathcal{BC}(X, F)$ is finite dimensional. The finite set of functions which map one point of $X$ to $1$ and the rest to $0$ are linearly independent and span the space. If $X$ is infinite then $\mathcal{BC} (X, F)$ is infinite dimensional.

5) Consider the polynomials on $[a,b]$. A basis for this space is $\{1, x, x^2, \dots\}$. Note that since polynomials can be of arbitrary degree on this interval, an infinite basis must be needed. But since the polynomials are a subset of all continuous functions on $[a,b]$, it must be the case that $\mathcal{C} ([a,b], \mathbb{C})$ is infinite dimensional.
\end{proof}

\begin{**}
A Banach space is either finite dimensional or uncountably dimensional.
\end{**}
\begin{proof}
Let $V$ be a countably infinite dimensional Banach space over some field, $F$. Let $\{v_1, v_2, v_3, \dots\}$ be a basis for $V$. By the Baire Category Theorem, $V$ is of the second category because it is complete. Then note that the span $\langle \{v_1, v_2, v_3, \dots \rangle$, for some $n$, is nowhere dense in $V$. This is because this is a finite dimensional proper subspace of a normed space. But then since the basis spans $V$, we have $V = \bigcup_{n=1}^{\infty} \langle \{v_1, v_2, v_3, \dots\} \rangle$ which shows that $V$ is of the first category, which is a contradiction.
\end{proof}

\end{flushleft}
\end{document}