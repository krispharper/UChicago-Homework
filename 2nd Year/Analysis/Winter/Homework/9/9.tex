\documentclass{article}
\usepackage{amsmath,amssymb,amsfonts,amsthm,fullpage}

\newtheorem{**}{** Problem}
\newtheorem{problem}{Problem}
\newtheorem{lemma}{Lemma}

\begin{document}
\begin{flushright}
Kris Harper\\

MATH 20800\\

March 2, 2009
\end{flushright}

\begin{center}
Homework 9
\end{center}

\begin{flushleft}

\begin{**}
Let $f : \mathbb{R}^n \rightarrow \mathbb{R}^m$. Then
\[
f((x_1, x_2, \dots , x_n)) = (f_1((x_1, x_2, \dots , x_n)), f_2((x_1, x_2, \dots , x_n)), \dots , f_m((x_1, x_2, \dots , x_n))).
\]
If $f_k : \mathbb{R}^n \rightarrow \mathbb{R}$ is differentiable for all $1 \leq k \leq m$, then $f$ is differentiable.
\end{**}
\begin{proof}
Since $f_k$ is differentiable for all $1 \leq k \leq m$, there exists a linear transformation $L_k$ such that for all $x \in \mathbb{R}^n$ we have
\[
\lim_{h \rightarrow 0} \frac{|f_k(x + h) - f_k(x) - L_kh|}{|h|} = \lim_{h \rightarrow 0} \frac{|f_k((x_1 + h_1, x_2 + h_2, \dots , x_n + h_n)) - f_k((x_1, x_2, \dots , x_n)) - L_kh|}{|h|} = 0
\]
But then we must have
\begin{align*}
0
&= \lim_{h \rightarrow 0} \frac{|(f_1((x_1+h_1, \dots , x_n+h_n))-f_1((x_1, \dots , x_n)), \dots , f_m((x_1+h_1, \dots , x_n+h_n)) - f_m((x_1, \dots , x_n)))|}{|h|}\\
&= \lim_{h \rightarrow 0} \frac{|(f_1((x_1+h_1, \dots , x_n+h_n)), \dots , x_n+h_n)), \dots , f_m((x_1+h_1, \dots , x_n+h_n)))|}{|h|}\\
&- \frac{(f_1((x_1, \dots , x_n)), \dots , f_m((x_1, \dots , x_n))) - Lh|}{|h|}\\
&= \lim_{h \rightarrow 0} \frac{|f(x + h) - f(x) - Lh|}{|h|}
\end{align*}
\end{proof}

\begin{**}
Show that if $f : U \rightarrow \mathbb{R}$, $U \subseteq \mathbb{R}^n$ is differentiable at $x \in U$ then $D_v f(x) = \nabla f(x) \cdot v$.
\end{**}
\begin{proof}
We have
\[
\nabla f(x) \cdot v = \sum_{i=1}^n D_if(x)v_i = \sum_{i=1}^n \lim_{t \rightarrow 0} \frac{f(x + te_i) - f(x)}{t} v_i = \lim_{t \rightarrow 0} \frac{f(x+tv) - f(x)}{t} = D_vf(x).
\]
\end{proof}

\begin{**}
Relative to the standard basis, we can represent $Df(a)$ by the $m \times n$ matrix $[D_j f_i (a)]$ where $j = 1, \dots , n$ and $i = 1, \dots , m$.
\end{**}
\begin{proof}
We already know that in one variable, $f : \mathbb{R}^n \rightarrow \mathbb{R}$ we have
\[
Df(x) = (D_1 f(x), \dots , D_n f(x)).
\]
This immediately extends to $m$ dimensions if $f : \mathbb{R}^n \rightarrow \mathbb{R}^m$. We know each component function, $f_i : \mathbb{R}^n \rightarrow \mathbb{R}$ with $i = 1, \dots m$, is differentiable. Then the $i$th row of $f'(x)$ is $f_i'(x)$.
\end{proof}

\begin{**}
Let $U$ be an open set in $\mathbb{R}^n$ and $f : \mathbb{R}^n \rightarrow \mathbb{R}$ such that $f$ is differentiable on $U$. Suppose $x,y \in U$ and the line segment
\[
L = \{(1-t)x + ty \mid 0 \leq t \leq 1\} \subseteq U.
\]
Then there exists $z \in L$ such that $f(y) - f(x) = Df(z) (y-x)$.
\end{**}
\begin{proof}
Let $F(t) = f((1-t)x + ty)$ for $0 \leq t \leq 1$. Then by the Mean Value Theorem there exists $s \in [0,1]$ such that $F'(s) = f(x) - f(y)$. Then by the Chain Rule note that $F'(s) = f'((1-s)x + sy) (x-y)$. Taking $z = (1-s)x + sy$ gives the desired result.
\end{proof}

\begin{**}
Let
\[
f(x,y) =
\begin{cases}
0 & \text{if $(x,y) = (0,0)$}\\
\frac{x^3y + xy^3}{x^2 + y^2} & \text{if $(x,y) \neq (0,0)$}.
\end{cases}
\]
Are $D_1 f(x,y)$ and $D_2 f(x,y)$ continuous at $(0,0)$?
\end{**}

Yes.
\begin{proof}
We have
\[
D_1f(x,y) = \frac{y(x^4 + 4x^2y^2-y^4)}{(x^2 + y^2)^2}.
\]
Since the power on the numerator always exceeds that of the denominator, we must have
\[
\lim_{(x,y) \rightarrow (0,0)} \frac{y(x^4 + 4x^2y^2-y^4)}{(x^2 + y^2)^2} = 0.
\]
A similar proof holds for $D_2f(x,y)$.
\end{proof}

\begin{**}
Let $f : \mathbb{R}^2 \rightarrow \mathbb{R}$ be the function defined in ** Problem 5. Is $D_2(D_1f)(0,0) = D_1(D_2f)(0,0)$?
\end{**}

No.
\begin{proof}
We have
\[
D_1f(x,y) = \frac{y(x^4 + 4x^2y^2 - y^4}{(x^2 + y^2)^2}
\]
and
\[
D_2f(x,y) = \frac{x(x^4 - 4x^2y^2 - y^4}{(x^2 + y^2)^2}.
\]
Note that $D_2f(x,0) = x$ for all $x$ and $D_1f(0,y) = -y$ for all $y$. Now $D_2(D_1f)(0,0) = D(D_1f(0,y)) = D(-y) = -1$ and $D_1(D_2f)(0,0) = D(D_2f(x,0)) = D(x) = 1$.
\end{proof}

\begin{**}
Take $f : U \rightarrow \mathbb{R}$ where $U \subseteq \mathbb{R}^n$ such that $f$ is differentiable and $D_j (D_i f)(x)$ exists for all $x \in U$. If $D_i (D_j f)$ is continuous for all $i,j$, then $D_i(D_jf) = D_j(D_if)$.
\end{**}
\begin{proof}
Let $x \in U$. Consider the function
\begin{align*}
F_{ij}(h)
&= (f(x_1, \dots , x_i+h, \dots , x_j+h, \dots , x_n) - f(x_1, \dots , x_i + h, \dots , x_j, \dots , x_n))\\
&- (f(x_1, \dots , x_i, \dots, x_j + h, \dots , x_n) + f(x_1, \dots , x_n))
\end{align*}
and let
\[
g(y) = f(x_1, \dots , y, \dots , x_j + h, \dots , x_n) - f(x_1, \dots , y, \dots , x_j, \dots , x_n)
\]
then
\[
F_{ij}(h) = g(x_i + h) - g(x_i).
\]
By the Mean Value Theorem there exists $c \in [x_i, x_i+h]$ such that
\[
g(x_i + h) - g(x_i) = g'(c) h = h (D_if(x_1, \dots , c, \dots, x_j + h, \dots , x_n) - D_1f(x_1, \dots , c, \dots , x_j, \dots , x_n)).
\]
Now use the Mean Value Theorem again on $D_if$ so that there exists $d \in [x_j, x_j+h]$ such that
\[
D_if(x_1, \dots , c, \dots, x_j + h, \dots , x_n) - D_if(x_1, \dots , c, \dots , x_j, \dots , x_n) = D_{ij} (x_1, \dots , c, \dots , d, \dots , x_n)h.
\]
Now we have
\[
F_{ij}(h) = h^2 D_{ij}(x_1, \dots , c, \dots , d, \dots , x_n).
\]
Note that as $h \rightarrow 0$ we have $c \rightarrow x_i$ and $d \rightarrow x_j$, so by the continuity of $D_{ij} f$ we have
\[
\lim_{h \rightarrow 0} \frac{F_{ij}(h)}{h^2} = \lim_{c,d \rightarrow 0,0} D_{ij}f(x_1, \dots , c, \dots , d, \dots , x_n) = D_{ij}f(x_1, \dots , x_i, \dots , x_j, \dots , x_n).
\]
But then it's clear that $F_{ij} = F_{ji}$ which results in $D_{ij}f = D_{ji}f$.
\end{proof}

\begin{problem}
Find $f'$ for the following:\\
1) $f(x,y,z) = x^y$\\
2) $f(x,y,z) = (x^y, z)$\\
3) $f(x,y) = \sin (x \sin y)$\\
4) $f(x,y,z) = \sin (x \sin (y \sin z))$\\
5) $f(x,y,z) = x^{y^z}$\\
6) $f(x,y,z) = x^{y+z}$\\
7) $f(x,y,z) = (x+y)^z$\\
8) $f(x,y) = \sin (xy)$\\
9) $f(x,y) = (\sin xy)^{\cos 3}$\\
10) $f(x,y) = (\sin xy, \sin (x \sin y), x^y)$.
\end{problem}
\begin{proof}
1)
\[
\left (
\begin{array}{ccc}
yx^{y-1} & e^{y \ln x} \ln x & 0
\end{array}
\right )
\]
2)
\[
\left (
\begin{array}{ccc}
yx^{y-1} & e^{y \ln x} \ln x & 0\\
0 & 0 & 1
\end{array}
\right )
\]
3)
\[
\left (
\begin{array}{cc}
\cos (x \sin y) \sin y & \cos (x \sin y) x \cos y
\end{array}
\right )
\]
4)
\[
\left (
\begin{array}{ccc}
\cos(x \sin (y \sin z)) \sin (y \sin z)) & \cos (x \sin (y \sin z)) \cos (y \sin z) \sin z & \cos (x \sin (y \sin z)) \cos (y \sin z) y \cos z
\end{array}
\right )
\]
5)
\[
\left (
\begin{array}{ccc}
y^z x^{y^z - 1} & e^{y^z \ln x} zy^{z-1} \ln x & e^{e^{z \ln y} \ln x} \left (\frac{1}{x} e^{z \ln y} + \ln x e^{z \ln y} \ln y \right )
\end{array}
\right )
\]
6)
\[
\left (
\begin{array}{ccc}
(y+z)x^{y+z-1} & e^{(y+z) \ln x} \ln x & e^{(y+z) \ln x} \ln x
\end{array}
\right )
\]
7)
\[
\left (
\begin{array}{ccc}
z(x+y)^{z-1} & z(x+y)^{z-1} & e^{z \ln (x+y)} \ln (x+y)
\end{array}
\right )
\]
8)
\[
\left (
\begin{array}{cc}
\cos(xy)y & \cos(xy)x
\end{array}
\right )
\]
9)
\[
\left (
\begin{array}{cc}
\cos(3) \sin(xy)^{\cos(3) - 1} y \cos(xy) & \cos(3) \sin(xy)^{\cos(3) - 1} x \cos(xy)
\end{array}
\right )
\]
10)
\[
\left (
\begin{array}{cc}
\cos(xy)y & \cos(xy)x\\
\cos (x \sin y) \sin y & \cos (x \sin y) x \cos y\\
yx^{y-1} & e^{y \ln x} \ln x
\end{array}
\right )
\]
\end{proof}

\begin{problem}
Find $f'$ for the following where $g : \mathbb{R} \rightarrow \mathbb{R}$ is continuous:\\
1) $f(x,y) = \int_a^{x+y} g$\\
2) $f(x,y) = \int_a^{xy} g$\\
3) $f(x,y,z) = \int_{xy}^{\sin(x \sin (y \sin z))} g$.
\end{problem}
\begin{proof}
1)
\[
\left (
\begin{array}{cc}
g(x+y) & g(x+y)
\end{array}
\right )
\]
2)
\[
\left (
\begin{array}{cc}
g(xy) y & g(xy) x
\end{array}
\right )
\]
3)
\[
g(\sin (x \sin (y \sin z))) Dh_1 (x,y,z) - g(x^y) Dh_2 (x,y,z)
\]
where $h_1 = \sin (x \sin (y \sin z))$ and $h_2 = x^y$ have solutions in Parts 1) and 4) of Problem 1.
\end{proof}

\begin{problem}
A function $f : \mathbb{R}^n \times \mathbb{R}^n \rightarrow \mathbb{R}^p$ is bilinear if for $x, x_1, x_2 \in \mathbb{R}^n$, $y, y_1, y_2 \in \mathbb{R}^m$ and $a \in \mathbb{R}$ we have
\[
f(ax, y) = a f(x, y) = f(x, ay),
\]
\[
f(x_1 + x_2, y) = f(x_1, y) + f(x_2, y),
\]
\[
f(x, y_1 + y_2) = f(x, y_1) + f(x, y_2).
\]
1) Prove that if $f$ is bilinear, then
\[
\lim_{(h,k) \rightarrow 0} \frac{|f(h,k)|}{|(h,k)|} = 0.
\]\\
2) Prove that $Df(a,b)(x,y) = f(a,y) + f(x,b)$.\\
3) Show that $Dp(a,b)(x,y) = bx + ay$ where $p : \mathbb{R}^2 \rightarrow \mathbb{R}$ is defined by $p(x,y) = xy$ is a special case of Part 2).
\end{problem}
\begin{proof}
Note that
\[
f(h,k) = \sum_{i=1}^n \sum_{j=1}^m h_i k_j f (e_i, e_j)
\]
and so this function is linear. Thus there exists some $M > 0$ such that
\[
|f(h,k)| \leq M \max (|h_i|) \max (|k_j|) \leq M |h| |k|.
\]
Since $|(h,k)| = \sqrt{|h|^2 + |k|^2}$, we need only show the result is true when $n = m = 1$ and $f$ is simply $p$, mentioned in Part 3). But this has already been shown to be true.\newline

2) We have
\[
\lim_{(h,k) \rightarrow 0} \frac{|f(a+h, b+k) - f(a,b) - f(a,k) - f(h,b)|}{|(h,k)|} = \lim_{(h,k) \rightarrow 0} \frac{|f(h,k)|}{|(h,k)|} = 0
\]
by bilinearity and Part 1).\newline

3) Taking $n=m=p=1$ we have $f : \mathbb{R}^2 \rightarrow \mathbb{R}$. If $f(x,y) = xy$ then from Part 2) we have $Df(a,b)(x,y) = f(a,y) + f(b,x) = bx + ay$.
\end{proof}

\begin{problem}
Define $IP : \mathbb{R}^n \times \mathbb{R}^n \rightarrow \mathbb{R}$ by $IP(x,y) = \langle x, y \rangle$.\\
1) Find $D(IP)(a,b)$ and $(IP)'(a,b)$.\\
2) If $f,g : \mathbb{R}^n \rightarrow \mathbb{R}$ are differentiable and $h : \mathbb{R} \rightarrow \mathbb{R}$ is defined by $h(t) = \langle f(t), g(t) \rangle$, show that
\[
h'(a) = \langle f'(a)^T, g(a) \rangle + \langle f(a), g'(a)^T \rangle.
\]\\
3) If $f : \mathbb{R} \rightarrow \mathbb{R}^n$ is differentiable and $|f(t)| = 1$ for all $t$, show that $\langle f'(t)^T, f(t) \rangle = 0$.\\
4) Exhibit a differentiable function $f: \mathbb{R} \rightarrow \mathbb{R}$ such that the function $|f|$ defined by $|f|(t) = |f(t)|$ is not differentiable.
\end{problem}
\begin{proof}
1) Since $IP$ is bilinear, we have $D(IP)(a,b)(x,y) = \langle b,x \rangle + \langle a,y \rangle$. Then $(IP)'(a,b) = (a,b)$\newline

2) Note that $h(t) = (IP) \circ (f,g)$. Now we simply use the Chain Rule and Part 1) to obtain the result.\newline

3) This is just Part 2) applied to $\langle f(t), f(t) \rangle = 1$. Differentiating both sides gives the desired result.\newline

4) Take $f(t) = t$. Then $|f(t)|$ is not differentiable at $0$.
\end{proof}

\begin{problem}
Let $E_i$ with $i = 1, \dots k$ be Euclidean spaces of various dimensions. A function $f : E_1 \times \dots \times E_k \rightarrow \mathbb{R}^p$ is called multilinear if for each choice of $x_j \in E_j$, $j \neq i$ the function $g : E_i \rightarrow \mathbb{R}^p$ defined by $g(x) = f(x_1, \dots , x_{i-1}, x, x_{i+1}, \dots , x_k)$ is a linear transformation.\\
1) If $f$ is multilinear and $i \neq j$, show that for $h = (h_1, \dots h_k)$, with $h_l \in E_l$ we have
\[
\lim_{h \rightarrow 0} \frac{|f(a_1, \dots , h_i, \dots , h_j, \dots , a_k)|}{|h|} = 0.
\]\\
2) Prove that
\[
Df(a_1, \dots a_k)(x_1, \dots x_k) = \sum_{i=1}^k f(a_1, \dots , a_{i-1}, x, a_{i+1}, \dots , a_k).
\]
\end{problem}
\begin{proof}
1) Since $f(a_1, \dots , h_i, \dots , h_j, \dots , a_k)$ is bilinear, this is an immediate result of Part 2) of Problem 3.\newline

2) This is a similar case to Part 3) of Problem 3. Using the definition of a derivative, we can expand the numerator in a similar fashion as in Part 3) of Problem 3. Then using Part 1) we obtain a similar result, with more terms. This limit finally goes to $0$ for the same reasons as in Part 3) of Problem 3.
\end{proof}

\begin{problem}
Regard and $n \times n$ matrix as a point in the $n$-fold product $\mathbb{R}^n \times \dots \times \mathbb{R}^n$ by considering each row as a member of $\mathbb{R}^n$.\\
1) Prove that $\det : \mathbb{R}^n \times \dots \times \mathbb{R}^n \rightarrow \mathbb{R}$ is differentiable and
\[
D(\det)(a_1, \dots , a_n)(x_1, \dots , x_n) = \sum_{i=1}^{n} \det (a_1, \dots , x_i, \dots , a_n)^T.
\]\\
2) If $a_{ij} : \mathbb{R} \rightarrow \mathbb{R}$ are differentiable and $f(t) = \det (a_{ij}(t))$, show that
\[
f'(t) = \sum_{j=1}^n \det
\left (
\begin{array}{ccc}
a_{11}(t), & \dots & , a_{1n}(t)\\
\vdots & & \vdots\\
a_{j1}'(t), & \dots & , a_{jn}'(t)\\
\vdots & & \vdots\\
a_{n1}(t), & \dots & , a_{nn}(t)
\end{array}
\right ).
\]\\
3) If $\det (a_{ij}(t)) \neq 0$ for all $t$ and $b_1, \dots , b_n : \mathbb{R} \rightarrow \mathbb{R}$ are differentiable, let $s_1, \dots , s_n : \mathbb{R} \rightarrow \mathbb{R}$ be the functions such that $s_1 (t), \dots , s_n(t)$ are the solutions of the equations
\[
\sum_{j=1}^n a_{ji}(t)s_j(t) = b_i(t)
\]
for $i = 1, \dots , n$. Show that $s_i$ is differentiable and find $s_i'(t)$.
\end{problem}
\begin{proof}
1) Since $\det$ is multilinear, this follows immediately from Problem 5, Part 2).\newline

2) This is a direct consequence of Part 1) and the Chain Rule.\newline

3) Using Cramer's Rule, we can write $s_i = \det (B_i) / \det (A)$ where $A = [a_{ij}(t)]$ and $B_i$ is the matrix obtained by replacing the $i$th column of $A$ with $(b_1(t), \dots , b_n(t))^T$. Taking the transpose of these matrices doesn't change the determinant, which allows us to use Part 2) and the quotient rule to find
\[
s_i'(t) = \frac{\det(B_i) \det'(A) - \det(A)\det'(B_i)}{\det^2(B_i)}.
\]
\end{proof}

\begin{problem}
Suppose $f : \mathbb{R}^n \rightarrow \mathbb{R}^n$ is differentiable and has a differentiable inverse $f^{-1} : \mathbb{R}^n \rightarrow \mathbb{R}^n$. Show that $(f^{-1})' (a) = (f' (f^{-1}(a)))^{-1}$.
\end{problem}
\begin{proof}
Note that $f \circ f^{-1}(x) = x$. Differentiating both sides we have $f'(f^{-1}(x))(f^{-1})'(x) = 1$. Dividing gives the result.
\end{proof}

\begin{problem}
A function $f : \mathbb{C} \rightarrow \mathbb{C}$ is complex differentiable at $z_0 \in \mathbb{C}$ if
\[
f'(z_0) = \lim_{z \rightarrow z_0} \frac{f(z) - f(z_0)}{z - z_0}
\]
exists. A function $f$ is analytic on an open set $U \subseteq \mathbb{C}$ if $f$ is differentiable at each point of $U$. Write $f(z) = u(x,y) + iv(x,y)$, where $u,v : \mathbb{R}^2 \rightarrow \mathbb{R}$, and $z = x + iy$.\\
1) Suppose $f$ is analytic on an open set $U \subseteq \mathbb{C}$. Show that $u$ and $v$ are differentiable on $U$ considered as a subset of $\mathbb{R}^2$.\\
2) Suppose $f$ is analytic on an open set $U \subseteq \mathbb{C}$. Show that $\frac{\partial u}{\partial x} = \frac{\partial v}{\partial y}$, and $\frac{\partial u}{\partial y} = -\frac{\partial v}{\partial x}$. These are the Cauchy-Riemann Equations.\\
3) If $U \subseteq \mathbb{C}$ is an open set and $u$ and $v$ are in $C^1(U)$ and satisfy the Cauchy-Riemann Equations, show that $f(z) = u(x,y) + iv(x,y)$ is analytic on $U$.\\
4) Find an example of a function $f : \mathbb{C} \rightarrow \mathbb{C}$ that is differentiable at one point, but not in a neighborhood of that point.
\end{problem}
\begin{proof}
1) Let $z_0 = x_0 + i y_0 \in U$ and consider
\begin{align*}
f'(z_0)
&= \lim_{z \rightarrow z_0} \frac{f(z) - f(z_0)}{z-z_0}\\
&= \lim_{x + iy \rightarrow x_0 + iy_0} \frac{u(x,y) + iv(x,y) - u(x_0,y_0) + iv(x_0,y_0)}{x+iy - x_0+iy_0}\\
&= \lim_{x + iy \rightarrow x_0 + iy_0} \frac{u(x,y) - u(x_0,y_0)}{x+iy - x_0+iy_0} + i\lim_{x + iy \rightarrow x_0 + iy_0} \frac{v(x,y) - v(x_0, y_0)}{x+iy - x_0+iy_0}.
\end{align*}
In $\mathbb{R}^2$ these last two terms correspond to
\[
\lim_{(x,y) \rightarrow (x_0,y_0)} \frac{u(x,y) - u(x_0,y_0)}{(x,y) - (x_0,y_0)}
\]
and
\[
\lim_{(x,y) \rightarrow (x_0,y_0)} \frac{v(x,y) - v(x_0,y_0)}{(x,y) - (x_0,y_0)}.
\]
Since these two limits exist, $u$ and $v$ are differentiable functions in $\mathbb{R}^2$.\newline

2) We have
\[
Df = \frac{\partial f}{\partial x} \frac{\partial x}{\partial z} + \frac{\partial f}{\partial y} \frac{\partial y}{\partial z} = \frac{1}{2} \left ( \frac{\partial f}{\partial x} - i \frac{\partial f}{\partial y} \right ).
\]
Substituting for $f(x+iy) = u(x,y) + iv(x,y)$ we have
\[
Df = \frac{1}{2} \left ( \left ( \frac{\partial u}{\partial x} + i \frac{\partial v}{\partial x} \right ) - i \left (\frac{\partial u}{\partial y} + i \frac{\partial v}{\partial y} \right ) \right ) = \frac{1}{2} \left ( \left ( \frac{\partial u}{\partial x} + i \frac{\partial v}{\partial x} \right ) + \left (-i\frac{\partial u}{\partial y} +  \frac{\partial v}{\partial y} \right ) \right ).
\]
Along the real axis $\partial f/\partial y = 0$, thus
\[
Df = \frac{1}{2} \left ( \frac{\partial u}{\partial x} + i \frac{\partial v}{\partial x} \right ).
\]
Along the imaginary axis $\partial f/\partial x = 0$, thus
\[
Df = \frac{1}{2} \left ( -i \frac{\partial u}{\partial y} + \frac{\partial v}{\partial y} \right ).
\]
The value of the derivative must be the same in so
\[
\frac{\partial u}{\partial x} = \frac{\partial v}{\partial y}
\]
and
\[
\frac{\partial u}{\partial y} = -\frac{\partial v}{\partial x}.
\]\newline

3) Given that $u$ and $v$ satisfy the Cauchy-Riemann equations, then we must have
\[
\frac{1}{2} \left ( \left ( \frac{\partial u}{\partial x} - \frac{\partial v}{\partial y} \right ) + i \left (\frac{\partial u}{\partial y} +  \frac{\partial v}{\partial x} \right ) \right ) = \frac{1}{2} \left ( \left ( \frac{\partial u}{\partial x} + i \frac{\partial v}{\partial x} \right ) + i \left (\frac{\partial u}{\partial y} + i\frac{\partial v}{\partial y} \right ) \right ) = \frac{1}{2} \left ( \frac{\partial f}{\partial x} + i \frac{\partial f}{\partial y} \right ) = \frac{d f}{d \overline{z}}.
\]
But then this directly implies the differentiability of $f$ since the conjugate function is continuous.\newline

4) Define $f(z) = x^2 + y^2 + ixy$ for $z = x + iy$. Then the Cauchy-Riemann equations are satisfied only at the origin. Thus, $f$ is differentiable at the origin, but not in any neighborhood of it.
\end{proof}

\begin{problem}
Define $f : \mathbb{C} \rightarrow \mathbb{C}$, $f(z) = e^z$ as follows: $f(z) = f(x+iy) = e^x \cos y + ie^x \sin y$. Show that $f$ is analytic on $\mathbb{C}$.
\end{problem}
\begin{proof}
We define $u(x,y) = e^x \cos y$ and $v(x,y) = e^x \sin y$. Note that
\[
\frac{\partial u}{\partial x} = e^x \cos y = \frac{\partial v}{\partial y}
\]
and
\[
\frac{\partial u}{\partial y} = -e^x \sin y = -\frac{\partial v}{\partial x}.
\]
By Part 3) of Problem 8 we see that $f$ is analytic on $\mathbb{C}$.
\end{proof}

\begin{problem}
Let $z_0 \in \mathbb{C}$ and define $f : \mathbb{C} \rightarrow \mathbb{C}$ by $f(z) = \sum_{n=0}^{\infty} a_n (z-z_0)^n$, where $a_n \in \mathbb{C}$ for all $n$. Let $r > 0$ be the radius of convergence of this power series.\\
1) Show that $f(z)$ is analytic on $B_r(z_0) = \{z \in \mathbb{C} \mid |z-z_0| < r\}$.\\
2) Show that the radius of convergence of the power series for $f'(z)$ is equal to $r$.
\end{problem}
\begin{proof}
1) Within the radius of convergence we can write
\[
f'(z) = \sum_{n=0}^{\infty} n a_n (z - z_0)^{n-1}
\]
which represents the term by term differentiation of $f(z)$.\newline

2) We have $r = 1/\limsup{n \rightarrow \infty} |a_n|^{1/n}$. The series $\sum_{n=0}^{\infty} n a_n x^{n-1}$ will converge when the series $\sum_{n=0}^{\infty} n a_n x^n$ converges. Now consider $\limsup_{n \rightarrow \infty} |na_n|^{1/n} = \limsup_{n \rightarrow \infty} n^{1/n} |a_n|^{1/n} = \limsup_{n \rightarrow \infty} |a_n|^{1/n}$. Thus the radius of convergence of this series is the same as that of $f(z)$.
\end{proof}

\begin{problem}
Let $f : \mathbb{R}^2 \rightarrow \mathbb{R}$ be defined by $f(x,y) = \sqrt{|x| + |y|}$. Find those points in $\mathbb{R}^2$ at which $f$ is differentiable.
\end{problem}
\begin{proof}
We have $f$ is differentiable at all points such that $x \neq 0$ and $y \neq 0$. Suppose that $x = 0$. Then we have
\[
f'(x,y) = \lim_{h \rightarrow 0} \frac{|\sqrt{|y+h_2|} - \sqrt{|y|}}{\sqrt{h_1^2 + h_2^2}}.
\]
Based on the powers of the numerator and the denominator, we see that this limit doesn't exist. A similar case holds for $y = 0$.
\end{proof}

\begin{problem}
Let $f : \mathbb{R}^n \rightarrow \mathbb{R}$ be a function such that $|f(x)| \leq ||x||^{\alpha}$ for some $\alpha > 1$. Show that $f$ is differentiable at $0$.
\end{problem}
\begin{proof}
For $x = 0$ we have
\[
f'(x) = \lim_{h \rightarrow 0} \frac{|f(x + h) - f(x)|}{|h|} \leq \lim_{h \rightarrow 0} \frac{||h||^{\alpha}}{||h||}.
\]
Since $\alpha$ is strictly greater than $0$, this limit goes to $0$ and so $f$ is differentiable at $0$.
\end{proof}

\begin{problem}
Let $f : \mathbb{R}^n \times \mathbb{R}^n \rightarrow \mathbb{R}$ be defined by $f(x,y) = x \cdot y$.\\
1) Show that $f$ is differentiable on $\mathbb{R}^n \times \mathbb{R}^n$.\\
2) Show that $Df(a,b)(x,y) = ay + bx$.
\end{problem}
\begin{proof}
Both parts follow from Problems 3 and 4.
\end{proof}

\end{flushleft}
\end{document}