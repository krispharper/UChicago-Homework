\documentclass{article}
\usepackage{amsmath,amssymb,amsfonts,amsthm,fullpage}

\newtheorem{**}{** Problem}
\newtheorem{problem}{Problem}
\newtheorem{lemma}{Lemma}

\begin{document}
\begin{flushright}
Kris Harper\\

MATH 20900\\

May 11, 2009
\end{flushright}

\begin{center}
Homework 6
\end{center}

\begin{**}
Let $(f_n)$ be an increasing sequence of nonnegative, measurable functions on $X$ and $f$ a function on $X$ such that
\[
f(x) = \lim_{n \rightarrow \infty} f_n(x).
\]
Then $f$ is measurable and
\[
\lim_{n \rightarrow \infty} \int_X f_n d\mu = \int_X f d\mu.
\]
\begin{proof}
Because $(f_n)$ is monotonically increasing, we know that $f(x) = \sup_n f_n(x)$ and thus $f$ is measurable since $f_n$ is measurable for all $n$. Also, since $f \geq f_n$ for all $n$ by monotonicity of the Lebesgue integral we immediately have
\[
\sup_n \int_X f_n d\mu \leq \int_X f d\mu.
\]
Let $S$ be the set of all simple functions on $X$ such that $0 \leq s \leq f$. For $\alpha < 1$ and $s \in S$ define
\[
E_n = \{x \in X \mid f_n(x) \geq \alpha s(x)\}.
\]
Note that $E_n$ is measurable and $E_n \subseteq E_{n+1}$. Additionally, $\bigcup_{n} E_n = X$ since $\lim_{n \rightarrow \infty} f_n(x) = f(x) \geq s(x) > \alpha s(x)$. Furthermore
\[
\int_X f_n d\mu \geq \int_{E_n} f_n d\mu \geq \alpha \int_{E_n} s d\mu.
\]
We now use the measure $\nu(E) = \int_X s d\mu$ to obtain
\[
\sup_n \int_X f_n d\mu \geq \alpha \int_X s d\mu.
\]
Note that this inequality is true for every $\alpha < 1$ and every $s \leq f$. Then we have
\[
\sup_n \int_X f_n d\mu \leq \sup_{s,\alpha} \int_X s d\mu = \int_X f d\mu.
\]
Since both inequalities are satisfied, the proof is complete.
\end{proof}
\end{**}

\begin{**}
Suppose $f$ and $g$ are measurable. Then $f+g$ is measurable and if $f$ and $g$ are nonnegative then
\[
\int_X (f+g) d\mu = \int_X f d \mu + \int_X g d\mu.
\]
\begin{proof}
We know that $f+g$ is measurable because
\[
\{x \in X \mid (f+g)(x) \leq a, a \in \mathbb{R}\} = (\{f = -\infty, g \neq \infty\} \cup \{f \neq \infty, g = -\infty\}) \cup (\{x \in X \mid f(x) < a - g(x), a \in \mathbb{R}\} \cap \{f \neq \pm \infty, g \neq \pm \infty\})
\]
and we can pick a rational number between $f$ and $a - g$. Now we have
\[
\int_X (f+g) d\mu = \sup_{s \leq f+g} \int_X s d\mu = \sup_{s \leq f, t \leq g} \int_X (s+t) d\mu.
\]
Let $\alpha_i$ and $\beta_i$ be constants and $A_i$ and $B_i$ sets such that $s(x) = \sum_{i=1}^{k} \alpha_i \chi_{A_i}$ and similarly for $t(x)$. Then if $E_ij = A_i \cap B_j$ we have
\[
\int_{E_{ij}} (s+t) d\mu = (\alpha_i + \beta_j) \mu(E_{ij})
\]
and
\[
\int_{E_{ij}} s d\mu + \int_{E_{ij}} t d\mu = \alpha_i \mu(E_{ij}) + \beta_i \mu(E_{ij}).
\]
Thus, the statement holds for the sets $E_{ij}$. But note that $E_{ij}$ is a disjoint union of $X$. Finally note that we can use the $\nu(E_{ij})$ metric so that the final statement holds for all simple functions. Then since $f$ and $g$ are simply supremums of the integrals of simple functions, we have the desired result.
\end{proof}
\end{**}

\end{document}