\documentclass{article}
\usepackage{amsmath,amssymb,amsfonts,amsthm,fullpage}

\newtheorem{**}{** Problem}
\newtheorem{problem}{Problem}
\newtheorem{lemma}{Lemma}

\begin{document}
\begin{flushright}
Kris Harper\\

MATH 20900\\

May 25, 2009
\end{flushright}

\begin{center}
Homework 8
\end{center}

\begin{**}
If $g \in N$, show that
\[
\int_{\mathbb{R}^n} f(g \cdot x) dx = \int_{\mathbb{R}^n} f(x) dx.
\]
\begin{proof}
Let
\[
\left (
\begin{array}{cccc}
1 & a_{12} & \dots & a_{1n}\\
0 & 1 & \dots & a_{2n}\\
\vdots &&& \vdots\\
0 & a_{n2} & \dots & 1
\end{array}
\right ).
\]
Then if $x = (x_1, x_2, \dots , x_n)^T$ we have
\[
g \cdot x=
\left (
\begin{array}{c}
x_1 + a_{12} x_2 + \dots + a_{1n} x_n\\
x_2 + a_{23} x_3 + \dots + a_{2n} x_n\\
\dots \\
x_n
\end{array}
\right ).
\]
Thus, $x_i$ will always appear without a multiple in the $i$th row of $g \cdot x$. Recall $dx_i$ is translation invariant. Since we integrate component-wise, and all other terms in this row are constants, we have the desired result.
\end{proof}
\end{**}

\begin{**}
What is the Haar measure on $\mathbb{C}^{\times} = \mathbb{T} \times \mathbb{R}_+^{\times}$?
\begin{proof}
We know that the Haar measure on $\mathbb{R}_+^{\times}$ is $dx/x$. Since we integrate component wise, we only need to find the Haar measure on $\mathbb{T}$ and then the product of the measures will suffice. Let $f$ be a function on $\mathbb{T}$. Then for a constant $a \in \mathbb{T}$ we have
\[
\int_{\mathbb{T}} f(ax) e^{i \theta} dx = \frac{1}{a} \int_{\mathbb{T}} f(ax) a e^{i \theta} dx = \int_{\mathbb{T}} f(x) e^{i \theta} dx.
\]
The desired measure is then $e^{i \theta} dxdy/x$.
\end{proof}
\end{**}

\begin{**}
For $\mathbb{C}$ and $\alpha \in \mathbb{C}$ what is $d(\alpha z)$?
\begin{proof}
Treat $\mathbb{C}$ as $\mathbb{R}^2$. Then $dz = dxdy$. Moreover, since we're only concerned with scaling areas in $\mathbb{R}^2$, we can consider only $|\alpha|$. Then we have
\[
d(\alpha z) = d(|\alpha|x) d(|\alpha| y) = |\alpha|^2 dxdy = |\alpha|^2 dz.
\]
\end{proof}
\end{**}

\begin{**}
Show that every character is unitary over the group $(\mathbb{Q}_p, +)$.
\begin{proof}
Let $\chi : \mathbb{Q}_p \rightarrow \mathbb{C}^{\times}$ be an additive character and let $a \in \mathbb{Q}_p$. Note that by $p$-adic expansion, $a = \sum_{k = \gamma}^{\infty} a_k p^k$ where $|a|_p = p^{-\gamma}$ and $0 \leq a_k \leq p-1$. Then we have
\[
\chi(a) = \chi \left (\sum_{k = \gamma}^{\infty} a_k p^k \right ) = \prod_{k = \gamma}^{\infty} \chi(a_kp^k) = \prod_{k = \gamma}^{-1} e^{2 \pi i a_k p^k}.
\]
The last result uses the tail of $a$, $\sum_{k = \gamma}^{-1} a_k p^k$. Note that $|\chi(a)| = 1$ by Euler's identity.
\end{proof}
\end{**}

\begin{**}
Show $\hat{G}$ is a group.
\begin{proof}
Let $\chi_1, \chi_2, \chi_3 \in \hat{G}$. Then, since associativity holds in $\mathbb{C}$, we have
\[
((\chi_1\chi_2)\chi_3)(x) = (\chi_1\chi_2)(x)\chi_3(x) = (\chi_1(x) \chi_2(x)) \chi_3(x) = \chi_1(x)(\chi_2(x)\chi_3(x)) = \chi_1(x)(\chi_2\chi_3(x)) = (\chi_1(\chi_2\chi_3))(x).
\]\newline
Now consider the $1$ function which sends all values to $1$. Then for $\chi \in \hat{G}$, $(1 \cdot \chi)(x) = 1 \chi(x) = \chi(x)$.\newline
Finally, for $\chi \in \hat{G}$, let $\chi^{-1} : G \mathbb{C}^{\times}$ such that $\chi^{-1}(x) = (\chi(x))^{-1}$. The fact that this map is continuous follows from the fact that $\chi(x) \neq 0$. To show $\chi^{-1}$ is a homomorphism, let $x, y \in G$. Then
\[
\chi^{-1}(xy) = (\chi(xy))^{-1} = (\chi(x)\chi(y))^{-1} = (\chi(x))^{-1} (\chi(y))^{-1} = \chi^{-1}(x) \chi^{-1}(y).
\]
Since associativity, identity and inverses hold, $\hat{G}$ is a group.
\end{proof}
\end{**}

\begin{**}
Read about the compact-open topology.
\begin{proof}
For the case of $\hat{G}$, let $K$ be a compact subset of $G$ and $U$ be an open subset of $\mathbb{C}$. If $V(K, U) \subseteq \hat{G}$ is the set of all elements of $\hat{G}$ such that $\chi(K) \subseteq U$, then $V(K,U)$ is a subbase for the compact-open topology.
\end{proof}
\end{**}

\begin{**}
Suppose $G = \mathbb{T}$ and $f \in L^1 (G)$ such that $f (e^{i \theta}) = 1$. Find $\hat{f}$.
\begin{proof}
By definition we know
\[
\hat{f} (\chi) = \int_{\mathbb{T}} f(x) \overline{\chi(x)} dx
\]
and since $f(e^{i \theta}) = 1$, this simplifies to
\[
\hat{f} (\chi) = \int_{\mathbb{T}} \overline{\chi(x)} dx.
\]
Since $|\chi(x)| = 1$ we can write $\chi(x) = e^{i(x+a)}$ for some constant $a \in \mathbb{R}$. Then we have
\[
\hat{f} = \int_{\mathbb{T}} e^{-i(x+a)}dx = \int_0^{2 \pi} e^{-i(x+a)}dx = 0.
\]
\end{proof}
\end{**}

\end{document}